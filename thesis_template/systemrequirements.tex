\chapter{System Requirements}\label{ch:system-requirements}

\section{Overview}

Due to the flexibility of CGKA in combination with key regression, 
the shared folder scheme can cater to a variety of different requirements and threat models.
This is part of the goal of the theoretical side of this
project: ideally, we wish do develop a set of systems which, depending on the strength of the adversary,
provides different levels of security. While all the systems in the set should share the same basic building
blocks, they might behave rather differently on an implementation level. Hence, the implementation
needs to restrict the setting such that the system requirements are clear.

In Section \cref{sc:mentalmodel} we introduce the abstract model of a Secure File Sharing construction (SSF).

In Section \cref{sc:threatmodel}, we describe the threat models that were considered in the theoretical side of this project
and we motivate the choice we made for the implementation. 

In Section \cref{sc:securityguarantees}, we introduce an informal description of the security guarantees 
that the implemented system aims to provide.

In Section \cref{sc:functionalrequirements}, we reason around the functionalities that the system should support in a real world setting.
More specifically, we aim to the develop a Minimal Viable Product (MVP).

\section{The Mental Model of Secure File Sharing} \label{sc:mentalmodel}

A Secure File Sharing (SSF) system aims to provide users, organised in groups, 
with the ability to share contents in within the group in a secure way.
Users are identified through a Public Key Infrastructure (PKI).

Borrowing the terminology and mental model from well-known cloud storage providers offering collaborative file sharing
~\footnote{Dropbox, Google Drive, One Drive etc.},
we introduce the concept of a ``shared folder'', or simply a folder in the system.
A folder contains one or more files, and its contents are accessible to a group of users sharing this contents.
To this end, the users need to agree on a shared secret that is used to protect the contents through cryptographic means.
The actual storage space is outsourced to a public cloud provider.

In addition, the group composition is dynamic, meaning that the set of users which has granted access to a folder can change over time.
On the theoretical side of this project, the research explored the applicability of Continuous Key Group Agreement schemes,
normally used in messaging applications, in order to efficiently handle a shared secret within a dynamic group of users.

\section{Threat Model} \label{sc:threatmodel}


\section{Security Guarantees} \label{sc:securityguarantees}

\section{Functional Requirements} \label{sc:functionalrequirements}

In industry terminology a first implementation which can be launched as a product is called Minimal Viable Product (MVP).
This project aims to explore such a real-world scenario, instead of developing a prototype implementation of the cryptographic construction.
To this end, we need to define functional requirements which a real-world user will want to consider the product usable, formulating them in a similar fashion to User Stories:
\begin{itemize}
    \item Accessibility: as a user I want my contents to be always accessible until deletion, possibly from any device I own capable to connect to the cloud provider.
    \item Efficiency: as a user I want to retrieve my contents efficiently, as well as uploading or sharing.
    \item Visibility of System Status: as a user I want to be able to retrieve the set of folders I have access to, list the contents of each of them and get feedback on operations I perform.
\end{itemize}

In terms of supported operations, we want our system to provide the following minimal set of operations:
\begin{itemize}
    \item Create a folder: a user can create a folder, the content will not be available to other users. Also, an honest system should not let other users discover the presence of this folder.
    \item List folders: a user should be able to see all folders he/she has access to.
    \item Uploading files to the folder: a user can upload files to folders he/she has access to.
    \item Downloading files from a folder: a user can download files from folders he/she has access to.
    \item Share a folder: a user should be able to share a folder with other users.
    \item List users currently registered in the system: a user should be able to see who has a service account, thus being able to participate in a shared folder.
\end{itemize}


