\chapter{Backround}\label{ch:background}

Before describing the implementation details, we introduce the theoretical
background for the project, especially regarding the Secure Shared Folder 
(SSF) scheme. We will introduce some notation (\cref{sc:notation}) and
describe existing cryptographic primitives (\cref{sc:SSKG}, \cref{sc:CGKA})
that are used in the scheme.
We will also give an informal description of the SSF scheme and its operations,
together with the threat model we refer to (\cref{sc:SSF}).
 
\section{Notation}\label{sc:notation}

\nd{TODO: once we defined some algorithms or anything, insert the notation here}

\section{Seekable Sequential Key Generators}\label{sc:SSKG}

Seekable Sequential Key Generators (SSKG) are cryptographic objects introduced by Marson et al.~\cite{ESORICS:MarPoe13}.
These are sequential pseudo random generators (SRG).
A sequential random generator is a stateful pseudo random generator (PRG), 
which outputs a fixed-length string for each invocation, 
thus producing a sequence of pseudo random strings in multiple invocations.
The security property mandates indistinguishability of the output from a uniformely random sampled string.
SSKG are SRG with the additional property of being seekable (through the \texttt{Seek} algorithm), 
which allows for a sublinear computation when attempting 
to retrieve an output of an arbitraty invocation in the future (non-subsequent)
from the starting state.
They are widely used in practice:
the logging service of the \texttt{systemd} system manager,
which is a core component in many Linux-based operating systems,
is using them to power fast verification of arbitraty log entries.
SSKG are provably (forward-)secure in the standard model.
In our implementation we will be using the latest tree-based version of SSKG~\cite{ESORICS:MarPoe14}.
We discuss the practical details more in-depth in \cref{sc:ssf-sskg}.

\section{Continuous Group Key Agreement}\label{sc:CGKA}

A continuous group key agreement (CGKA) scheme~\cite{C:ACDT20} 

\section{Secure Shared Folder}\label{sc:SSF}

\subsection{Novel Security Notions for Persistent Data}

\subsection{Double Key Regression}\label{ssc:DKR}

\subsection{The Scheme}

\section{Related Work}

