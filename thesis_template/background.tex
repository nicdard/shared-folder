\chapter{Backround}\label{ch:background}

Before describing the implementation details, we introduce the theoretical
background for the project, especially regarding the Secure Shared Folder 
(SSF) scheme. We will introduce some notation (\cref{sc:notation}) and
describe existing cryptographic primitives (\cref{sc:SSKG}, \cref{sc:CGKA})
that are used in the scheme.
We will also give an informal description of the SSF scheme and its operations,
together with the threat model we refer to (\cref{sc:SSF}).

\section{Notation}\label{sc:notation}

\nd{TODO: once we defined some algorithms or anything, insert the notation here}

\section{Seekable Sequential Key Generators}\label{sc:SSKG}

Seekable Sequential Key Generators (SSKG) are cryptographic objects introduced by Marson et al.~\cite{ESORICS:MarPoe13}.
These are sequential pseudo random generators (SRG).
A sequential random generator is a stateful pseudo random generator (PRG), 
which outputs a fixed-length string for each invocation, 
thus producing a sequence of pseudo random strings in multiple invocations.
The security property mandates indistinguishability of the output from a uniformely random sampled string.
SSKG are SRG with the additional property of being seekable (through the \texttt{Seek} algorithm), 
which allows for a sublinear computation when attempting 
to retrieve an output of an arbitraty invocation in the future (non-subsequent)
from the starting state.
They are widely used in practice:
the logging service of the \texttt{systemd} system manager,
which is a core component in many Linux-based operating systems,
is using them to power fast verification of arbitraty log entries.
SSKG are provably (forward-)secure in the standard model.
In our implementation we will be using the latest tree-based version of SSKG~\cite{ESORICS:MarPoe14}.
We discuss the practical details more in-depth in \cref{sc:ssf-sskg}.

\section{Continuous Group Key Agreement}\label{sc:CGKA}

A continuous group key agreement (CGKA) scheme~\cite{C:ACDT20}
allows a long-lived, dynamic, asynchronous group of users to agree 
continuously on shared symmetric secrets.
CKGA allow a group of \texttt{n} users to perform critical 
operations in \texttt{log(n)} time,
thus it can be used in practice with a large number of members.
The primitive guarantees post-compromise security (PCS) and forward secrecy (FS).

A first E2E primitive for an asynchronous group key exchange 
has been introduced by Cohn-Gordon et al.~\cite{CCS:CCGMM18}.
The authors present a way to achieve PCS in an asynchronous group messaging system. 
To this end they design the Asynchronous Ratcheting Trees (ART),
which is internally using a Diffie-Hellman tree to organise the
public and private secrets for each users.
The primitive has been of significant interest in the industry and
got a dedicated working groups by the IETF.
The ART construction was later replaced by the TreeKEM
construction proposed by Bhargavan et al.~\cite{TreeKEM}.
In TreeKEM, members are organised in a tree-shaped structure
as in ART, however the keys stored in the tree for each group
can be any keypair supporting key encapsulation (KEM).
The key advantage of TreeKEM over ART is that
most operations are ``mergeable'':
any device receiving two
concurrent operations will be able to process and execute both of them,
instead of refusing one of them.
Alwen et al.~\cite{C:ACDT20} slightly modify the TreeKEM construction
to achieve \textit{optimal} FS. Further refinements for efficiency
and security have been studied in recent years~\cite{CCS:ACDT21}~\cite{EC:AANKPPW22}~\cite{C:AlwMulTse23}.

Of particular interest is the family of CGKA protocols called admin-CGKA (A-CGKA).
In this declination of CGKA, a subgroup of the users are admins. 
Admins can perform additional operations that are otherwise disallowed, such as removing another member of the group.
The SSF scheme is based on a version of A-CGKA using a CGKA within CGKA to manage the admin state, as proposed by
B{\'a}lbas et. al.~\cite{USENIX:BalColVau23}.

Most of the research on CGKA is part of the messaging layer security (MLS) protocol from IETF.
In the implementation, we indeed make use of a library implementing MLS and not just CGKA.
We discuss in further details in \cref{ch:ssf}.

\section{Secure Shared Folder}\label{sc:SSF}

\subsection{Novel Security Notions for Persistent Data}

\subsection{Double Key Regression}\label{ssc:DKR}

\subsection{The Scheme}



