\chapter{Engineering Gaps}\label{ch:gaps}



\section{On the Type Safety of Opaque Types}\label{sc:gap-type-safety-of-opaque-byte-arrays}
The attentive reader must have realised that most of the data either in the state
or in the message exchange in our implementations is represented as opaque \texttt{Uint8Array}.
We point specifically to the implementation of the cryptographic primitives:
\cref{sc:ssf-sskg}, \cref{sc:DKR-implementation}, \cref{sc:js-bindings-for-mls}, \cref{sc:CGKA-implementations}.

This is usually the case for cryptographic heavy software written in any language, 
and can easily cause bugs in the implementation, as the type checker
would gladly accept any \texttt{Uint8Array} passed to a function, or as object 
property and so on even if the contents are meant for a different usage. 
During the implementation we encountered this problem multiple times,
where calling a function with the wrong order of the parameter, where
multiple \texttt{Uint8Array} are passed as arguments, can cause bugs
very hard to fix without long debugging sessions. Note that it can be
really easy to make such an error, especially while writing thousands
of lines of code and, in this case, without anyone else reviewing the code.
We believe the discussion of this practical issue can be of great interest for other
engineers encountering this problem, as well as it highlights yet another
difficulty in writing and deploying correct cryptographic software.

The na\"ive solution currently adopted in the code (for time constraints reasons)
is to always use a wrapper object containing all the parameters of the function
as properties. In this way, we need to give a name to each parameter
in the function declaration as well as to the arguments at function calling
time, as the object needs to be constructed. This will make clear which
opaque byte array is passed to which argument, and clarify the intent.
With this pattern is also possible to perform runtime validation
while creating the object with the arguments if required.
We highlight that this methodology also helps in a team effort, to make
code reviews easier. Further, this solution applies as well to JavaScript,
as it does rely on plain JS objects to convey the semantic. It is also
easy to translate this technique to any other language supporting objects.
The downside however, is that at runtime an object is created, which adds 
an overhead in terms of memory allocation.

A TypeScript-specific solution which avoids the runtime overhead, is to use
so-called ``branded types''~\cite{vanderkam2019effective, goldberg2022learning}. 
With branded types, we can construct two 
different types of the same underlying \texttt{Uint8Array} only
at compile time. After the code is type-checked and transpiled down to JS,
we will just obtain the same code we would have written without branded types.
Branded types are thus type-safe opaque types. We leave an in-depth explanation
of the technique to the cited literature. We plan to refactor the code base
to use branded types to better specify the semantic of primitive types,
like \texttt{Uint8Array}, \texttt{string} (for example storing PEM certificates),
\texttt{number} (positive numbers, integers, etc.).

Type-safety is a strong requirement when writing
cryptographic software, and we want to point out that especially for cryptographic
software, the semantic of the data should be conveyed by the type,
to enhance readability, maintainability and reduce the number of
bugs in fairly complex crypto systems.