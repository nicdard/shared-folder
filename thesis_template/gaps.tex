\chapter{Engineering Gaps}\label{ch:gaps}

This chapter summarise the engineering gaps (\cref{sc:summary-of-contributions})
we encountered during the implementation of the system,
both for the baseline and the SSF scheme implementation.
The engineering gaps highlight the practical issues
and differences between the theoretical model and 
the real-world implementation.

For each gap, we refer to the relevant sections to guide the reader
to the details given in the overall content of this document.
Looking in retrospective, these gaps are main starting
points for further reflections on the current
state of real-world cryptography and lesson learnt.

We close the chapter by presenting the future work (\cref{sc:future-work}).

\section{On the Cryptographic Ecosystem of the Browser}\label{sc:gap-webcrypto-api}

The Web Crypto API (\cref{sc:webcrypto-api})
has been cited during this work
multiple times. 
The API has many limitations 
(\cref{sc:baseline-protocol}, \cref{sc:ssf-sskg}, \cref{sc:ssf-double-prf}, \cref{sc:Web-Crypto-API-implementations:-non-standard-behaviours}),
when compared to the cryptographic libraries available
in other languages and runtimes.
Writing the same code just for a Desktop application
would have been way easier.
Although the API design tries to avoid developers from
wrongly using cryptographic primitives, it lacks
the flexibility to allow advanced use cases.

We think this API should
be enhanced to provide more building blocks
for advanced use cases, to allow high quality
cryptography to be shipped in the browser to
end users. We claim this is not a small issue,
as most of today services are nowadays accessible
from the browser and security is a raising concern.
While the cryptographic research is advancing and new
schemes are proposed, those advancements remain hard to
deploy in browser runtimes, where they can 
protect millions of users.

\section{On Code Portability and Heterogeneous Device Capabilities}\label{sc:gap-code-portability}

In this work, we have also tried to showcase the
portability of the cryptographic primitives
between browsers and desktop runtimes.
We found Node.js to be an alternative runtime supporting
the same Cryptographic APIs as the browser, although
with some differences (\cref{sc:Web-Crypto-API-implementations:-non-standard-behaviours}).

We have used WebAssembly to have access to
library code otherwise unavailable
in the browser, like the mls-rs library
(\cref{sc:CGKA-implementations}). 

WebAssembly is a new raising technology, which could
allow for more portability and reuse of the code on
heterogeneous platforms and devices. 
If cryptographic primitives
are integrated in the WebAssembly runtime,
through dedicated instructions guaranteeing
certain requirements for the security of cryptographic
code (\cref{sc:abstract-to-real}, (\cref{sc:browser-runtimes}, \cref{sc:webcrypto-api})),
code portability issues could be mitigated.
While in the theoretical model the primitives are
mathematical objects with defined properties,
these properties might not hold at runtime in the
implementation, especially across different runtime,
implementations and devices.
Wasm could be an abstraction layer providing the
same cryptographic primitives across different
runtimes.

\section{On the Design and Implementation Efforts of Cryptography}\label{sc:gap-crypto-primitives-design-implementation}

Designing cryptographic primitives is a complex task,
which involves proving the security of the scheme. The prototyping phase, where ideas are gathered, 
and the scheme is designed is usually done through 
discussions and whiteboard sessions.

The implementation of such ideas can instead take months.
It requires setting up all the development environment,
required dependencies (\cref{ch:setup}), and writing thousands of lines of code.
The final version of this project 
is more than 18600 lines of code, excluding code that
is generated through our tooling (\cref{sc:PKI}, \cref{sc:client-overview}), 
accounting for around other 6000 lines,
for a total of more than 25000 lines of code.
The complete git repository, including
also package manager files and documentation, 
is more than 53000 lines.
When we compare with the pseudocode provided in the manuscripts,
it is clear that the implementation effort, especially
when targeting a real-world scenario, requires much
more effort. 

Also note that every change requires
changing many lines of code. Many parts of the code
were rewritten during the implementation to try out
different approaches or because changes were made to the
design of the primitives (see also \cref{sc:collaboration-crypto-se}). 
Ability to prototype with
code requires a lot of engineering effort and
expertise. Normally indeed, only proof-of-concept
implementations are created in research,
distancing the design of new primitives from their
real-world applications.

Primitives targeting complex systems with complex interactions
between different components, like the GKP scheme (\cref{sc:gkp-scheme}), are the
ones most likely to suffer from this gap between design
and implementation.

\subsection{Implementing Assumptions}\label{sc:implementing-assumptions}

An important aspect of the effort required to implement
cryptographic primitives is the need to implement also
all the dependencies, which are normally assumed to exist
in the mathematical model. For example, a PKI is normally
assumed to handle identities, but we needed to construct a server for it (\cref{sc:PKI}).
Another such example is the delivery service for MLS and
GKP, which is just described by its properties in both schemes,
requiring a server to be implemented as well (\cref{ssc:delivery-service}).

\section{On the Non-Cryptographic Guarantees}\label{sc:gap-non-crypto-guarantees}

While the security proofs of GKP~\cite{GKP} consider availability out of
scope, the real-world expectation for the SSF scheme
is that the system should be
available, and the persisted data should be present in 
a folder until deletion is ordered by a user with
legitimate access.

In the implementation this is all handled by the SSF Proxy server
(\cref{sc:ssf-proxy-server}).
The server assure authentication and access control
to the data, thus for example protecting from actors 
external to the shared folder to delete its files (\cref{sc:cloud-storage-access-and-billing}). 

\section{On Client Execution and State Management}\label{sc:gap-execution-multi-tenancy-state-management}

The description of the operations in the cryptographic
constructions are usually abstracting away many details,
in particular related to the state management.
The pseudocode of D[F, S] and GRaPPA~\cite{GKP} (\cref{sc:background-generalised-DKR}, \cref{sc:gkp-scheme}) 
does not specify how the state of the client is persisted 
and how the client can recover from a crash or transient error.
The procedures of the protocols are usually described as
mathematical functions, thus, not holding any state across
invocations.

In our implementation we have to deal with the above issues,
to guarantee that the client state is always correct,
so that if an operation is started is either completed
or rolled back to the previous correct state.
Borrowing from database terminology, we say that the client
state must be handled ``transactionally'', to ensure that
a client will always be able to continue updating its state
correctly and consistently.

We remind the reader that some stateful cryptographic implementations do not
allow rolling back to a previous state after an operation
that changes the internal state has been executed (\cref{sc:js-bindings-for-mls}).
This creates issues that can only be solved by changing
the details of how the procedures are executed (\cref{sc:state-sync-rollbacks}).

We highlight that the transactional state management problem
is particularly interviewed with the server synchronization
mechanism (\cref{sc:gap-synchronization-server-state}).


\section{On Synchronization and Server State}\label{sc:gap-synchronization-server-state}

Although in the scheme and related constructions the description
does specify that the server should not hold any state,
this is only true in terms of cryptographic state, i.e., for
the key material.

The SSF Gateway server provides the clients with a way to order and synchronize
their concurrent operations on the actual shared cloud storage (\cref{sc:ssf-file-changes-sync}).
We remind the reader that 
the server is used also as an MLS/GRaPPA delivery service (\cref{ssc:delivery-service}).
Therefore, in the implementation, for a correct execution of the protocols, 
we need the server
to keep track of a portion of the state of the groups, 
shared-folders and clients (\cref{ssc:delivery-service}, \cref{sc:state-sync-rollbacks}).


\section{On the Abstract Modelling of Cloud Storage}\label{sc:gap-abstract-cloud-storage}

As discussed in \cref{sc:cloud-storage} and more in details in 
\cref{scc:cloud-storage-assumptions},
\cref{sc:cloud-storage-access-and-billing}
and \cref{sc:ssf-proxy-server} the cloud storage modelling
as read/write operations on virtually infinite storage
is simplification that removes multiple practical problems. 
In reality, implementors need to
deal with the details of each cloud provider, the billing attribution
and the choice of the actual technology to use among
the different available options.
Also, a production-ready
implementation should allow for different cloud storage
providers to be used, with a simple configuration change.

The above points required the implementation of the
SSF Gateway server (\cref{sc:ssf-proxy-server}).
The Gateway acts as an abstraction layer between clients
and cloud providers, thus closing the gap
between the abstract model and the real-world providers.

\section{On Performances}\label{sc:gap-performance}

Performance is only studied in the context of the
cryptographic scheme itself, looking at the time and
space complexity of the primitives.
However, since the correct execution of the protocol
could be dependent on server components for the synchronization
of the state, like for the GKP or SSF scheme, the performance
can be greatly affected by the server implementation.

Protocols where multiple clients need to maintain
a global shared state, like the GKP scheme, can especially
suffer this problem. The complexity of the client
execution can become negligible compared to a server
becoming the bottleneck of the entire system.
The design of primitives which do not require 
global state synchronization to advance the local
client state should be preferred
for real-world applications, as they are easier 
and more efficiently scalable.

%We observe that GKP (hence also SSF) suffer from the CAP problem (\cref{sc:ssf-file-changes-sync}).\footnote{The CAP theorem states that it is impossible for a distributed system to simultaneously provide more than two out of the following three guarantees: Consistency, Availability, Partition tolerance.}
%However, this is a latent issue in the scheme,
%as availability is kept out of scope 
%(\cref{sc:gap-non-crypto-guarantees}). 


\section{On the Design of new Cryptographic Primitives: a Feedback Loop}\label{sc:collaboration-crypto-se}

Recalling the discussion of the bugs and the enhancements 
proposals in the implementation of the GRaPPA construction
(\cref{sc:GRaPPA-bugs}, \cref{sc:DKR-enhancements})
we want to stress the importance for
cryptographers to work closely with software engineers
when designing new cryptographic primitives.

First, software engineers can provide a different, more
practical, point of view to the problems the primitive
wants to address, and provide background knowledge
on the current state-of-the-art for the application
domains the primitive can be applied to.
Second, an implementation targeting a real-world scenario
can uncover many issues in the original design, which could be
overlooked in the mathematical model or by a toy implementation.
As in other software, many problems arise only when
we consider the system at scale, or if we want certain
user expectations to be met.

We think that the design of new cryptographic primitives and
the implementation of the constructions in real-world
setting should go hand in hand. 
In our case, the implementation, 
initially guided by the pseudocode 
description, has discovered bugs in the constructions which
ultimately were caused by a gap in the mathematical model.
The enhancements we are proposing are actively discussed
with the authors of the original manuscript, and we hope
to see them included in the next version.
This is a very clear
example, among many others during this work which was conducted 
with continuous back and forward discussions, 
where the implementation details entailed 
an in-depth analysis of the implications on the model and 
assumptions taken in the design of the primitive.
In the end, a feedback loop between cryptographers and engineers 
is established leading to
a better analysis of the primitive and its applications,
and a more robust and secure implementation.


\section{On the Type-Safety of Opaque Types}\label{sc:gap-type-safety-of-opaque-byte-arrays}
The attentive reader must have realised that most of the data either in the state
or in the message exchange in our implementations is represented as opaque \texttt{Uint8Array}.
We point specifically to the implementation of the cryptographic primitives:
\cref{sc:ssf-sskg}, \cref{sc:DKR-implementation}, \cref{sc:js-bindings-for-mls}, \cref{sc:GRaPPA-implementation}.

This is usually the case for cryptographic heavy software written in any language, 
and can easily cause bugs in the implementation, as the type checker
would gladly accept any \texttt{Uint8Array} passed to a function, or as object 
property and so on even if the contents are meant for a different usage. 
During the implementation we encountered this problem multiple times,
where calling a function with the wrong order of the parameter, where
multiple \texttt{Uint8Array} are passed as arguments, can cause bugs
very hard to fix without long debugging sessions. Note that it can be
really easy to make such an error, especially while writing thousands
of lines of code and, in this case, without anyone else reviewing the code.
We believe the discussion of this practical issue can be of great interest for other
engineers encountering this problem, as well as it highlights yet another
difficulty in writing and deploying correct cryptographic software.

The na\"ive solution currently adopted in the code (for time constraints reasons)
is to always use a wrapper object containing all the parameters of the function
as properties. In this way, we need to give a name to each parameter
in the function declaration as well as to the arguments at function-call
time, as the object needs to be constructed. This will make clear which
opaque byte array is passed to which argument, and clarify the intent.
With this pattern is also possible to perform runtime validation
while creating the object with the arguments if required.
We highlight that this methodology also helps in a team effort, to make
code reviews easier. Further, this solution applies as well to JavaScript,
as it does rely on plain JS objects to convey the semantic. It is also
easy to translate this technique to any other language supporting objects.
The downside however, is that at runtime an object is created, which adds 
an overhead in terms of memory allocation.

A TypeScript-specific solution which avoids the runtime overhead, is to use
so-called ``branded types''~\cite{vanderkam2019effective, goldberg2022learning}. 
With branded types, we can construct two 
different types of the same underlying \texttt{Uint8Array} only
at compile time. After the code is type-checked and transpiled down to JS,
we will just obtain the same code we would have written without branded types.
Branded types are thus type-safe opaque types. We leave an in-depth explanation
of the technique to the cited literature. We plan to refactor the code base
to use branded types to better specify the semantic of primitive types,
like \texttt{Uint8Array}, \texttt{string} (for example storing PEM certificates),
\texttt{number} (positive numbers, integers, etc.).

Type-safety is a strong requirement when writing
cryptographic software, and we want to point out that especially for cryptographic
software, the semantic of the data should be conveyed by the type,
to enhance readability, maintainability and reduce the number of
bugs in fairly complex crypto systems.
In the future, we
might hope to see the rise of new programming languages
or extensions of existing ones to enable the native usage
of opaque types with an associated semantic for the reasons 
explained above.


\section{Future Work}\label{sc:future-work}

One of the initial practical goals of this work
was to implement an MVP of our secure folder shared folder
real-world system. At the end of this thesis,
we can say we have \textbf{almost} achieved this goal.
We refer here to the SSF scheme implementation.
As we have seen in the previous chapters,
the current implementation suffers multiple issues,
which we summarise in the following points as future work:
\begin{itemize}
    \item Implement the changes to DKR and GKP after the discussion with the original authors of the primitives is finalised (\cref{sc:DKR-enhancements}).
    \item Maintain the state of the client between multiple user sessions (\cref{CLI}, \cref{sc:js-bindings-for-mls}). We remind that this includes the completetion of two separate tasks:
    \begin{itemize}
        \item Implement a browser-compatible storage layer for the mls-rs library (\cref{sc:MLS-enhancements}).
        \item Write a browser-compatible implementation of \texttt{GKPStorage} (\cref{ssc:GKP-persistent-storage}).
    \end{itemize}
    \item Enhance application message encryption capabilities of mls-rs (\cref{sc:MLS-enhancements}).
    \item Consequently, refactor the resiliency protocol in the GRaPPA implementation, by sending only one control message as in the original design (\cref{sc:state-sync-rollbacks}).
    \item Add X.509 certificate support to the MLS client (\cref{sc:MLS-enhancements}), to substitute the basic credentials with proper PKI support.
    \item Research a workaround to avoid compatibility problems between different browsers' implementations of the Web Crypto API or propose a change of the API specification to the W3C  (\cref{sc:Web-Crypto-API-implementations:-non-standard-behaviours}). We recall that some needed cryptographic operations are currently not supported in all major browsers, specifically Safari and Firefox. 
    \item Optimise the storage of files' private data (\cref{sc:ssf-file-encryption}).
    \item Partially refactor the types in the client implementation to use branded types (\cref{sc:gap-type-safety-of-opaque-byte-arrays}).
\end{itemize}

Further optimisations and enhancements can be done
to the server components, particularly we note that
the PKI server should be replaced with a real CA
implementation. The SSF Gateway server could be optimise
for performance, but this is not a priority of the
next development iteration.

Finally, continue to study the applications of the GKP scheme in
real-world scenarios, and improve the primitives.
