\chapter{Conclusion}\label{ch:conclusion}

In this thesis, we have implemented an MVP of the
Secure Shared Folder (SSF) scheme. We have conducted
the implementation targeting a real-world scenario,
thus uncovering many engineering gaps.
The project result includes around 50000 lines
between code, tests and documentation as well as
contributions to open-source libraries and to the 
DKR and GKP primitives from Backendal, B{\'a}lbas and Haller~\cite{GKP}.

The work has uncovered issues in the cryptographic constructions
and the cryptographic ecosystem of the browsers.
We give a detailed account of the issues and a survey
of the state-of-the-art of the technologies
we considered during the implementation.
We further identified workarounds and solutions
to some of the issues we found, 
allowing us to implement an almost production-ready MVP.
Finally, we demonstrated the required engineering work 
to implement such systems and we contributed many lessons learnt
that can be used in the future.


In \cref{sc:future-work} we detail
the future work that can be done to improve and finalise
the current implementation as well as to continue 
the research underlying the code.


\section{Future Work}\label{sc:future-work}

As we have seen in the previous chapters,
the current implementation suffers multiple issues,
which we summarise in the following points as future work:
\begin{itemize}
    \item Implement the changes to DKR and GKP after the discussion with the original authors of the primitives is finalised (\cref{sc:DKR-enhancements}).
    \item Maintain the state of the client between multiple user sessions (\cref{CLI}, \cref{sc:js-bindings-for-mls}). We remind that this includes the completion of two separate tasks:
    \begin{itemize}
        \item Implement a browser-compatible storage layer for the mls-rs library (\cref{sc:MLS-enhancements}).
        \item Write a browser-compatible implementation of \texttt{GKPStorage} (\cref{ssc:GKP-persistent-storage}).
    \end{itemize}
    \item Enhance application message encryption capabilities of mls-rs with a mechanism to encrypt an application message with the next epoch key when a commit is stashed in the local MLS client state (\cref{sc:MLS-enhancements}). The encrypted application payload could be sent together with the commit message. This would simplify the usage of MLS as a transport layer for GKP messages or any other primitive the requires broadcasting encrypted messages to a dynamic group of users. 
    \item Consequently, refactor the resiliency protocol in the GRaPPA implementation, by sending only one control message as in the original design (\cref{sc:state-sync-rollbacks}).
    \item Add X.509 certificate support to the MLS client (\cref{sc:MLS-enhancements}), to substitute the basic credentials with proper PKI support.
    \item Research a workaround to avoid compatibility problems between different browsers' implementations of the Web Crypto API or propose a change of the API specification to the W3C  (\cref{sc:Web-Crypto-API-implementations:-non-standard-behaviours}). Recall that some needed cryptographic operations are currently not supported in all major browsers, specifically Safari and Firefox. 
    \item Optimise the storage of files' private data by dividing the metadata file by epoch, and store each part separately in the cloud storage (\cref{sc:ssf-file-encryption}).
    \item Partially refactor the types in the client implementation to use branded types (\cref{sc:gap-type-safety-of-opaque-byte-arrays}).
\end{itemize}

Further optimisations and enhancements can be done
to the server components, particularly we note that
the PKI server should be replaced with a real CA
implementation. The SSF Gateway server could be optimised
for performance, but this is not a priority of the
next development iteration, since this component will only be
under stress if a significant number of users
access the same folders simultaneously.

Finally, continue to study the applications of DKR and GKP schemes in
real-world scenarios is itself of independent interest: 
improving the primitives to potentially work without a 
centralised service handling the synchronisation and global ordering 
of the operations could result in faster and easier adoption of such 
key agreement protocols in the area of distributed systems.

