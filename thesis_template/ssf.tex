\chapter{Secure Shared Folder}\label{ch:ssf}



\section{CGKA}

CGKA is a major component in the SSF scheme.
The theoretical construction uses 2 CGKA groups to handle respectively:

\begin{itemize}
    \item Admins: the group of users that have access to all secrets in the key schedule, and can thus cryptographically limit access to new uploaded content to existing users.
    \item Members: the group of users that have only access to secrets that are shared by the admins.
\end{itemize}

The details of the scheme will be given in the following sections.

\section{SSKG}\label{sc:ssf-sskg}

The scheme uses SSKG~\cref{sc:SSKG}, 
however there is no browser-compatible version implementation available\footnote{We found a reference Go implementation~\cite{SSKGGo}, which was used as reference together with the algorithm description in the paper.}
to the best of our knowledge.
We therefore re-implement the pseudo-code in~\cite{ESORICS:MarPoe14} using TypeScript.
The choice on the tree-based implementation is motivated both by its efficiency
compared to the numerical construction and because it only uses common cryptographic
primitives, such as hash functions, PRGs or block ciphers, which are supported by the WebCrypto API.
More importantly for the practical use case, is the added \texttt{SuperSeek}
functionality, which is a generalisation on top of the \texttt{Seek} procedure.
While \texttt{Seek} can only calculate an output starting from the initial state, 
\texttt{SuperSeek} can start the forward derivation from an arbitrary point of the sequence.
This allows the usage of SSKG in double key regression\cref{ssc:DKR},
as we can just share the state of the SSKG in the forward chain from 
the epoch we want to give access to. 
We pay more in terms of additional bytes sent over the wire, as the state is not only the current secret,
but all of the stack (which is in log(n)) \nd{TODO make this better, and use math log}

The subfolder \texttt{ssf-client/src/protocol/sskg/} contains the code for the SSKG module.
The file \texttt{sskg.ts} specify the functionalities exported by an instance of the object.
The file \texttt{TreeSSKg.ts} contains the implementation.
The code provides also utilities for serializing and deserializing the object to persist the state in the client.
To this end, we use the CBOR\nd{TODO: cite CBOR, add a section in setup, as this was a choice made.} standard binary encoding.

We specifically give an HKDF-based implementation of binary-tree SSKG.
The WebCrypto API supports the HKDF functionalities, however,





\section{KaPPA}

