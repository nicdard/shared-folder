\chapter{Secure Shared Folder}\label{ch:ssf}

In this chapter we describe in details the Secure Shared Folder (SSF) implementation.
We detail the cryptographic implementation and usages of supporting
libraries, as well as their implications in terms of runtime execution
and the challenges we encountered (\cref{sc:ssf-sskg}, \cref{sc:ssf-double-prf}, \cref{sc:DKR-implementation}, \cref{sc:GRaPPA-implementation}). 
For each of them, we provide a description of how the implementation diverges from
the paper's construction pseudocode and the reasons behind them.
While implementing these underlying primitives indeed, we also discovered 
and fixed a few major issues, which caused failures in the 
synchronization of the cryptographic state among clients.
We detail our findings in~\cref{sc:GRaPPA-bugs}.
In this context, we also discuss the state synchronization and rollbacks
in the SSF scheme~\cref{sc:state-sync-rollbacks}, where we also discuss
in details the importance of modelling precisely the
interactions between clients and servers. 
In~\cref{sc:MLS-enhancements} we also
propose enhancement to the MLS library and in general we detail
what features MLS implementations should provide to better support
the implementation of new cryptographic primitives on top of
CGKA and/or MLS.
Finally, we provide our enhancement proposals to the primitives, 
such that they can better model the different client entities with respect
to their cryptographic state (\cref{sc:DKR-enhancements}).

\section{SSKG implementation: TreeSSKG}\label{sc:ssf-sskg}

The DKR construction uses SSKG, which is introduced in \cref{sc:SSKG}.
However, there is no browser-compatible implementation available to the best of our knowledge.\footnote{We found a reference Go implementation~\cite{SSKGGo}, which was used as reference together with the pseudocode in~\cite{ESORICS:MarPoe14}}
We therefore re-implement the pseudocode in~\cite{ESORICS:MarPoe14} using TypeScript.
The choice of the tree-based implementation is motivated both by its efficiency
compared to the numerical construction and because it only uses common cryptographic
primitives, such as hash functions, PRGs or block ciphers, which are supported by the Web Crypto API.
More importantly for the practical use case, is the added \texttt{SuperSeek}
functionality, which is a generalisation on top of the \texttt{Seek} procedure.
While \texttt{Seek} can only calculate an output starting from the initial state, 
\texttt{SuperSeek} can start the forward derivation from an arbitrary point of the sequence.
This allows the usage of SSKG inside the GRaPPA construction,
as we can just share the state of the SSKG chains from and until
the epochs we want to give access to. 
We pay more in terms of additional bytes sent over the wire, each SSKG
state in the tree-based construction requires up to $O(log(n))$,
where $n$ is the number of elements in the sequence generated by the SSKG.

The subfolder \texttt{ssf-client/src/protocol/sskg/} of the project contains the code for the SSKG module.
The file \texttt{sskg.ts} specify the functionalities exported by an instance of the object.
The file \texttt{treeSSKg.ts} contains the implementation of the tree-based SSKG
as a TypeScript class \texttt{TreeSSKG}. The unit tests are in \texttt{test/treeSSKG.test.ts} file,
completely covering the code, with some additional randomized tests for further validation of the code.
The \texttt{TreeSSKG} class maintains the state internally in fields:
\begin{itemize}
    \item A read-only \texttt{name}, used to identify the SSKG instance and helpful in debugging.
    \item A read-only \texttt{totalNumberOfEpochs}, a positive integer representing the total number of elements derivable from this TreeSSKG instance.
    \item A mutable \texttt{stack} field, a stack data structure, i.e. a JS array, storing the internal state of the SSKG as detailed in~\cite{ESORICS:MarPoe14}. The stack is used as a more convenient and performant way of representing a pre-order traversal on the tree structure on which the SSKG is based on. Each element of the stack is a tuple of the form $[s, h]$, where $s$ is an element in the pseudo-random sequence, stored as a byte array of type \texttt{ArrayBuffer}, while $h$ is the height of the node storing $s$ in the tree.  
\end{itemize}

The implementation required the following cryptographic operations:
\begin{itemize}
    \item Generate a starting element (seed) for the SSKG.
    \item A pseudo-random function (PRF) to derive the next element in the sequence. 
\end{itemize}

This seed is generated using the Web Crypto API \texttt{subtle.generateKey} function 
to create an HMAC SHA-256 symmetric key which is then exported to raw bytes using \texttt{exportKey}.

The Web Crypto API provides the necessary cryptographic primitives
through the \texttt{subtle} object, with the functions
\texttt{generateKey} and \texttt{deriveKey},
which support HMAC and HKDF algorithms.
We recall that the Web Crypto API representation for a
key, is made through the \texttt{CryptoKey} JS object,
which is a wrapper around the actual key material.
This object contains information about the key, such as
whether the key is a symmetric or asymmetric key, in the latter
case if it is private, the algorithm for which the key is used
and whether it can be exported or not to e.g. raw bytes.
To implement the PRF, we would need to call the \texttt{deriveKey}
function passing a HKDF \texttt{CryptoKey} to derive a new key.
The new derived key would later be used to derive the next
element in the SSKG generated sequence. However,
a call to \texttt{deriveKey} cannot produce a \texttt{CryptoKey}
designated for HKDF, thus we cannot reuse the output for a new
key derivation directly. We resort to use HMAC keys,
as we notice that the HKDF internally uses HMAC.
To assure that the bits of the keys are not truncated,
we fix the hash function to SHA-256 in all cryptographic
operations that require a hash function.
So when invoking any API call where an HKDF or HMAC algorithm are involved,
we always specify that the hash function is SHA-256.
To use the HMAC key in the key derivation, we export it
though \texttt{exportKey} to raw bytes, and then import it
back to a new \texttt{CryptoKey} designated for HKDF.
In this way the generation of a SSKG seed is done through a call
to \texttt{generateKey} to create an HMAC key, which is exported
and stored as raw bytes inside the stack data structure in the
\texttt{TreeSSKG} instance field. The PRF is then operating on the
raw bytes as explained above. The method \texttt{GetKey}
is used to retrieve a \texttt{CryptoKey} designed for HKDF.
This is done by first applying the PRF, with a \texttt{key}
label, and then importing the bytes obtained from the PRF
through \texttt{importKey}. 

In our implementation we also expose two additional methods:
\texttt{getRawKey} and \texttt{clone} which are not present 
in the SSKG primitive. 
\texttt{getRawKey} method is equal to \texttt{GetKey}
but does not perform the final import of the key, thus it 
returns the current element of the sequence as raw bytes
in an \texttt{ArrayBuffer}. 
This utility will be used in~\cref{sc:ssf-double-prf}.
The \texttt{clone} method, as the name intuitively suggests, 
is used to create a new instance
of the \texttt{TreeSSKG} class with a deep copy of the state.
\footnote{In object-oriented programming, when copying an object we normally distinguish between shallow and deep copy. 
A shallow copy is a copy of the object where the fields are 
copied by reference, while a deep copy is a copy where the 
fields are copied by value. 
In our case, we will need a deep copy of the state, 
as the state is mutable, and we want to avoid side effects 
when manipulating the copy.} 
Refer to \cref{sc:DKR-implementation} for its usages inside DKR. 
Also, it has extensively been used in our unit tests, to check the equivalence
of state updates performed through the different SSKG operations.

\paragraph{Serialization} The code provides also utilities for serializing and deserializing 
the object to persist the state in the client's browser storage
as well as sending it to the other members in the group.
Since the internal state contains raw bytes, we use the concise binary 
object representation (CBOR)~\cite{rfc8949} standard to encode it. 
CBOR supports natively the encoding of byte arrays, which is not the case
for JSON.

\subsection{Asynchronous classes initialization in TypeScript}\label{pg:async-classes-init}
We highlight that all the code using the Web Crypto API is 
asynchronous by design to avoid blocking
the main thread of the browser. Blocking the JS main thread
would result in a really poor user experience, as the browser
would be unresponsive while waiting for the cryptographic
operations to complete. This introduces some complexity
when writing stateful classes like \texttt{TreeSSKG},
or later the implementations of DKR (\cref{sc:DKR-implementation})
and GRaPPA (\cref{sc:GRaPPA-implementation}), as the state
of the object is not immediately available after the constructor
is invoked. In TypeScript, one way of elegantly handling
asynchronous constructors, which are not supported, is to make the constructor private
and instance fields that require asynchronous initialization
private. Then expose a static ``factory'' method to create an 
instance of the class.
This method will return a \texttt{Promise} containing the object.
The \texttt{Promise} will be resolved after:
\begin{enumerate}
    \item The constructor has 
    completed its execution, where all synchronous available state
    is initialized, and an instance of the object has been allocated,
    but the fields requiring asynchronous initialization are still
    empty.
    \item The remaining empty private fields are also filled with the missing asynchronous state.
\end{enumerate}
Keeping the constructor private avoids any client code to
instantiate the object directly, thus protecting from
partial object state creation.
We use this pattern in all TypeScript classes that require asynchronous initialization.

\section{Sign using HMAC as a double-PRF}\label{sc:ssf-double-prf}
In \cref{sc:DPRF} we recall the notion of double-PRF security.
As we have seen, under the assumption of fixed key length HMAC 
can be used to instantiate a double-PRF secure construction.
In GRaPPA the generation of shared keys is delegated to the
DKR construction (\cref{sc:DKR-implementation}), which 
internally combines elements of fixed
length from two chains to derive a key (\cref{sc:DKR}).
In our implementation elements generated from any \texttt{TreeSSKG}
always have fixed length of 256 bits, so we can use HMAC to 
combine backward and forward chain elements together to
produce a DKR key, which will be returned by GRaPPA through
the \texttt{GetKey} operation.

The code can be found in {\texttt{ssf-client/src/protocol/doubleprf/}}.
The implementation simply accepts two byte arrays $k_1$ and $k_2$ 
as \texttt{ArrayBuffer} in input and execute the {\texttt{subtle.sign}} 
function of the Web Crypto API. 
Unfortunately again, the Web Crypto API does not provide a direct way to
use HMAC as a function that works on raw bytes. We instead
need to first import $k_1$ in a \texttt{CryptoKey} object
designated for HMAC (using SHA-256 algorithm) and sign, 
and then perform the signature
algorithm treating $k_2$ as the message to sign.
The resulting \texttt{ArrayBuffer} containing the
256 bits signature is then imported back as
a \texttt{CryptoKey} object, designated for HKDF.
We checked the underlying implementations of the Web Crypto API
both for Chrome and Node.js and made sure the bits
from both $k_1$ and $k_2$ are fully used. We remind the
reader that we have fixed our
hashing algorithm to SHA-256 in all cryptographic
objects that require a hash function. Thus, we do not provide
crypto agility, but greatly simplify and speed up the 
implementation, and have the re-assurance that the bits
of our secrets are never truncated during key derivations
because of errors with mixing different hash functions
operating on different bit vector lengths. 
We have defined global constants to specify the parameters
for the cryptographic operations in a common place,
the \texttt{ssf-client/src/protocol/commonCrypto.ts} file.

\section{DKR implementation: KaPPA}\label{sc:DKR-implementation}

In \texttt{ssf-client/src/protocol/key-progression/} we provide
the implementation and tests for the DKR primitive instantiation
D[F, S]~\cite{GKP}. In \texttt{dkr.ts} we provide the 
type definitions for a double key regression, resembling
the DKR syntax. As specified in the D[F, S] construction, we use
SSKG (\cref{sc:ssf-sskg}) to implement efficiently hash-chains supporting seek capabilities from any element of the sequence. 
We only use the methods exported by the
\texttt{SSKG} interface to implement the operations from the primitive. 
In \texttt{dkr.ts} we also provide the types for the various
concepts used in the DKR primitive (\cref{sc:background-generalised-DKR}):
\begin{itemize}
    \item An epoch is just a plain \texttt{number}, aliased as \texttt{Epoch} for readability. We recall that JS numbers are always 64-bits double precision floating point (\cref{sc:webcrypto-api}), however epochs are always positive integers (see \cref{sc:gap-type-safety-of-opaque-byte-arrays}).
    \footnote{For the curious practitioner: the TypeScript type-system allows the usage of union types, e.g. ``\texttt{type positiveUntil3} $ = 1 | 2 | 3; $''. However, the type checker in the TS compiler has an upper limit in the number of elements in a union type. Through meta programming we could generate a type to express a range of positive integers up to a given number. Again however, the TS compiler limits the recursion depth to 999 (in the version used in this project). Another option would be to write a conditional type that checks if the string representation of the number contains a dot and/or a minus sign, and returns a type \texttt{never} in these cases, the number literal otherwise. The problem with each of the options presented is that in practical terms either we hit a compiler limit or create a type which becomes unusable in practice.}
    \item An epoch interval $[l, r]$ is represented as a JS object with properties \texttt{left} and \texttt{right}, where both properties are of type \texttt{Epoch}.
    \item The DKR blocks are expressed as an enum type \texttt{BlockType}.
    \item Each forward chain is represented as a tuple of two elements $[e, ssgk]$ called \texttt{ForwardChain}, where $sskg$ is an \texttt{SSKG} object (\cref{sc:ssf-sskg}) and $e$ is the DKR \texttt{Epoch} at which we released the first element of $sskg$.
    \item Each backward chain is a tuple of three elements $[e, sskg, N]$ called \texttt{BackwardChain}, where $sskg$ is an \texttt{SSKG} object (\cref{sc:ssf-sskg}) and $e$ is the DKR \texttt{Epoch} at which we release the first element (in backward, i.e., reverse, order) of $sskg$, similarly to the forward chains, while the number $N$ specify an upper limit on the number of elements that can be generated by $sskg$. Recall that $sskg$ maintains internally a read-only property \texttt{totalNumberOfEpochs}. It follows that $N \leq$ \texttt{totalNumberOfEpochs}.
    \item We also provide a type to model an ``interval state'', which we call \texttt{DoubleChainsInterval}. An interval state is a slice extraced from a full DKR state, which gives access to forward and backward elements in a given \texttt{EpochInterval}. The property \texttt{forwardChainsInterval} (resp. \texttt{backwardChainsInterval}) contain the slices (as JS arrays) of the forward (resp. backward) chains, where the first (resp. last) chain are shrunk (\cref{sc:DKR}). We notice that both intervals and extensions of the DKR syntax map to this type.
\end{itemize}
The class \texttt{KaPPA} in {\texttt{kappa.ts}} implements the
D[F, S] construction. The state comprises:
\begin{itemize}
    \item \texttt{maxEpoch}, of type \texttt{Epoch}, representing the current epoch of the DKR state. This means for all epochs from 0 to \texttt{maxEpoch} we can get a key.
    \item \texttt{forwardChains}, a JS array of \texttt{ForwardChain}.
    \item \texttt{backwardChains}, a JS array of \texttt{BackwardChain}.
    \item A read-only \texttt{maximumIntervalLengthWithoutBlocks}, a positive integer representing the maximum number of elements that can be generated by an SSKG used internally.
    Indeed, when creating a \texttt{TreeSSKG} instance, we set the \texttt{totalNumberOfEpochs} to this value (\cref{sc:ssf-sskg}).
\end{itemize}

As recalled above, numbers are 64-bits double precision floating point in JS,
therefore the maximum epoch to which we can progress in a DKR instance is $2^{53} - 1$.
If we progress the DKR state every second, thus creating a new key
every second, this would correspond to keys for
a time span of more than 285 million years. 
Therefore, we can safely
assume that the epoch space is large enough for any practical use case.
Further we highlight that all calls to the Web Crypto API are embedded 
inside the \texttt{TreeSSKG} instances that made up the chains in \texttt{KaPPA}. 

We briefly recall that the operations of the DKR primitive are:
Init, Progress, GetInt, CreateExt, ProcExt, GetKey. Further we recall
that the operation GetKey is defined for both a full DKR state and
an interval state.
While the mathematical
description of D[F, S] does only partially distinguish between
a full DKR state and an interval state, we have chosen to model
this distinction more clearly in the implementation.
Precisely, we map an instance of \texttt{KaPPA} to a user with
read/write access to the DKR state, while a read-only
user will never construct a \texttt{KaPPA} instance, but will
only maintain and query interval states of type
\texttt{DoubleChainsInterval}. Even if the distinction does not
provide any additional cryptographic guarantee, it serves the purpose of
making the code better understandable and usable, as well
as helps in the identification of potential bugs in the implementation
or errors and enhancements in the primitives as explored in \cref{sc:correcting-primitives}.

The distinction is reflected on how the operations are attached
to the \texttt{KaPPA} class. All operations that either require
to access the full DKR state or to update it, are instance
methods of the \texttt{KaPPA} class. We call these admin methods,
as they will be used only by admins of the folders in GRaPPA (\cref{sc:GRaPPA-implementation}).
The only exception is the Init operation, which is a static method 
of the class for the motivations explained in \cref{pg:async-classes-init}.
Therefore, \texttt{Progress}, \texttt{GetInt} and \texttt{CreateExt} are all instance methods.
Diverging from the DKR syntax, they do not take the DKR state in input
as parameter, as we can access private instance state directly,
better modelling the fact that DKR is stateful.
Further, \texttt{Progress} does not return the modified state as output, 
but instead it modifies the state of the \texttt{KaPPA} instance in memory.
Modelling the primitive as a class with private state, also
allows us to easily keep multiple different instances
in memory, which is practically useful for multi-tenancy, i.e., 
for a client handling multiple shared folders with GRaPPA.
The ProcExt and GetKey operations are instead static methods,
thus usable without constructing a \texttt{KaPPA} instance,
and are the only methods that can be used by members in GRaPPA,
which do not have write access to the DKR state (\cref{sc:CGKA-implementations}).
GetKey is also defined as an instance method for convenience,
but internally it will first construct an interval state and
then call the static GetKey method on it, as described in the
DKR syntax.

\paragraph{Serialization} As seen in SSKG, our implementation also provides utilities
to serialize and deserialize the state of the DKR object
using CBOR encoding, as we will ultimately need to
store the state in the client storage.
The upper bound on the cost of serializing the state is determined
by the number of chains (\texttt{TreeSSKG} instances),
as we need to recursively serialize the state of SSKG first.
Since we need to support also member state serialization and deserialization,
we provide static methods to serialize and deserialize interval
states of type \texttt{DoubleChainsInterval}.

\paragraph{KaPPA time efficiency}

Regarding the internals and efficiency of the D[F, S] construct
implementation \texttt{KaPPA}, we translated the helper functions 
GetFChains and GetBChains to get
slices of forward and backward chains into a modified binary search,
to avoid the linear search in the original pseudocode.
We notice that the chains are stored ordered by epoch of instantiation,
and recalling the types \texttt{ForwardChain} and \texttt{BackwardChain},
we can see that the first element of the tuples is the epoch.
We therefore have written a generic binary search function that,
given an epoch, returns the index in the array of the forward (resp. backward)
chains of type \texttt{ForwardChain} (resp. \texttt{BackwardChain})
where the first tuple element is equal to the input epoch. If the epoch is not
present, the function returns the index of the element with the 
closest epoch that is smaller than the input epoch.
As all \texttt{KaPPA} instance methods, i.e., admin operations,
are relying on these helper functions, each of them has a time
complexity of $O(log(n))$, where $n$ is the number of forward (resp. backward)
chains in the \texttt{KaPPA} instance.

\section{Using the MLS Library from JS Runtime}\label{sc:js-bindings-for-mls}

To build the GRaPPA protocol, we need a browser compatible implementation
of MLS (\cref{scc:clients}).
In this section we detail how we use and export the functionalities from mls-rs
into JS. We also give an overview of the internal state management
that is relevant for later discussions on state synchronization and resiliency
against errors (\cref{sc:state-sync-rollbacks}).
We recall from \cref{ch:setup} that mls-rs is a Rust library, which we
compile to WASM to execute in the browser.
Further we recall that WASM runtime has a separate memory and runs in parallel
to the JS main thread in a separate execution environment, however, communication
is possible across boundaries. Some restrictions apply to the data types
that can be transferred between the two environments, as WASM natively supports
only numeric types and arrays. Thus, the general strategy 
is to minimize sending objects or any other complex data structures 
while writing our bindings.

\paragraph{Compilation and Dependency Management}
As we are using \texttt{wasm-pack} to compile mls-rs to WASM, we automatically
get a ready to use JS module, including the type definitions for TypeScript,
which can be installed as a NPM dependency. The \texttt{wasm-pack} also 
provides the JS shim to allow sending strings and work with the WASM linear
memory. Further, we can compile asynchronous Rust code to WASM, as support is also
provided by the toolchain. For details, see the documentation in~\cite{WasmBindgen}

The code for exposing the mls-rs functionalities is provided in the Rust library crate
\texttt{ssf/}. We detail how to compile the code in the \texttt{README.md}
as well as how to run the unit tests written with \texttt{wasm-bindgen-test}.
We remind the reader, that the tests successfully execute both in Chrome browser and Node.js.
The crates from AWS mls-rs library are included in \texttt{Cargo.toml}
file, where we specify our own branch of the repository which 
includes the changes we made to the library to fix the compatibility issues
with Node.js (\cref{sc:MLS-enhancements}). From the library, we use the following crates:
\begin{itemize}
    \item \texttt{mls-rs}: the main library for the MLS protocol, exporting the Rust client and state storage in-memory providers.
    \item \texttt{mls-rs-crypto-webcrypto}: the cryptographic provider implementation for WASM target. This is based on the Web Crypto API.
    \item We further explicitly specify the \texttt{mls-rs-core} crate, because we want to use our own branch of the library and enable the compilation options needed for the compatibility.
\end{itemize}

For practicioners, although missing from the library documentation, we highlight that
the correct compilation of the library to WASM is possible only by enabling the
compiler flag \texttt{rustflags = "--cfg mls\_build\_async"}. This option is added
to the \texttt{Cargo.toml} under the \texttt{[build]} stanza.

\paragraph{Error Handling}
The mls-rs library uses an internal error type to handle errors, as generally
done in Rust. The \texttt{MlsError} enum represent all the possible errors generated
by the library, with their description.
In the JS bindings written in \texttt{ssf/lib.rs} every function returns a Rust
\texttt{Result} type, expressing either a successful result or an error.
The error is always converted to a string, the description of the error itself,
before sending it to JS. 
Note that since all the functions we export are asynchronous,
the \texttt{Result} is converted to a JS \texttt{Promise} in the JS shim.
A \texttt{Promise} object in JS represents the eventual completion (or failure) 
of an asynchronous operation, and its resulting value or error.
When calling the bindings from our TypeScript client implementation,
we can therefore easily handle the errors,
as if the code was natively written in JS.

\paragraph{Application Messages}
The main entry point of the library is the \texttt{Client} struct, which
is an object providing functionalities to create a MLS (internally CGKA)
group and evolve its state. Further, the \texttt{Client} allows the creation
and encryption of application messages (\cref{sc:MLS}). While in the pseudocode 
of GRaPPA there is no explicit mention to MLS protocol, but only to CGKA,
we notice that while executing the operations of the GRaPPA protocol,
parts of the control message are encrypted through advanced authenticated
encryption standard (AEAD) using the CGKA shared epoch secret.
Instead of manually implementing this encryption, we just use the
support for application messages provided by MLS. This difference highlights
that although for demonstration purposes the MLS protocol is kept out of scope,
practically the GRaPPA protocol is build on top of MLS and not only CGKA, 
and further motivate our choice to use an MLS library.

\paragraph{Key Packages}
The \texttt{Client} object provides also functionalities to create
key packages, which are the cryptographic material needed to invite
a new member to the group. A key package can be created through the provided
bindings and returned as a \texttt{Uint8Array}, the serialization and
deserialization of the key package (when processing it) is handled by the 
library itself. 


\paragraph{State Management}
A \texttt{Client} object can handle multiple MLS groups.
An abstraction layer on top of the storage is provided by the library,
thus allowing for different storage providers to be implemented and used.
The core library provides an in-memory storage provider.
The \texttt{mls-rs-provider-sqlite} crate implements a storage provider
on top of the relational database engine SQLite. However, this is not compatible
with the WASM target, as SQLite is not available in the browser environment.

We resort on the in-memory storage provider, which is not persisted
across browser sessions. We propose an implementation of a browser compatible
storage provider in~\cref{sc:MLS-enhancements}, but leave it out of scope
for this project, because of the time constraints.
A further problem arising when using the in-memory storage provider
is that since we are making calls from the JS runtime to the WASM runtime,
after a call to the WASM runtime ends, the WASM memory would be freed and
therefore the state would be lost.
We solve this problem by instantiating the storage provider related objects
in a lazy-initialised global map addressed by used identity, which is kept in
the WASM memory across calls. This hack is not ideal and only temporary,
until we implement a browser compatible persistency solution.
Also, to avoid losing the changes in the group state between calls to
exported functions, we need to write the state back to the storage provider
after each operation that modifies it.

On a related note, we recall that the MLS protocol is using a proposal
and commit mechanism to evolve the state of the group (\cref{sc:MLS}).
The \texttt{Client} object can be used to get a handle to the \texttt{Group}
object, which is the main interface to interact with a CGKA group.
The \texttt{Group} object provides functionalities to create proposals,
commit them, and apply the state changes. Further, it also provides
functionalities to encrypt and decrypt application messages.
We notice that creating a commit message, which is the final step
in the proposal and commit mechanism, does not update the \texttt{Group}
current state until a call to apply the changes is made on the object.
In between the creation of the commit message and the
application of the changes, the \texttt{Group} object maintains
the pending commit in a staging area~\cite{AWSMLSGroup}.
This staging area is also persisted in the storage provider.
However, the modifications to the cryptographic state of the group
deriving from the pending commit are not taken into consideration
when encrypting application messages, i.e., the encryption
is done using the current state of the group only, and not the pending commit.
This limitation has effects on the state synchronization and rollbacks
as described in \cref{sc:state-sync-rollbacks}.


\section{Delivery Service}\label{ssc:delivery-service}

The delivery service (DS) is a new component of our system needed 
to support the usage of MLS (\cref{sc:MLS}).
The DS is responsible for delivering messages to clients in order.

We implement the abstract functionalities of the DS inside our server existing SSF Proxy
server. As the SSF Proxy server is already responsible for the creation of folders,
authentication of users, and the access control of users to folder, 
it is a natural choice to extend its functionalities to also provide APIs
for the DS.
To this end we modify and extend the SSF Proxy server and its SQL database as described
in the following of this section.
In short, we add new endpoints to receive, fetch and acknowledge
GRaPPA control messages, as well as key packages.
We store in the ``ds'' MySQL database the new entities, and use the
DB transaction support to synchronize multiple clients executing
the GRaPPA protocol. Further we use the server as a broadcaster to
deliver the messages to the clients in order. The ordering guarantee
is provided by our usage of MySQL.

\paragraph{Additional Database Entities}
We extend the SQL database ``ds'' with three different tables:
\begin{itemize}
    \item \texttt{key\_packages} storing the key packages (\cref{sc:js-bindings-for-mls}) generated by the MLS clients for later retrieval by a different user which wants to send an invitation to join a folder.
    \item \texttt{pending\_group\_messages} storing the first part of a GRaPPA control messages of type \texttt{Proposal} that are pending and need to be delivered to the clients (\cref{sc:GRaPPA-implementation}).
    \item \texttt{application\_messages} storing the second part of a GRaPPA control messages of type \texttt{ApplicationMessageForPendingProposals}, that are pending and needs to be delivered to the clients (\cref{sc:GRaPPA-implementation}).
\end{itemize}

All the data from the client is sent as a byte array, and stored in each table
directly by the server. The server is not aware of the content of the messages,
and does not need to be.

The \texttt{key\_packages} table has the following columns:
\begin{itemize}
    \item \texttt{key\_package\_id} as a primary key, a unique autoincremented identifier for the key package.
    \item \texttt{key\_package} as a \texttt{BLOB}, storing the key package.
    \item \texttt{user\_email} a foreign key to \texttt{users.user\_email}, storing the email of the user that created the key package. A \texttt{CASCADE DELETE} constraint is set to delete the key package when the referenced user entity is deleted to automatically clean up the database.
\end{itemize}
Key packages do not belong to a folder, but just to a user, and are used to
add a new user to a folder in GRaPPA, since the user needs to be added to the
member CGKA group (\cref{sc:gkp-scheme}).

The \texttt{pending\_group\_messages} table has the following columns:
\begin{itemize}
    \item \texttt{message\_id} as a primary key, a unique \texttt{INTEGER} autoincremented for the message.
    \item \texttt{payload} as a \texttt{BLOB}, storing the message.
    \item \texttt{folder\_id} a foreign key to \texttt{folders.folder\_id}, storing the unique identifier of the folder to which the message belongs. A \texttt{CASCADE DELETE} constraint is set to delete the message when the referenced folder entity is deleted to automatically clean up the database.
    \item \texttt{user\_email} a foreign key to \texttt{users.user\_email}, storing the email of the user to which the message has to be delivered. A \texttt{CASCADE DELETE} constraint is set to delete the message when the referenced user entity is deleted to automatically clean up the database.
\end{itemize}
We point out that since we store messages by \texttt{user\_email} we are replicating
the \texttt{payload} content for each user that is part of the folder 
referenced by \texttt{folder\_id}. This will be changed and was done as a first
development iteration to simplify the implementation. Ideally this table should be
normalized, and a new table containing only the \texttt{message\_id}
and the \texttt{payload} should be created, to store the content of the messages
indipendently from the receiving users.

This table is effectively modelling an ordered queue of messages:
the messages are globally ordered by insertion time through the \texttt{message\_id},
so for each user and folder, the subset of messages belonging to that
user and folder is also ordered by insertion time.
We call a queue the ordered subset of messages belonging to a user and folder in the table.
We will see in the API description how the server guarantees consistent state updates
of the clients.

The \texttt{application\_messages} table has the following columns:
\begin{itemize}
    \item \texttt{id} as a primary key, a unique \texttt{INTEGER} autoincremented for the application message.
    \item \texttt{message\_id} a foreign key to \texttt{pending\_group\_messages.message\_id}, storing the unique identifier of the message which is completed by this application message. A \texttt{CASCADE DELETE} constraint is set to delete the application message when the referenced pending group message entity is deleted to automatically clean up the database.
    \item \texttt{payload} as a \texttt{BLOB}, storing the message.
\end{itemize}

This table stores the second part of the GRaPPA control messages, which reference 
the first part stored in the \texttt{pending\_group\_messages} table.
Similarly, we simplify and speed up development by replicating the
payload for each foreign key to a pending group message.

\paragraph{DS APIs}
We extend the SSF Proxy server with the following new endpoints to support the DS functionalities:
\begin{itemize}
    \item \texttt{POST /users/keys}: upload a key package.
    \item \texttt{GET /users/<folder\_id>/keys}: fetch and delete a key package given an existing folder id. The folder id parameter is needed for access control, to verify that the user requesting the key package of another user is part of the corresponding folder. The server will return an error in case the user for which the key is requested is already part of the given folder.
    \item \texttt{POST /folders/<folder\_id>/proposals}: upload the first part of a GRaPPA control message for the specified folder. The server checks that the sender identity is part of the folder is part of the folder, and that the user's pending message queue for the folder is empty. The upload will transactionally check and update the database, inserting the message in the \texttt{pending\_group\_messages} table, replicated for each user in the folder as known by the server through the \texttt{users\_folders} table, excluding the sender user.
    \item \texttt{PATCH /folders/<folder\_id>/proposals}: upload the second part of a GRaPPA control message for the specified folder and message ids sent in the request body. The server performs the usual access control checks, and transactionally check and insert the message in the \texttt{application\_messages} table, replicated for each message id in the request body. This a PATCH request, as logically it is patching the first part of the GRaPPA control message with the second part. Furthermore, a notification containing the folder id is sent to the clients receiving the message, to signal that they need to try fetching new GRaPPA control messages from the server. 
    \item \texttt{GET /folders/<folder\_id>/proposals}: fetch the first GRaPPA complete control message for the specified folder from the user's queue. The server checks that the user is part of the folder, and returns the first pending message together with the corresponding application message if available.
    \item \texttt{DELETE /folders/<folder\_id>/proposals/<message\_id>}: delete the GRaPPA complete control message for the specified folder and sender user. The server deletes both related entries, transactionally from the tables \texttt{pending\_group\_messages} and \texttt{application\_messages}.
    \item \texttt{PATCH /v2/folders/<folder\_id>}: given the folder id, share the corresponding folder with the user specified in the request body. The body further includes the bytes of the first part of the GRaPPA control message which invites the new member to the folder. Since this operation requires a different DB transaction to be performed compared to the original version of the endpoint, we defined a v2 version of the endpoint to also avoid regressions in the baseline implementation.
\end{itemize}

We notice that, although in the GKP scheme and in the GRaPPA construction nor capture the
interaction with the server, neither the server
is assumed to hold any state, in our implementation we
need to take both things into account. 
The DS state, which is made up by the pending messages, reflects the fact that
the cryptographic state of the clients for which the queues are non-empty,
is out of sync with respect of other group members.
We will discuss the implications in \cref{ssc:GKP-client-middleware} and \cref{sc:state-sync-rollbacks}.

\section{GKP implementation}\label{sc:GRaPPA-implementation}

The group key progression (GKP) primitive (\cref{sc:SSF}) is instantiated in the
GRaPPA construction, of which the pseudocode is provided in~\cite{GKP}.
We implement GRaPPA in TypeScript, similarly to the DKR primitive.
The code can be found in \texttt{ssf-client/src/protocol/group-key-progression/},
including the unit tests in \texttt{test/grappa.test.ts}.

\paragraph{Types Design}
As in the other primitive implementations,
we provide the TS types for the GKP entities used 
in the GRaPPA construction (\cref{sc:gkp-scheme}) inside the \texttt{gkp.ts} file:
\begin{itemize}
    \item We model the state of both members and admins as TS interfaces. 
    TS interfaces are extendable, so we factor out the common
    properties to keep our code DRY in the type definitions.\footnote{DRY stands for Don't Repeat Yourself, a software development principle that aims to reduce repetition of code patterns.}
    Having shared properties factored out also make them always
    available to the type checker while writing common code paths for both members and admins.
    In the \texttt{BaseState}, the \texttt{cgkaMemberGroupId} property stores the unique identifier of the group to which all members belong to. We will use the unique folder id returned by the SSF Proxy server at folder creation (\cref{sc:ssf-proxy-server}). This is represented as an opaque \texttt{Uint8Array}. 
    The interface \texttt{MemberState} extending the \texttt{BaseState} adds:
    \begin{itemize}
        \item A \texttt{role} property, a literal string \texttt{member}.
        \item An \texttt{interval} property, of type \texttt{DoubleChainsInterval}, storing the interval state to which the member has been given access to (\cref{sc:DKR-implementation}).
    \end{itemize}
    The admin state modelled in \texttt{AdminState}, also extending \texttt{BaseState}, adds:
    \begin{itemize}
        \item A \texttt{role} property, a literal string \texttt{admin}.
        \item A \texttt{cgkaAdminGroupId} property, of type \texttt{Uint8Array}, storing the unique identifier of the group to which all admins belong to. We will deterministically construct this identifier concatenating a prefix \texttt{ADMIN-} with the \texttt{cgkaMemberGroupId} 
        \item A \texttt{dkr} property, holding an instance of \texttt{KaPPA} (\cref{sc:DKR-implementation}).
    \end{itemize}
    Finally, a type \texttt{ClientState} represents either a member or an admin state, and will be used
    inside the \texttt{GRaPPA} class.
    \footnote{As the \texttt{MemberState} and \texttt{AdminState} interfaces 
    extend the \texttt{BaseState} interface and they both declare 
    a \texttt{role} property, after a check for the value of role in the code,
    the compiler will be able to infer which properties are available in the object, 
    providing type safety and avoiding runtime errors. This type system feature is called discriminated union.~\cite{TSDisciminatedUnions}.}
    
    \item We define the types corresponding to each command from GKP (\cref{sc:gkp-scheme}),
    and we make clear distinction between commands that target another user and commands that
    target the user executing them.\footnote{To avoid bloating the description, we forward to the code for the details. We again make use of discriminated union modelling each command with its own type.} 
    We notice that the arguments to \texttt{ExecCtrl} all take
    either zero or one argument, which is the target user identifier.
    The type \texttt{ControlCommand} is a union of all the possible types of commands.

    \item The \texttt{Proposal} type partially represent the messages exchanged in GRaPPA 
    to carry out the various commands. The name carries out the semantic
    that a proposal could be rejected by the system (\cref{sc:state-sync-rollbacks}).
    Each proposal embeds a property \texttt{cmd}
    specifying the exact type of the command and its arguments if any.
    Then, depending on the \texttt{cmd}, the proposal will additionally contain:
    \begin{itemize}
        \item \texttt{memberControlMsg}, a \texttt{Uint8Array} containing the encrypted member CGKA control message (all). Corresponds to the variable $T_M$ in the pseudocode.
        \item \texttt{memberWelcomeMsg}, a \texttt{Uint8Array} containing the CGKA welcome message for a new joining member (\texttt{Add}). Corresponds to variable $W_M$ in the pseudocode.
        \item \texttt{adminControlMsg}, a \texttt{Uint8Array} containing the encrypted admin CGKA control message (\texttt{AddAdm}, \texttt{RemAdm}, \texttt{UpdAdm}, \texttt{Rem}, \texttt{RotKeys}). Corresponds to the variable $T_A$ in the pseudocode.
        \item \texttt{adminWelcomeMsg}, a \texttt{Uint8Array} containing the CGKA welcome message for a member which is granted admin privileges (\texttt{AddAdm}). Corresponds to variable $W_A$ in the pseudocode.
    \end{itemize}

    \item The \texttt{AcceptedProposal} type is used to model a \texttt{Proposal}
    that has been accepted by the delivery service (\cref{ssc:delivery-service}),
    and is therefore considered applied in all clients state, meaning that
    the changes specified in the proposal fields, which are CGKA proposals (\cref{sc:CGKA}),
    have been committed inside the respective CGKA state (\cref{sc:js-bindings-for-mls}, \cref{sc:state-sync-rollbacks}).
    An \texttt{AcceptedProposal} is a \texttt{Proposal} with the additional \texttt{messageId} field of type \texttt{number},
    which is a unique global identifier of the message in the system, and is added by the DS if the proposal is accepted.

    \item \texttt{ApplicationMessageForPendingProposals} is the type modelling a partial message exchanged in GRaPPA
    to carry out the various commands. The name carries out the semantic of being composed of 
    MLS application messages (\cref{sc:MLS}). This message is sent to the delivery service
    with references to the \texttt{AcceptedProposal} that are completed by this application
    message, meaning that they constitute together the full control message (or welcome message) as described in
    GRaPPA~\cite{GKP}. The references are a list of \texttt{messageId}, i.e. an array of \texttt{number}.

\end{itemize}

The reader should notice the difference between the modelling of message exchange
in the GRaPPA construction and in the implementation. This
difference can be seen already in the types, where the \texttt{Proposal} type is only 
partially modelling the return values from the operations of 
GRaPPA pseudocode, representing only the CGKA control messages. 
As described in \cref{sc:js-bindings-for-mls},
the library mls-rs does not allow a stashed change in the internal 
cryptographic group secret of CGKA that has not been fully applied,
i.e. the state is advanced and there is no possibility to rollback, 
to encrypt an application message until the proposals are committed and applied. 
Thus, we need to split proposals and commits from what we call GRaPPA application messages (\cref{sc:state-sync-rollbacks}, \cref{ssc:delivery-service}).

\paragraph{Object-oriented Design}
Inside \texttt{gkp.ts}, we also provide the interface exposing the functionalities 
of the \texttt{GKP} primitive (\cref{sc:gkp-scheme}). 
The procedures \texttt{JoinCtrl} and \texttt{InitUser} are however not instance
methods of the class, but static methods. This is because one is used to create
a GRaPPA client to actively create a new group, while the other is used to
join an existing group to which we are invited. 

The TS class \texttt{GRaPPA} implements GRaPPA and maintains the state internally, similarly to the \texttt{KaPPA} class.
The class holds the following state:
\begin{itemize}
    \item \texttt{uid}: the \texttt{Uint8Array} representing the unique identifier of the user, this corresponds to the \texttt{user\_email} in the public certificate from the PKI and with which the user register itself to the SSF Proxy.
    \item \texttt{middleware}: an instance of the \texttt{GKPClientMiddleware} class, which is used to abstract away calls to the backend. It is injected in the constructor of the class, it is useful mostly for testing purposes (\cref{ssc:GKP-client-middleware}).
    \item \texttt{state}: an object of \texttt{ClientState} type, which is either a \texttt{MemberState} or an \texttt{AdminState} depending on the role of the user in the group. As this field depends on the user being part of a group, it is \texttt{undefined} until either \texttt{JoinCtrl} or \texttt{Create} are successfully executed. 
\end{itemize}

Similarly to what is happening in \texttt{KaPPA}, \texttt{GRaPPA} methods changing the state
do not return the new state, but instead apply the changes to the internal one.
Each method when executed successfully, will also try to serialize the state to persistent
storage before completing.


\subsection{GKP Client Middleware}\label{ssc:GKP-client-middleware}
The biggest difference between GRaPPA and the implementation in \texttt{GRaPPA}
is that we need to interact with a real server.

The interaction is abstracted away in \texttt{GKPClientMiddleware} type, also
defined in \texttt{gkp.ts}. The middleware is used to handle serialization of
\texttt{Proposal} and \texttt{ApplicationMessageForPendingProposals} objects
to byte arrays using CBOR and to send them to the server using the autogenerated
APIs from the OpenAPI specification of the SSF Proxy server (\cref{sc:ssf-proxy-server}).
This enhances readability of the \texttt{GRaPPA} code, as well as maintainability,
as the code for the communication with the server is kept in a single place.
We implement the middleware in the \texttt{dsMiddleware.ts} file.
A mock implementation is provided in the \texttt{inMemoryMiddleware.ts} file, 
and it is used to test the \texttt{GRaPPA} class in isolation from the server.
We highlight that the serialization code for the message is kept separate
from the implementation of GRaPPA, as the \texttt{GRaPPA} class should ideally only
deal with the cryptographic operations.

\subsection{State Synchronization and Rollbacks in real-world GRaPPA}\label{sc:state-sync-rollbacks}

Global state synchronization and state rollbacks are a crucial part of the \texttt{GRaPPA} implementation,
to maintain the consistency of the group state across all clients, which is 
key for a client own ability to continue to operate and access the folder in case of
errors or rejection of a proposal by the DS.

Ideally, the middleware calls should not be interleaved with the cryptographic operations.
We would like to have a unique GRaPPA control message, including both parts we model in our types, 
to be sent to the server
at once and then delivered to all recipients.
However, as we have seen in \cref{sc:js-bindings-for-mls},
since we are calling the mls-rs library across process boundaries, i.e. from JS
runtime into WASM runtime, we cannot avoid writing the changes to the state
in the persistent storage.
Further, the DS, hence the SSF Proxy server, needs to ensure that messages are delivered in
a global order to all clients, and is offloaded the responsibility to 
keep the client synchronized (\cref{ssc:delivery-service}).

Let's consider the case where a client starts executing a \texttt{ExecCtrl} operation:
the client needs first to check if the operation is accepted by the DS,
and then apply the changes to the state. The only trusted source of truth
on the synchronization status of the system among clients of a folder is
the DS, as the DB is acting as a global transactionally consistent state.
Now if the client proposing a change to the group state tries to send it
to the DS, and the DS rejects the proposal, the client should roll back the changes as well.
Since there is no way to roll back the changes with the in-memory storage provider
of the mls-rs library, we therefore need to split the GRaPPA control messages
in two parts, as already explained when describing the types in \cref{sc:GRaPPA-implementation}
and the DS API in \cref{ssc:delivery-service}.
Now that we have motivated the choice, the GRaPPA \texttt{ExecCtrl} protocol is modified as follows:
\begin{enumerate}
    \item Depending on the operation, the client propose and commit the respective changes internally to the CGKA group state, some of those operation involve both member and admin CGKA groups. The pending commit(s) is written to the in-memory storage (inside WASM).
    \item The client sends the first part of the GRaPPA control message (of type \texttt{Proposal}) to the server, through the \texttt{POST} proposal endpoint. In case the operation originating the proposal is the addition of a new member to the folder, use the v2 \texttt{PATCH} folders endpoint (\cref{ssc:delivery-service}).
    \item The DS process the \texttt{POST} proposal request (or v2 \texttt{PATCH} folders request) and depending whether the pending message queue for the sender in the targeted folder is empty or not:
    \begin{itemize}
        \item accept the message and send back all the \texttt{messageId} created by the DB to the client;
        \item reject the message and send back an error to the client.
    \end{itemize}
    \item In case an error is received, the client discards the pending commit(s), thus rolling back the changes to the CGKA group state(s). The client will then try to sync the outdated local state, by fetching the latest GRaPPA control messages from the DS until \texttt{Not Found} error is returned, and returns the error to the user.
    \item If the \texttt{POST} proposal call is successful, the client applies the changes to the state. 
    \item Then it computes the application message and encrypts it through a call to the JS bindings. At this point the encryption will be performed with the updated CGKA epoch secret. 
    \item Send the second part of the GRaPPA control message to the DS. Retry with an exponential backoff in case of errors.
    \item Finalise the operation, by also serializing and persist the eventual changes to the internal \texttt{dkr} object.
\end{enumerate}

\section{Web Crypto API implementations: non-standard behaviours}\label{sc:Web-Crypto-API-implementations:-non-standard-behaviours}

In this section we describe a gap in the Web Crypto API specification and
implementation which creates a portability issue affecting the mls-rs library
and, in general, the deployability of advanced cryptographic primitives
in the browser environment.

AWS mls-rs library implementors made the library available to WASM compilation
targets. The support is experimental, and involves for now only the implementation
of the cryptographic operations needed to execute MLS on top of the Web Crypto API.
\footnote{For practicioners: the library abstracts the underlying cryptographic code in Rust traits which are implemented for different runtime environments/cryptographic providers. The one used for the WASM target is implemented in the crate \texttt{mls-rs-crypto-webcrypto}.}
While looking at the history commits of the library, we found out that
apparently the support for WASM builds was added only for Chrome runtime.
Indeed, integration tests are available
which prove the compatibility and avoid regressions in Chrome.
However, as seen in \cref{ch:setup} we use Node.js to develop our client
CLI to avoid dealing with a UI in our first implementation.
Thanks to our multi-runtime setup, we discovered the portability issue
hidden in the library and in the implementations of the Web Crypto API.

When using elliptic-curve crypto primitives, the mls-rs library is 
calculating the public key from the bytes of a DER-encoded~\cite{Kaliski2002ALG} private key 
by passing the bytes of the key to the Web Crypto API \texttt{importKey}
call~\cite{WebCryptoAPIImportKey}. In Chrome, the underlying BoringSSL~\cite{BoringSSL} on which the Web Crypto API 
are implemented on, calculates
the public part of the key which is appended to the bytes provided in input.
However, this behaviour is non-standard and is not supported in other
major Web Crypto API implementations:
\begin{itemize}
    \item In Node.js, the \texttt{importKey} call will just import the key into a \texttt{CryptoKey} JS object, without any error~\cite{WebCryptoAPICryptoKey}. However, the underlying bytes are not modified. The mls-rs library has a check to detect the missing public key part and will then throw a runtime error. No documentation is provided on this issue, and we discover it by debugging the library compiled in WASM, which is not a trivial task. In the end we resorted on reading the source code. Although the documentation states that the cryptographic support is only experimental, it would be beneficial to add more details on this problem.
    \item In Safari, the \texttt{importKey} call will throw an error.
    \item In Firefox, the \texttt{importKey} call will throw an error.
\end{itemize}

The only case we could develop a work-around for is Node.js. Since in this runtime
the public key is not calculated but at the same time the operation does not fail,
after importing the private key we can force the calculation of the public key with an additional
export/import operation to \texttt{jwk} back to \texttt{pkcs8}~\cite{JsonWebKey, rfc5958}.
Indeed, by performing this \texttt{exportKey} call to produce the \texttt{jwk}
representation of the key, the curves coordinates $x$ and $y$ are calculated.
Therefore, importing back the key in \texttt{pkcs8} format will produce the correct
public key. The code is published in a pull request to the mls-rs library~\cite{AWSNodeJSCodeContributions}.

We notice that the mls-rs uses the Web Crypto API in a way conforming to the
specification, as the definition of the DER-encoded private key does not
restrict to the usage of elliptic curves private key info which must 
contain the public key field, but only should~\cite{rfc5915}:
\begin{quote}
    publicKey contains the elliptic curve public key associated with
    the private key in question.  The format of the public key is
    specified in Section 2.2 of [RFC5480].  Though the ASN.1 indicates
    publicKey is OPTIONAL, implementations that conform to this
    document SHOULD always include the publicKey field.
\end{quote}
We further notice that mls-rs implementation use case seems legitimate,
as the public key calculation would require non-constant time code
to execute on top of a secret key. Therefore, we would propose
to change the Web Crypto API \texttt{importKey} specification to
explicitly require the public key calculation, so that all
implementations are aligned with the same behaviour. In this way,
the API would support more advanced cryptographic operations on top
of the existing support for elliptic curves cryptography, such as the ones
required while trying to compute a key pair from a symmetric private key.

\section{mls-rs: WASM build enhancements}\label{sc:MLS-enhancements}

While working with the MLS library we found out that the WASM target
is not fully production ready:
\begin{itemize}
    \item The library is not fully portable to all major browser (\cref{sc:Web-Crypto-API-implementations:-non-standard-behaviours}).
    \item Node.js is not supported. We add support for the compiled WASM to run in Node.js by adding bindings to the Web Crypto API implementation from \texttt{node:crypto} module of Node.js~\cite{NodeJsWebCryptoAPI}. We adapt the existing JS inline code included in the library to be compatible with the JS module resolution mechanism and syntax of Node.js~\cite{AWSNodeJSCodeContributions}.\footnote{For practitioners: JS modules evolved during the years, Node.js is using CommonJS modules, which is the standard outside browser runtimes. These modules are also used in the npm ecosystem (the package manager for JS). The syntax of CommonJS is using plain JS objects and functions (require) to perform import and exports. Browsers use currently ES modules, which were added in 2015 to the JS standard and widely adopted in 2020. ES modules use the \texttt{import} and \texttt{export} keywords to perform the same operations.}
    \item The client state is not persisted and only kept in memory for now. We propose a strategy to persist the client-side state in the browser's using the Indexed Database API~\cite{MlsRsWebStorageProvider, IndexedDBAPI}.
    \item The X.509 certificates are not supported to manage identities of the users. This imposes the usage of Basic Credentials, which lower the security of the implementation as those credentials are not validated while performing group operations. We propose to add support for X.509 certificates in the library~\cite{MlsRsX509Certificates}. Our proposal aims to support the use case where the CA certificate is embedded in the client code or where the CA certificate is loaded at runtime through a different mechanism. This use case could be applied for example in a corporate environment.
\end{itemize}

\nd{propose enhancement to the library to allow encryption under stashed commits.}

\section{Correcting the primitives}\label{sc:correcting-primitives}

During the implementation and testing the GRaPPA constructions,
we found issues in the pseudocode descriptions as provided
in the original manuscript of \cite{GKP}.
In \cref{sc:GRaPPA-bugs} we provide the full explanations of the bugs, 
and the corrections we applied. 
We also discuss the implications of our corrections. 

The bugs we highlight in the following sections
derive from the underlying primitive not being expressive and
precise enough to model the interactions between distributed clients
playing either the role of admin or member.
In the generalized DKR primitive, the read/write (admin) 
and read-only (member) capabilities (see \cref{sc:DKR-implementation})
are not modelled at all in the mathematical definitions of the operations.
We explore the implications and attempt to correct
the primitive in~\cref{sc:DKR-enhancements}.

\subsection{GRaPPA bugs}\label{sc:GRaPPA-bugs}

While implementing the GRaPPA construction, we found out two bugs.
The bugs are both leading to the same issue: the state of the
sender admin and receiving admins can go out of sync after a certain
sequence of operations are executed.

\paragraph{Bug 1:} The admin operations \texttt{Rem}, \texttt{RemAdm} and
\texttt{RotKeys} require to advance the DKR state from the current epoch
$e_{c}$
to the next epoch $e_{r}$ and add blocks, meaning that new chains are created, 
either a forward, backward or both.
When an admin receives messages containing such operations and related
state updates, it will execute the \texttt{ProcCtrlAdmin} procedure.
In this procedure, the receiver admin will reconcile the local DKR state
with the state update received from the sender admin.
We note that the receiving admin at the time of receiving the message
has its local state synced up to epoch $e_{c}$.
While the \texttt{RemAdm} and \texttt{RotKeys} operations are sending 
the full DKR state, \texttt{Rem} is just sending an extension of the current state. 
This extension was however ignored by the receiving admin, 
which would just get out of sync. We adjusted the \texttt{ProcCtrlAdmin}
procedure to also handle the case where an extension is sent,
and process it. To this end, the receiving admin calculates the
interval of its complete local DKR state, from epoch 0 to the current
local epoch $e_c$. Then it will extend the interval with the extension
received from the sender admin, thus having access to the state up to
the epoch $e_r$. Finally, the resulting interval is used to override the
local DKR state of the admin. 
We notice that the sender admin might still
have access to a bigger state than the receivers,
as receiving admins are forced to shorten their last backward chain
to epoch $e_r$ to correctly process the extension.
However, the admin group will still be able to progress forward
to a new epoch $e_{r+1}$, and maintain all the local views of the DKR state
synchronized. We distinguish two cases, depending on which admin is
executing the next operation:

\begin{itemize}
    \item If the next operation is performed by the same sender admin,
the state of all other admins will be updated with a new extension or with
a full state update, depending on the operation.

\item When the next operation is performed by a different sender admin,
then a new backward chain will be started. The new initial element will 
be sent in the extension and the admin still holding the larger state
will also shrink its state to the epoch $e_r$. 
Then the extension will be processed, so the new initial element
will be added to start a new backward chain, and the state will be
synchronized up to epoch $e_{r+1}$. All other receiving admins will
also process the extension and will be able to progress to the new epoch,
following the same procedure, with the detail that the last backward chain
has already been shortened to epoch $e_r$ in the previous operation.

\end{itemize}

The implications of this na\"ive approach, where admins use the same
mechanism of members to extend their local state, is that in case admins
alternate in performing operations, at each operation a new backward chain
will be started, resulting in an average space complexity of $O(n)$ 
for the initial elements.
Another possible solution would be to send the DKR state also
in case of a \texttt{Rem} operation. This would instead 
increase the bandwidth usage.

\paragraph{Bug 2:} The second bug is caused by an optimization,
with which admins spare on the bandwidth usage while sending the new DKR state
from a sender admin to the receiver admins.
An admin performing the \texttt{Add}, \texttt{UpdAdm} or \texttt{AddAdm}
operation needs to advance the global
epoch of the group and only release a new forward chain element.
This is performed by calling the \texttt{progress} procedure on the current
DKR state. No block is needed, as we only need to disallow
the new member of the group from generating keys corresponding
to epochs before the one in which he joined. Therefore, the receiving
admins would need only to call progress on their local copy of the DKR
global state and should be able to calculate the new forward chain element.
However, recalling the generalisation of the DKR construction,
we might be exactly at the epoch corresponding to a maximum chain length,
either for the latest forward chain, the latest backward chain or both.
In this case, the sender admin would need to generate a new random initial 
element for the new chain(s). If also the receivers randomly sample new 
initial elements in their local state, all admins would be out of sync.
The fix we have implemented is to also send an extension in those cases,
with the same implications as seen above for the other bug.


\subsection{DKR enhancements}\label{sc:DKR-enhancements}

In the discussion on the implications of the adjustments to GRaPPA in ~\cref{sc:GRaPPA-bugs},
we have seen how the admins can get out of sync.
We highlight that the current na\"ive solution still
has some drawbacks, as the state of all admins is not
fully synchronized, where an admin maintains a larger state
comprising all the latest backward chain elements other admins
lost access to.

The key point we want to stress is that the DKR primitive
does not capture the distinction between a full DKR state,
with read/write access, and a partial DKR state, with read-only access
together with the associated operations.
We claim that this distinction is crucial to understand the discrepancy
existing between the state of two clients with different capabilities,
namely admins and members, and provide a more detailed specification
of the two compared to the one provided in the original manuscript of \cite{GKP} (\cref{sc:background-generalised-DKR}):

\begin{itemize}
    \item A DKR full state, which we will simply call DKR state,
    comprises: 
    the current epoch $e_{max}$, 
    the complete list of backward chains
    and the complete list of forward chains,
    from epoch $0$ to epoch $e_{max}$ and 
    a parameter $N$ indicating the maximum length of a chain.
    Each chain is stored from its initial element. 
    In particular, the latest backward chain
    is stored from the initial element of the element sequence order,
    which corresponds to the latest possible element which will be released.
    The state thus contains all the elements to derive the state up to
    the current $e_{max}$ epoch and possibly beyond. 
    In case the latest forward chain is not fully used, i.e.,
    the latest $N$-element was not yet released, and the backward
    chain is not also fully used, i.e., the element
    corresponding to $e_{max}$ is not the initial element of the chain,
    it is possible to compute new elements of both
    chains, which can be used to derive keys for the next epochs.
    \item An interval state is a subset of the DKR state, which
    allows deriving keys for the epochs represented in the interval.
    It comprises: the epoch interval with the starting ($e_{left}$) and ending ($e_{right}$) 
    epochs, with the constraint $0 \leq e_{left} \leq e_{right} \leq e_{max}$,
    where $e_{max}$ is the current epoch of the DKR state from which the interval
    state is extracted. Also, it includes the slices of the backward
    and forward chains corresponding to the epochs in the interval.
    More precisely, the forward (respectively backward) chain from which the
    element for the epoch $e_{left}$ (respectively $e_{right}$) can be derived
    is shrunk to that element, disallowing the derivation of any key outside the epoch interval.
\end{itemize}

When we consider a shared DKR state among multiple clients,
as in the implementation of GRaPPA, it becomes clear
that an operation to extend a copy of the DKR state is
missing in the scheme.
We therefore propose to add the following operation to the DKR
primitive, of which we provide an informal description:

\begin{itemize}
    \item \texttt{CreateFExt($st$, $l$)}, on input the DKR state $st$ and an epoch $l$,
    returns a full-extension $fext$ or error. The full-extension is an interval state
    starting at epoch $l$ and ending at the current $e_{max}$ epoch of the DKR state
    $st$, similarly to an extension. However, in a full-extension we do not 
    specify the ending epoch, as the latest backward chain included in the interval
    could give access to state beyond the current epoch. Practically speaking,
    the full-extension is an extension where the latest backward chain is not shrunk.
    We further highlight that full-extensions always comprises the latest current epoch,
    as they serve the specific purpose of extending the state of a replica of the DKR state.
    Also, a full-extension contains the current epoch $e_{max}$.

    \item \texttt{ProcessFExt($st$, $fext$)}, on input the DKR state $st$ and a full-extension $fext$,
    returns the updated DKR state $st'$ or error. The procedure processes the full-extension
    $fext$ on the provided $st$ as the existing \texttt{ProcExt} operation processes an extension $ext$
    on a provided interval state $int$, with the additional
    operation of setting the state current epoch $st.e_{max}$ to the current epoch contained
    in the full-extension $fext.e_{max}$.
    
\end{itemize}

\paragraph{Solving the Bugs in GRaPPA}
Equipped with the new full-extension entity, its creation and processing operations,
the enhanced DKR primitive can be now used in the GKP instantiation
GRaPPA to solve the bugs from \cref{sc:GRaPPA-bugs}.
Instead of applying the na\"ive solution, where the admins
extend their local state as members, or they always send to each
other the full state, the admins can now send full-extensions
among them for all operations that require to progress the DKR state,
i.e. all admin operations.
This will allow them to keep their local copy of
the DKR state synchronized, with an optimal constant bandwidth 
consumption, thus serving the purpose of correcting the 
protocol and clarify how to minimize the bandwidth consumption.

\paragraph{Optimise Transactional Storage of DKR}

With the new operations \texttt{CreateFExt} and \texttt{ProcessFExt},
we can optimise the transaction storage of the DKR state in a client.
This use case is not only present inside GRaPPA, but in any practical
usage of DKR alone. However, a related discussion can be found in \cref{sc:state-sync-rollbacks}.

Let's imagine the case in which a client 

\subsection{The need for a collaboration between the cryptographic and software engineering communities}\label{sc:collaboration-crypto-se}

This section well exemplify the feedback loop we think is 
beneficial to both cryptographers and software engineers.
We think that design of new cryptographic primitives and
the implementation of the constructions in real-world
setting should go hand in hand. 
The implementation, initially guided by the pseudocode 
description, has discovered bugs in the constructions which
ultimately were caused by a gap in the mathematical model.
The enhancements we are proposing are actively discussed
with the authors of the original manuscript, and we hope
to see them included in the next version. This is one clear
example, among many others during this work, 
where the discussion of the implementation details entailed 
an in-depth analysis of the implications on the model and 
assumptions taken in the design of the primitive.
