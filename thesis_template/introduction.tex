\chapter{Introduction}

Nowdays storing data in public cloud storage systems is a well adpoted practice,
from private users to business customers, 
including even software companies relying on major cloud providers as a storage solution.
A broad variety of data is indeed stored in major cloud providers, even highly sensitive data,
such as medical records.
In cloud environments, data is stored in various locations and normally it is replicated
around the world. Users don't have any physical access to their sensitive data.
Furthermore, cloud providers are normally multi-tenat, meaning that multiple customers
can use their computing power and storage space at the same time, sharing the physical hardware.
While internet communications and instant messaging are normally End-to-end encrypted (E2EE)




\section{Problem}

Protecting data in transit is a well-studied problem.
Security properties such as forward security (FS) and post-compromise security (PCS)
have been formalised for encrypted communication channels.
The ephemeral nature of data in transit allows naturally the formalisation of security models
with such properties.
Protocols, such as Signal and MLS, are implementing these properties in the messaging area,
establishing those properties for end-to-end encrypted (E2EE) communication channels.  
The security models and properties for data at rest is instead lgging behind.
While E2EE encrypted storage solution exist and are deployed by major cloud providers,
little attention is yet given to the achievable security properties 
and wheteher similar or same level of guarantee can be achieved as for data in transit.
Furthermore, common capability of sharing such persistent data greetly extends the attack surface.
Sharing of data can resemble a group chat, however, the \textbf{persistent}.



\section{Theoretical Solution}


\section{Focus of this Thesis}

The focus of this work is the implementation of the proposed theoritical solution,
closing the gaps between the abstract logical world of the proposed scheme
and the actual real-world system.
The implementation should be conducted following the best engeeniring practices 
and guarantee the security expressed in the theoritical design.
A great attention is given to the real-world aspect of this implementation,
posing challenges that would be otherwise uncovered in just a proof-of-concept implementation.
The final outcome of the engineering work is primarly a minimal viable product (MVP) executing the protocol.

\subsection{Summary of Contributions}

We implement an MVP running the secure file sharing scheme (SSF) and a baseline implementation:

\begin{itemize}
    \item We benchmark the SSF implementation against the baseline.
    \item We show how targeting a real-world setting is beneficial to the cryptographic community and its research, driving questions and possible solutions. A sinergy between cryptographers and practicioners is beneficial and needed to both, helping mitigate the amount of vulnerabilities normally found in software we use everyday.
    \item We survey the major problems arising while translating the scheme into a concrete, deployable artifact. In particular, we list the ``Engineering Gaps'' that are uncovered with this work.
\end{itemize}

\section{Outline}