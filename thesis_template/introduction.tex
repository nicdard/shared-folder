\chapter{Introduction}
In this chapter, we provide an overview of the motivation
behind this work. 
We then give a
description of the mental model of the system we aim to build,
the Secure Shared Folder (SSF) system.

We end the chapter with a summary of our contributions,
and an outline of the organisation of the rest of the document.

\section{Motivation}\label{intro-motivation}

File storage on public cloud providing sharing capabilities is a common service used by many people and organizations.
Some notable examples of such services in the wild are Dropbox, Google Drive, One Drive, iCloud, Mega, etc.
Interestingly, only some major solutions provide end-to-end encryption (E2EE) for the user uploading 
the data, normally only to premium or enterprise users~\cite{Dropbox,googleWorkspaceE2EE,Apple,Mega}.
This security feature is rather new and adopted only lately by the major players in the market.

The popularity of cloud storage services has increased in the last decades,
and projections show that they will store around 50\% of global data by 2025.
This amounts to around 100 zettabytes~\cite{SteveMorgan}.
Given this enormous amount of data and the wide range of users of these services, 
we believe the application of new primitives providing enhanced security guarantees
to data stored in the cloud is a relevant and timely research topic.

Our work investigates the application of a new scheme
to manage keys shared among a group of users (\cref{sc:gkp-scheme}),
the Group Key Progression primitive from Backendal, B{\'a}albas and Haller~\cite{GKP}, 
to secure file sharing in the cloud. 
Further, we generalise the concept of files to any data a group of users wants to share.
Thus, we expand the scope to cloud storage services like AWS S3, Azure Blob Storage, Google Cloud Storage, etc.
Sharing capabilities are not normally supported by default in these systems, but can be built on top of them.
We aim to construct
a system that can be used by real-world users.
We uncover issues in the schemes from~\cite{GKP}
that arise only when concretely applying them
to a real-world setting. The investigation also
highlights how abstractions normally used in academic
research are not standing the test of reality.

%\section{Secure Shared Folder}\label{sc:intro-SSF}

%We informally introduce a Secure Shared Folder (SSF) system and its
%characteristics.
%In \cref{sc:mental-model} we provide a detailed description
%of the mental model of the system, to be taken as a reference
%of the system we want to implement.
%Follows an informal description of the SSF scheme in \cref{sc:SSF-scheme}.
%The SSF scheme is the main focus of this work, where the
%cryptographic primitives from~\cite{GKP} are used as
%building blocks. More specifically, we will directly use the
%group key progression (GKP) primitive, of which we
%also report the description in \cref{sc:gkp-scheme}. 
%Finally, we describe the threat model for the SSF scheme in \cref{sc:threat-model}.

\section{Secure Shared Folder: the Mental Model}\label{sc:mental-model}

We give a high-level overview of the system we build,
which is the main focus of this work, 
called the ``Secure Shared Folder'' (SSF) system.
The SSF system aims to provide users, organised in groups, 
with the ability to share content securely within the group.
Users are identified through a Public Key Infrastructure (PKI),
which can be used to assign and verify identities.

Borrowing the terminology from well-known cloud storage providers offering collaborative file sharing,\footnote{Dropbox, Google Drive, One Drive etc.}
we introduce the concept of a ``shared folder'', or simply a folder in the system.
A folder contains one or more files, and its content is accessible to a group of users sharing it.
To this end, the users need to agree on a (multiple) shared secret(s) that is (are) used to protect the content cryptographically.
The storage space for the files, and possibly for the cryptographic and private state, is outsourced to a public cloud provider.
An example of a cryptographic state that might be outsourced includes the encryption keys of the files, if multiple ones are used, encrypted under the
shared secret. The shared secret is instead kept locally by each member. The private state could include sensible metadata of the files,
like the name, the author, etc. The server should not be able to access it.

The group composition is dynamic, meaning that the set of users granted access to a folder can change over time.
The goal of SSF is to enforce access control to file content through cryptography.
Finally, we assume an asynchronous setting, meaning that users can
perform operations even when other members are not online.
This assumption is required to allow the system to be used in practice
and is in line with other well-known systems providing file sharing.

\section{Summary of Contributions}\label{sc:summary-of-contributions}

We implement a minimal viable product (MVP) of the Secure Shared Folder (SSF) system.
Together with the MVP, we also provide a baseline implementation:
a simpler, less secure, version of cloud-based file sharing. Motivation
and details of the baseline are given in \cref{ch:baseline}.
We summarise the main contributions:

\begin{enumerate}
    \item We show how targeting a real-world setting is 
    beneficial to the cryptographic community and its 
    research, driving questions and possible solutions. 
    A synergy between cryptographers and practitioners 
    is beneficial and needed for both. It would help to 
    reduce the number of vulnerabilities normally found 
    in software we use every day and help design ecosystems 
    to develop better and more secure software. 
    Writing cryptographic software is hard, 
    and any mistake can compromise the security of the overall 
    system. This work shows how cryptography 
    should consider the actual runtime environment 
    and all other engineering problems in its formalisations. 
    These cannot be left as implementation details, 
    as (even good) software engineers normally lack the 
    knowledge to address such decisions. These might influence 
    the correctness of the system and break its security guarantees.
    \item We survey the major problems arising while 
    translating the scheme into a concrete, deployable 
    artefact. In particular, we list the ``Engineering Gaps''
    that are uncovered with this work (\cref{ch:gaps}). 
    This term refers to 
    the problems that are not normally addressed in the 
    academic research, but are crucial to the success of 
    the implementation.
    \item We discover and fix two major bugs in the 
    constructions which are the building blocks of 
    the SSF scheme (\cref{sc:GRaPPA-implementation}). 
    Together with our adjustments, we provide a 
    detailed explanation of the cause of the bugs and 
    propose an extension of the underlying primitives to 
    better model and cover the problematic cases.
\end{enumerate}

This thesis also contains several side contributions, 
which might be of interest to
practitioners embarking on a similar implementation journey:
\begin{enumerate}
    \item We provide an in-depth description of the project 
    setup, and detail the interoperability between programming
    languages, tools and libraries used, as well as 
    organization of our automated tests and component 
    virtualization. The usage of several technologies 
    together made the implementation possible in a 
    restricted timeframe by a single developer, 
    we believe this can be of great interest to many other 
    developers, and also for projects outside the crypto space.
    \item We provide a detailed description of our usage of 
    the Web Crypto API to construct the cryptographic 
    primitives needed to run the protocol. 
    Similar techniques could be used to implement other 
    cryptographic primitives normally found in the 
    literature but not directly available in the browser. 
    We also document portability issues we discovered
    in the Web Crypto API implementations across different 
    runtimes. Further, we provide the code for the workaround 
    we implemented to solve them.
    \item We give open-source access to the codebase for 
    further research and development.\footnote{The code will be made public as soon as the related research~\cite{GKP} is also published.}
\end{enumerate}

\section{Outline}\label{sc:outline}

In \cref{ch:background} we give a brief overview of the background, particularly,
we present the cryptographic primitives used as building blocks in the SSF scheme. 

The main contributions are then presented, divided as follows: 
in \cref{ch:setup} we describe the setup and technologies 
chosen for the implementation of both the baseline and 
the SSF scheme. We mainly describe the common choices here,
while going into more details later 
in \cref{ch:baseline,ch:ssf} for the specific choices 
in the two cases.
\cref{ch:baseline} describes the baseline implementation, 
its protocol and architectural design.
In \cref{ch:ssf} we describe the SSF implementation. 
We describe the related SSF scheme and then 
describe the implementation of the cryptographic
primitives which are used as well as the new server components
we introduce to support the SSF scheme. We
also detail the changes needed to the cryptographic primitives
from~\cite{GKP}.
In \cref{ch:gaps} we summarise the gaps uncovered during 
the implementation. This will provide the reader with 
all the lessons learnt through this work.
Finally, we conclude with a retrospective summary
and future work in \cref{ch:conclusion}.

Researchers interested in exploring how their assumptions 
can impact the system in a real-world setting and explore 
their implications can primarily read \cref{ch:gaps}, 
and follow the references to the other chapters as needed.
Readers especially interested in the implementation of the SSF 
scheme, as a complement to~\cite{GKP}, 
can read \cref{ch:ssf}. 
Again, backward references are provided to the 
relevant background and system design sections,
to allow easy navigation through the document.
Practitioners could instead focus primarily on
\cref{ch:setup}, especially looking at the
technology survey paragraphs which are interleaved 
to the system description. 
They can also read about practical 
lesson learnt in \cref{ch:gaps}.
