%% (Master) Thesis template
% Template version used: v1.4
%
% Largely adapted from Adrian Nievergelt's template for the ADPS
% (lecture notes) project.


%% We use the memoir class because it offers a many easy to use features.
\documentclass[11pt,a4paper,titlepage]{memoir}

%% Packages
%% ========

%% LaTeX Font encoding -- DO NOT CHANGE
\usepackage[OT1]{fontenc}

%% Babel provides support for languages.  'english' uses British
%% English hyphenation and text snippets like "Figure" and
%% "Theorem". Use the option 'ngerman' if your document is in German.
%% Use 'american' for American English.  Note that if you change this,
%% the next LaTeX run may show spurious errors.  Simply run it again.
%% If they persist, remove the .aux file and try again.
\usepackage[english]{babel}

%% Input encoding 'utf8'. In some cases you might need 'utf8x' for
%% extra symbols. Not all editors, especially on Windows, are UTF-8
%% capable, so you may want to use 'latin1' instead.
\usepackage[utf8]{inputenc}

%% This changes default fonts for both text and math mode to use Herman Zapfs
%% excellent Palatino font.  Do not change this.
\usepackage[sc]{mathpazo}

%% The AMS-LaTeX extensions for mathematical typesetting.  Do not
%% remove.
\usepackage{amsmath,amssymb,amsfonts,mathrsfs}

%% NTheorem is a reimplementation of the AMS Theorem package. This
%% will allow us to typeset theorems like examples, proofs and
%% similar.  Do not remove.
%% NOTE: Must be loaded AFTER amsmath, or the \qed placement will
%% break
\usepackage[amsmath,thmmarks]{ntheorem}

%% LaTeX' own graphics handling
\usepackage{graphicx}


%% This allows you to add .pdf files. It is used to add the
%% declaration of originality.
\usepackage{pdfpages}


%% Our layout configuration.  DO NOT CHANGE.
%% Memoir layout setup

%% NOTE: You are strongly advised not to change any of them unless you
%% know what you are doing.  These settings strongly interact in the
%% final look of the document.

% Dependencies
\usepackage{ETHlogo}

% Turn extra space before chapter headings off.
\setlength{\beforechapskip}{0pt}

\nonzeroparskip
\parindent=0pt
\defaultlists

% Chapter style redefinition
\makeatletter

\if@twoside
  \pagestyle{Ruled}
  \copypagestyle{chapter}{Ruled}
\else
  \pagestyle{ruled}
  \copypagestyle{chapter}{ruled}
\fi
\makeoddhead{chapter}{}{}{}
\makeevenhead{chapter}{}{}{}
\makeheadrule{chapter}{\textwidth}{0pt}
\copypagestyle{abstract}{empty}

\makechapterstyle{bianchimod}{%
  \chapterstyle{default}
  \renewcommand*{\chapnamefont}{\normalfont\Large\sffamily}
  \renewcommand*{\chapnumfont}{\normalfont\Large\sffamily}
  \renewcommand*{\printchaptername}{%
    \chapnamefont\centering\@chapapp}
  \renewcommand*{\printchapternum}{\chapnumfont {\thechapter}}
  \renewcommand*{\chaptitlefont}{\normalfont\huge\sffamily}
  \renewcommand*{\printchaptertitle}[1]{%
    \hrule\vskip\onelineskip \centering \chaptitlefont\textbf{\vphantom{gyM}##1}\par}
  \renewcommand*{\afterchaptertitle}{\vskip\onelineskip \hrule\vskip
    \afterchapskip}
  \renewcommand*{\printchapternonum}{%
    \vphantom{\chapnumfont {9}}\afterchapternum}}

% Use the newly defined style
\chapterstyle{bianchimod}

\setsecheadstyle{\Large\bfseries\sffamily}
\setsubsecheadstyle{\large\bfseries\sffamily}
\setsubsubsecheadstyle{\bfseries\sffamily}
\setparaheadstyle{\normalsize\bfseries\sffamily}
\setsubparaheadstyle{\normalsize\itshape\sffamily}
\setsubparaindent{0pt}

% Set captions to a more separated style for clearness
\captionnamefont{\sffamily\bfseries\footnotesize}
\captiontitlefont{\sffamily\footnotesize}
\setlength{\intextsep}{16pt}
\setlength{\belowcaptionskip}{1pt}

% Set section and TOC numbering depth to subsection
\setsecnumdepth{subsection}
\settocdepth{subsection}

%% Titlepage adjustments
\pretitle{\vspace{0pt plus 0.7fill}\begin{center}\HUGE\sffamily\bfseries}
\posttitle{\end{center}\par}
\preauthor{\par\begin{center}\let\and\\\Large\sffamily}
\postauthor{\end{center}}
\predate{\par\begin{center}\Large\sffamily}
\postdate{\end{center}}

\def\@advisors{}
\newcommand{\advisors}[1]{\def\@advisors{#1}}
\def\@department{}
\newcommand{\department}[1]{\def\@department{#1}}
\def\@thesistype{}
\newcommand{\thesistype}[1]{\def\@thesistype{#1}}

\renewcommand{\maketitlehooka}{\noindent\ETHlogo[2in]}

\renewcommand{\maketitlehookb}{\vspace{1in}%
  \par\begin{center}\Large\sffamily\@thesistype\end{center}}

\renewcommand{\maketitlehookd}{%
  \vfill\par
  \begin{flushright}
    \sffamily
    \@advisors\par
    \@department, ETH Zurich
  \end{flushright}
}

\checkandfixthelayout

\setlength{\droptitle}{-48pt}

\makeatother

% This defines how theorems should look. Best leave as is.
\theoremstyle{plain}
\setlength\theorempostskipamount{0pt}

%%% Local Variables:
%%% mode: latex
%%% TeX-master: "thesis"
%%% End:


%% Theorem environments.  You will have to adapt this for a German
%% thesis.
%% Theorem-like environments

%% This can be changed according to language. You can comment out the ones you
%% don't need.

\numberwithin{equation}{chapter}

%% German theorems
%\newtheorem{satz}{Satz}[chapter]
%\newtheorem{beispiel}[satz]{Beispiel}
%\newtheorem{bemerkung}[satz]{Bemerkung}
%\newtheorem{korrolar}[satz]{Korrolar}
%\newtheorem{definition}[satz]{Definition}
%\newtheorem{lemma}[satz]{Lemma}
%\newtheorem{proposition}[satz]{Proposition}

%% English variants
\newtheorem{theorem}{Theorem}[chapter]
\newtheorem{example}[theorem]{Example}
\newtheorem{remark}[theorem]{Remark}
\newtheorem{corollary}[theorem]{Corollary}
\newtheorem{definition}[theorem]{Definition}
\newtheorem{lemma}[theorem]{Lemma}
\newtheorem{proposition}[theorem]{Proposition}

%% Proof environment with a small square as a "qed" symbol
\theoremstyle{nonumberplain}
\theorembodyfont{\normalfont}
\theoremsymbol{\ensuremath{\square}}
\newtheorem{proof}{Proof}
%\newtheorem{beweis}{Beweis}



%% Make document internal hyperlinks wherever possible. (TOC, references)
%% This MUST be loaded after varioref, which is loaded in 'extrapackages'
%% above.  We just load it last to be safe.
\usepackage[linkcolor=black,colorlinks=true,citecolor=black,filecolor=black]{hyperref}

%% This allows you to use "\cref{sec:your-section}" instead of writing out 
%% "Section~\ref{sec:your-section}", also works for tables, figures, etc.
%% Use \Cref at the beginning of a sentence and \cref elsewhere.
%% Must be loaded after hyperref.
\usepackage[capitalise]{cleveref}


%% Document information
%% ====================

\title{Bridging the Gap: Implementing a Secure File Sharing System}
\author{Nicola Dardanis}
\thesistype{Master's Thesis}
\advisors{Advisors: Professor Kenny Paterson, Dr. Matilda Backendal, Dr. Matteo Scarlata}
\department{Applied Cryptography Group\\Institute of Information Security\\Department of Computer Science}
\date{October 18, 2024}

\begin{document}

\frontmatter

%% Title page is autogenerated from document information above.  DO
%% NOT CHANGE.
\begin{titlingpage}
  \calccentering{\unitlength}
  \begin{adjustwidth*}{\unitlength-24pt}{-\unitlength-24pt}
    \maketitle
  \end{adjustwidth*}
\end{titlingpage}

%% The abstract of your thesis.  Edit the file as needed.
\begin{abstract}
In this thesis we explore advanced security primitives
for data at rest in real-world applications: we choose
file sharing on public cloud as our target.
We design and implement
a Secure Shared Folder (SSF) scheme, on top of the primitives from~\cite{GKP}.
With this implementation,
we uncover issues in the cryptographic constructions, in academic
modelling and assumptions,
in the cryptographic ecosystem of the browsers
and in cryptographic libraries. 
Together, all the issues result in gaps
that need to be addressed by implementors 
to make the system production-ready.
The implementation uncovered errors in the pseudocode
of the primitives from~\cite{GKP}, which we fixed.
To solve the errors, we 
contribute to the primitives with enhancements to
better model and fit the reality of their applications. 
With this work, we provide also a survey of the state-of-the-art
of the technology we considered during the implementation,
to serve as a guide for future researchers and developers 
which dare to take the same path.
Together with this thesis, we give public access to the repository
made up of more than 50000 lines between code, tests and documentation.
\end{abstract}


%% TOC with the proper setup, do not change.
\cleartorecto
\tableofcontents
\mainmatter

%% Your real content!
\chapter{System Requirements}\label{ch:system-requirements}

\section{Overview}

Due to the flexibility of CGKA in combination with key regression, 
the shared folder scheme can cater to a variety of different requirements and threat models.
This is part of the goal of the theoretical side of this
project: ideally, we wish do develop a set of systems which, depending on the strength of the adversary,
provides different levels of security. While all the systems in the set should share the same basic building
blocks, they might behave rather differently on an implementation level. Hence, the implementation
needs to restrict the setting such that the system requirements are clear.

In Section \cref{sc:mentalmodel} we introduce the abstract model of a Secure File Sharing construction (SSF).

In Section \cref{sc:threatmodel}, we describe the threat models that were considered in the theoretical side of this project
and we motivate the choice we made for the implementation. 

In Section \cref{sc:securityguarantees}, we introduce an informal description of the security guarantees 
that the implemented system aims to provide.

In Section \cref{sc:functionalrequirements}, we reason around the functionalities that the system should support in a real world setting.
More specifically, we aim to the develop a Minimal Viable Product (MVP).

\section{The Mental Model of Secure File Sharing} \label{sc:mentalmodel}

A Secure File Sharing (SSF) system aims to provide users, organised in groups, 
with the ability to share contents in within the group in a secure way.

Borrowing the terminology and mental model from well-known cloud storage providers offering collaborative file sharing
~\footnote{Dropbox, Google Drive, One Drive etc.},
we introduce the concept of a ``shared folder'', or simply a folder in the system.
A folder contains one or more files, and its contents are accessible to a group of users sharing this contents.
To this end, the users need to agree on a shared secret that is used to protect the contents through cryptographic means.
The actual storage space is outsourced to a public cloud provider.

In addition, the group composition is dynamic, meaning that the set of users which has granted access to a folder can change over time.
On the theoretical side of this project, the research explored the applicability of Continuous Key Group Agreement schemes,
normally used in messaging applications, in order to efficiently handle a shared secret within a dynamic group of users.

\section{Threat Model} \label{sc:threatmodel}


\section{Security Guarantees} \label{sc:securityguarantees}

\section{Functional Requirements} \label{sc:functionalrequirements}

In industry terminology a first implementation which can be launched as a product is called Minimal Viable Product (MVP).
This project aims to explore such a real-world scenario, instead of developing a prototype implementation of the cryptographic construction.
To this end, we need to define functional requirements which a real-world user will want to consider the product usable, formulating them in a similar fashion to User Stories:
\begin{itemize}
    \item Accessibility: as a user I want my contents to be always accessible until deletion, possibly from any device I own capable to connect to the cloud provider.
    \item Efficiency: as a user I want to retrieve my contents efficiently, as well as uploading or sharing.
    \item Visibility of System Status: as a user I want to be able to retrieve the set of folders I have access to, list the contents of each of them and get feedback on operations I perform.
\end{itemize}





\chapter{Reference Implementation}\label{ch:reference-implementation}

\section{Overview}

In this chapter we will present the design and implementation of the reference implementation, 
so called ``Baseline'' borrowing from Benchmarking terminology, 
which is a simpler version of the system offering lower guarantees in terms of security,
using a na\"ive cryptographic construction to manage the folder state.
This implementation is first addressed as it is much simpler in terms of cryptography,
but still requires all the supporting infrastructure that will be reused also for the SSF scheme development.

Already the implementation of this simpler system brings to light many problems that in the abstract model are not considered which we will use to justify our choices.
We call those problems ``Engineering Gaps'', as we uncover them only while actually translating the model in real, working code for an MVP.
Those engineering gaps guide also our choices for the tech stack to use.

\section{Baseline Specification}\label{sc:baseline-specification}

The idea behind implementing a Baseline is to benchmark the application to compare performance degradation of the whole system when heavier cryptography is used.
The Baseline offers a limited set of security guarantees and a simplified architecture. 
The protocol uses simpler cryptographic primitives, requiring also less engineering effort.
However we want to be able to compare the results to the best of our possibilities. 
To this end, we aim at re-using all of the components and libraries that can be shared and we develop,
as well as using the same set of techologies, such as the programming language, the execution platform etc, 
that might affect the performance, thus minimising the differences.

Recalling the Mental Model described in Chapter \cref{ch:system-requirements}, Section \cref{sc:mentalmodel}, 
we are going to describe specifically the Baseline's Model:
\begin{itemize}
    \item Each user is represented as its own long term public keys. A user can register itself to the system using the identity provided by the PKI.
    \item The folder has a metadata object associated with it. This contains the Folder Key (\texttt{Fk}), a shared key among all participants to the shared folder, that is encrypted under the public identity of each of the participants.
    \item Each file is encrypted under a random generated key uniquely used for that file. Those are also part of the metadata object and encrypted under \texttt{Fk}.
\end{itemize}

The instantiation of this simplified protocol is based on the following cryptographic primitives:
\begin{itemize}
    \item Elliptic Curve Integrated Encryption Standard (ECIES) to share the \texttt{Fk} from one user to another, without bounding the sender identity in the encryption.
    \item AES-GCM for the encryption under \texttt{Fk} of the file keys and the metadata of the files, i.e. the name.
    \item AES-GCM for the encryption of files. 
\end{itemize}

\section{Starting with the User in mind} \label{sc:starting-with-the-user}

The real-world settings is central in our implementation work. 
The exploration starts indeed by taking into consideration both the requirements from the cryptographic construction as well as the expectation of a User to approach the implementation using the right technology stack.
A first expectation of modern users is the ability to access services from any device that can navigate online.
A easy and portable solution is to let the user access the system from a Web Browser.

\section{The Gaps between Code and Math: Devices, Execution Platforms and Implementation Efforts}

\paragraph{Devices} are normally completely abstracted away during the formalisation of a cryptographic scheme. 
They are treated as black boxes, having some abstract capabilities, like CPU, memory, storage and network connectivity.
However, in everyday life, we use many different devices with very unhamrmonized capabilities, such as smartphones, laptops, desktop computers, tablets etc.
All of the above devices can have very different capabilities, starting from the hardware to the software layers like the operating system.
A goal of the MVP implementation is therefore also code portability, the ability of the code we write to run on such heterogenous platforms.

\paragraph{Crypto primitives} in the theoretical constructions are mathematical objects, pure functions.
As such support of those mathematical functions becomes an implementation detail that is abstracted away, also given that the device as described above is also thought as a set of harmonized computational capabilities.
When transalting those methematical objects into code, we need to check if those primitives are supported by the execution platforms we are working on. 
This is especially important to guarantee security properties carried by the crypto primitives, for example if constant time operations or secure memory deletion operations are required.
Working inside a Web Browser is therefore very different from working in a Desktop environment, as the set of libraries providing such guarantees deeply changes. 
In Web environments we can rely on Web Cryptography API which is a JavaScript API natively supported by all major browsers providing native, although very limited, support for cryptographic operations.
In other execution environments, like a desktop application, way more libraries are available with native support for many more cryptographic operations.

\paragraph{Prototyping} can be addressed simply by writing on a whiteboard and brainstorm with the other researchers when thinking about the theoritical construction. 
Most of the effort will be spent later in later trying to prove the soundness of the constructions and ideas that come out from the design of the scheme.  
Coding the solution generally requires instead way more human time.
Furthermore, once a decision is taken is really difficult to go back, because it might involve changing thousands of lines of code.
However during the implementation a feedback loop naturally takes place, where the ideas and construction from the theoretical side guide the protocol implemntation and the implementation uncovers problems and guide further research.

\section{Client Scaffolding}
To summarise and clarify, while choosing the technology stack for the client code, we need to satisfy the following requisites:
\begin{itemize}
    \item Runnable in a Web Browser.
    \item Possibility to run CGKA to be able to develop the actual protocol later.
    \item Development agility and possibility to prototype solutions avoiding unnecessary work.
    \item Simple to benchmark, while still provide easy integration of more user-friendly components.
\end{itemize}

\paragraph{CGKA} is normally studied as a core component of Messaging Layer Security (MLS) protocol.
Most of the open source implementations are indeed part of libraries implementing the full or partial MLS protocol.
The most relevant ones to the best of our knowledge are:
\begin{itemize}
    \item OpenMLS, available in multiple languages but not production-ready. For instance, it doesn't support X509 certificates for identities.
    \item Java BouncyCastle includes a CGKA only library.
    \item AWS-Lab Rust\footnote{https://www.rust-lang.org/} library, a full implementation of MLS sponsored by Amazon Web Services (AWS). 
\end{itemize}

Other minor implementation are available, but are mostly broken or outdated.
The only solution that is both production-ready and is runnable in the browser is the AWS-Lab Rust library, thanks to the bindings to WebAssembly.
Those bindings, allow us to compile the Rust code, normally not executable in the browser, in WebAssembly, which is instead supported alongside JavaScript.
The importance of this compilation target is due to the fact that the code is not only translated in JavaScript through compilation, but it also rely on the native Web Cryptography API for cryptographic operations and therefore provides us with all the security guarantees for a correct implementation.
This address the first two requisites above.

\paragraph{Development agility and benchmarking} of the client code is achievable by writing code that is supported in both browser and desktop environments.
In the JavaScript world, a natual choice is to target Chrome browser and NodeJs:
\begin{itemize}
    \item Chrome is a major browser used in millions of devices.
    \item NodeJs is another JavaScript / WebAssembly\footnote{https://webassembly.org/} execution platform, that runs natively on the Operating System.
    \item Both Chrome and NodeJs internally use the V8 Virtual Machine, an executor for JavaScript and WebAssembly.
    \item The Web Cryptography API that are required both to run CGKA and to write all other cryptographic components are available inside both execution environments.
\end{itemize}

To further ease the development work, the client code is all developed in TypeScript\footnote{https://www.typescriptlang.org/}, a superset of JavaScript adding types to the language, so that when properly configured a vast class of 
errors are statically checked instead of discovered only later through testing.

\section{The Identity Gap: develop a PKI}

Papers in cryptography normally assume the existance of a Public Key Infrastructure (PKI).
This is usually used to solve the identities problem.
Since we want to be able to easily develop the code, we rely for the real implementation on standard X509 certificates.
A simple CA server is available in the project repo, exposing a simple API. This CA server could be further developed for a real installation.
The MVP ideally targets a company or organization as a real use-case. 
Thus, it makes sense to have an internal identity provider.

Once running the server as described in the project repo instruction, the endpoints are available under \texttt{https://localhost:8000/swagger-ui}.

This server is implemented in Rust, using Rocket\footnote{https://rocket.rs/} framework to develop a concurrent server.
The choice of the language in this case is driven by the fact that Rust is already part of the technologic stack and given its safety guarantees in terms of memory and types is a natural choice to pick for cryptographic related work.
All of the server is developed in \textbf{safe} Rust.
Furthermore, with the ability to compile the code in WebAssembly, we are re-using some of the X509 parsing and validation code inside the client.
A major downside of this choice is that currently Rust as a language and in terms of available frameworks is not yet mature enough for Web Development, compared to older techonologies like Java\footnote{https://docs.oracle.com/javase/8/docs/technotes/guides/language/index.html} and Spring\footnote{https://spring.io/} or C\#\footnote{https://learn.microsoft.com/en-us/dotnet/csharp/} and ASP.NET\footnote{https://dotnet.microsoft.com/en-us/apps/aspnet}.

The public certificates are stored inside a MySQL\footnote{https://www.mysql.com/} db, which is dockerized\footnote{https://www.docker.com/} and run as a container, allowing for clean starts and ease the testing and installation process, as well as portability on different OS.

\section{Communication Channels}

In the theoretical model, communication channels are abstract and exist without introducing any further complexity in terms of time spent to send messages.
In our implementation, we provide secure and efficient communication channels, by adopting Transport Layer Security (TLS).
Client code bundles the PKI server public certificate to be able to verify the PKI server identity when establishing connections to it.

Furthermore, the model doesn't assume either a client-server or a peer-to-peer architecture, leaving the choice to the implementation.
Due to time constraints, in this project we take a client-server approach for the implementation of the SSF scheme, where clients are not sending messages between themselves but always relying on a server.
This simplify development, and can be seen as a fair approach, given the fact that we want anyway to outsource the storage to an external cloud provider, which anyway embodies the server in the architecture.

Considering the development agility and the need for prototyping in the implementation, we reduce the time spent to write and change the client code by generating the code calling the server API using OpenAPI Specification (OAS)\footnote{https://swagger.io/specification/} and compatible code generators.
OAS define a standard, language-agnostic interface to HTTP APIs which allows both humans and computers to discover and understand the capabilities of the service without access to the source code.
More in detail:
\begin{enumerate}
    \item we annotate the server code with utoipa\footnote{https://github.com/juhaku/utoipa};
    \item from those annotations we generate a yaml\footnote{https://yaml.org/} file containing the OAS of the server;
    \item we read the specification from yaml and generate the TypeScript code using @hey-api/openapi-ts\footnote{https://github.com/hey-api/openapi-ts} code generator.
\end{enumerate}
In the client code, we add a layer of abstraction between the generated code and the actual calls to the generated code, to minimise the amount of changes required in case of a server update.

\begin{figure}
    
\end{figure}


\appendix

\chapter{Appendix}\label{ch:sample-appendix}


\backmatter

\bibliographystyle{plain}
\bibliography{refs}

\includepdf[pages={-}]{declaration-originality.pdf}

\end{document}
