%% (Master) Thesis template
% Template version used: v1.4
%
% Largely adapted from Adrian Nievergelt's template for the ADPS
% (lecture notes) project.


%% We use the memoir class because it offers a many easy to use features.
\documentclass[11pt,a4paper,titlepage]{memoir}

%% Packages
%% ========

%% LaTeX Font encoding -- DO NOT CHANGE
\usepackage[OT1]{fontenc}

%% Babel provides support for languages.  'english' uses British
%% English hyphenation and text snippets like "Figure" and
%% "Theorem". Use the option 'ngerman' if your document is in German.
%% Use 'american' for American English.  Note that if you change this,
%% the next LaTeX run may show spurious errors.  Simply run it again.
%% If they persist, remove the .aux file and try again.
\usepackage[english]{babel}

%% Input encoding 'utf8'. In some cases you might need 'utf8x' for
%% extra symbols. Not all editors, especially on Windows, are UTF-8
%% capable, so you may want to use 'latin1' instead.
\usepackage[utf8]{inputenc}

%% This changes default fonts for both text and math mode to use Herman Zapfs
%% excellent Palatino font.  Do not change this.
\usepackage[sc]{mathpazo}

%% The AMS-LaTeX extensions for mathematical typesetting.  Do not
%% remove.
\usepackage{amsmath,amssymb,amsfonts,mathrsfs}

%% NTheorem is a reimplementation of the AMS Theorem package. This
%% will allow us to typeset theorems like examples, proofs and
%% similar.  Do not remove.
%% NOTE: Must be loaded AFTER amsmath, or the \qed placement will
%% break
\usepackage[amsmath,thmmarks]{ntheorem}

\usepackage{crypto-environments}

%% LaTeX' own graphics handling
\usepackage{graphicx}

%\usepackage{crypto-game-environments}

%% This allows you to add .pdf files. It is used to add the
%% declaration of originality.
\usepackage{pdfpages}


%% Our layout configuration.  DO NOT CHANGE.
\input{layoutsetup}

%% Theorem environments.  You will have to adapt this for a German
%% thesis.
\input{theoremsetup}


%% Make document internal hyperlinks wherever possible. (TOC, references)
%% This MUST be loaded after varioref, which is loaded in 'extrapackages'
%% above.  We just load it last to be safe.
\usepackage[linkcolor=black,colorlinks=true,citecolor=black,filecolor=black]{hyperref}

%% This allows you to use "\cref{sec:your-section}" instead of writing out 
%% "Section~\ref{sec:your-section}", also works for tables, figures, etc.
%% Use \Cref at the beginning of a sentence and \cref elsewhere.
%% Must be loaded after hyperref.
\usepackage[capitalise]{cleveref}


%% Document information
%% ====================

\title{Bridging the Gap: Implementing a Secure File Sharing System}
\author{Nicola Dardanis}
\thesistype{Master's Thesis}
\advisors{Advisors: Professor Dr. Kenny Paterson, Matilda Backendal, Matteo Scarlata}
\department{Applied Cryptography Group\\Institute of Information Security\\Department of Computer Science}
\date{October 28, 2024}

\begin{document}

\frontmatter

%% Title page is autogenerated from document information above.  DO
%% NOT CHANGE.
\begin{titlingpage}
  \calccentering{\unitlength}
  \begin{adjustwidth*}{\unitlength-24pt}{-\unitlength-24pt}
    \maketitle
  \end{adjustwidth*}
\end{titlingpage}

%% The abstract of your thesis.  Edit the file as needed.
\begin{abstract}
In this thesis, we explore advanced security primitives
for data at rest in real-world applications: we choose
file sharing on public cloud as our target.
We design and implement
a Secure Shared Folder (SSF) scheme, 
on top of the Group Key Progression primitive from Backendal, B{\'a}lbas and Haller.
With this implementation,
we uncover issues in the cryptographic constructions, in academic
modelling and assumptions,
in the cryptographic ecosystem of the browsers
and in cryptographic libraries. 
Together, all the issues result in gaps
that need to be addressed by implementors 
to make the system production-ready.
The implementation uncovered errors in the pseudocode
of the primitives from Backendal et al., which we fixed.
To solve the errors, we 
contribute to the primitives with enhancements to
better model and fit the reality of their applications. 
With this work, we also provide a survey of the state-of-the-art
of the technology we considered during the implementation,
to serve as a guide for future researchers and developers 
which dare to take the same path.
Our implementation is publicly available on GitHub as a reference.
We seek to shed light on the many problems developers face 
in implementing cryptography
in real applications. These problems
should be addressed by the cryptographic community
together with other organizations that maintain the standards
of the most widely used technologies to date
to allow and simplify the development of more secure applications.
\end{abstract}
    

%% TOC with the proper setup, do not change.
\cleartorecto
\tableofcontents
\mainmatter

%% Your real content!
\chapter{Introduction}

Nowdays storing data in public cloud storage systems is a well adpoted practice,
from private users to business customers, 
including even software companies relying on major cloud providers as a storage solution.
A broad variety of data is indeed stored in major cloud providers, even highly sensitive data,
such as medical records.
In cloud environments, data is stored in various locations and normally it is replicated
around the world. Users don't have any physical access to their sensitive data.
Furthermore, cloud providers are normally multi-tenat, meaning that multiple customers
can use their computing power and storage space at the same time, sharing the physical hardware.
While internet communications and instant messaging are normally End-to-end encrypted (E2EE)




\section{Problem}

Protecting data in transit is a well-studied problem.
Security properties such as forward security (FS) and post-compromise security (PCS)
have been formalised for encrypted communication channels.
The ephemeral nature of data in transit allows naturally the formalisation of security models
with such properties.
Protocols, such as Signal and MLS, are implementing these properties in the messaging area,
establishing those properties for end-to-end encrypted (E2EE) communication channels.  
The security models and properties for data at rest is instead lgging behind.
While E2EE encrypted storage solution exist and are deployed by major cloud providers,
little attention is yet given to the achievable security properties 
and wheteher similar or same level of guarantee can be achieved as for data in transit.
Furthermore, common capability of sharing such persistent data greetly extends the attack surface.
Sharing of data can resemble a group chat, however, the \textbf{persistent}.



\section{Theoretical Solution}


\section{Focus of this Thesis}

The focus of this work is the implementation of the proposed theoritical solution,
closing the gaps between the abstract logical world of the proposed scheme
and the actual real-world system.
The implementation should be conducted following the best engeeniring practices 
and guarantee the security expressed in the theoritical design.
A great attention is given to the real-world aspect of this implementation,
posing challenges that would be otherwise uncovered in just a proof-of-concept implementation.
The final outcome of the engineering work is primarly a minimal viable product (MVP) executing the protocol.

\subsection{Summary of Contributions}

We implement an MVP running the secure file sharing scheme (SSF) and a baseline implementation:

\begin{itemize}
    \item We benchmark the SSF implementation against the baseline.
    \item We show how targeting a real-world setting is beneficial to the cryptographic community and its research, driving questions and possible solutions. A sinergy between cryptographers and practicioners is beneficial and needed to both, helping mitigate the amount of vulnerabilities normally found in software we use everyday.
    \item We survey the major problems arising while translating the scheme into a concrete, deployable artifact. In particular, we list the ``Engineering Gaps'' that are uncovered with this work.
\end{itemize}

\section{Outline}
\chapter{Backround}\label{ch:background}

In this chapter, we introduce the theoretical
background for our implementation, especially regarding the Secure Shared Folder 
(SSF) scheme. We introduce some notation (\cref{sc:notation}) and
describe existing cryptographic primitives (\cref{sc:SSKG,sc:DKR,sc:CGKA})
that are used in the scheme.
We also provide an informal description of the novel primitives 
underlying the SSF scheme and their operations (\cref{sc:SSF}).

\section{Notation}\label{sc:notation}

We assume the following conventions:
\begin{itemize}
    \item Intervals are inclusive, i.e., $[a, b] = \{x \in \mathbb{N} \mid a \leq x \leq b\}$.
    \item A sequence of elements of length $n$ starts at index $0$ and ends at index $n-1$, as in common practice in programming languages.
    \item \texttt{Monospace font} is used for references to pseudocode and code references in running text.
\end{itemize}

\section{Double-PRF}\label{sc:DPRF}

A pseudorandom function family (PRF) $F$ is a collection of functions
with two arguments $x$ and $y$, respectively a key and a message
represented as bit strings, which returns bit string $z$ as output.
Informally we think of a PRF as being \textit{secure} if, for any given $x$
uniformly sampled from the key space, the function of one argument
$F_x(y) = F(x, y)$ is indistinguishable from a random function, i.e.
an efficient adversary cannot distinguish with a significant advantage
between a function chosen randomly from the PRF family and a random oracle.
A \textit{swap-PRF} is a PRF where swapping the keys for the messages
still results in indistinguishability from random~\cite{EPRINT:BelLys15}.
When $F$ is both a PRF and a swap-PRF, it is called a \textit{dual-PRF}.
As analysed and proved by Backendal et al.~\cite{C:BBGS23} HMAC is not always a
dual-PRF. However, if fixed-length keys are used, no matter the length,
then it is provably a dual-PRF.

The \textit{double-PRF} security notion by Backendal et al.
is slightly stronger than dual-PRF security.
It requires the function family to be indistinguishable from a random 
function not only when keyed either by the first input or the second input
but as well as when keyed through both inputs at the same time.
This means that an adversary simultaneously has access to both 
oracles for the PRF and its swap-PRF.
They also show that double-PRF security is implied by dual-PRF security,
thus HMAC is a double-PRF in the case of fixed-length keys.
See \cref{sc:ssf-double-prf} for details on the implementation of double-PRF in SSF.

\section{Dual-Key Regression}\label{sc:DKR}

\begin{figure}
    \centering
    \includegraphics[width=0.8\textwidth]{figures/dkr.drawio.png}
    \caption{Dual-Key Regression (DKR), displaying the forward and backward chains and corresponding sequence of secrets derived through the application of the KDF. }
    \label{fig:dkr}
\end{figure}

A hash chain is a sequence of values
$h_{0}, ..., h_{n}$ which starting from an initial
element $h_0$ progresses by iteratively calculating
a given hash function $H$ on the predecessor,
i.e., given $h_i$, the next element $h_{i+1}$ is computed as $h_{i+1} = H(h_i)$.
Due to the pre-image resistance of the hash function $H$, 
it is computationally hard to find $h_i$ given $h_{i+1}$.
We call an ``interval'' $[a, b]$ a subsequence of elements
of a hash chain starting from index $a$ and ending
on index $b$ inclusive of the chain.
Dual-key regression (DKR) introduced by Shafagh et al.~\cite{USENIX:SBRH20} is a
cryptographic primitive based on hash chains, that
allows sharing a potentially very large interval 
of secrets by only sharing a small state.
Dual-key regression (\cref{fig:dkr}) uses a pair of hash chains,
a ``forward'' and ``backward'' chain, with an additional
parameter $N$ denoting the
upper bound on the maximum length of each chain.
A forward chain is just a hash chain as defined
above, $f_{0}, ..., f_{N}$.
A backward chain instead is a hash chain, of which the
elements are released in reverse order of computation
$b_{N}, ..., b_{0}$, meaning that the element of the chain 
at index $i$ will be the element calculated by $N - i$
iterations of the hash function on the initial $b_{0}$
element. Notice that the DKR upper bound parameter
$N$ is imposed by the fact that elements of the backward chains are taken in
reverse order and the hash function cannot be inverted.
Finally, a DKR secret is calculated by applying
a key derivation function (KDF) that takes as input elements from both chains at the same time corresponding to the same index, i.e.\!
elements $f_{i}$ and $b_{(N - i)}$.
The KDF we will use will be a double-PRF (\cref{sc:DPRF}, \cref{sc:ssf-double-prf}).


\section{Seekable Sequential Key Generators}\label{sc:SSKG}

Seekable Sequential Key Generators (SSKG) are cryptographic objects introduced by Marson et al.~\cite{ESORICS:MarPoe13}.
They are sequential pseudo-random generators (SRG) with additional properties.
A sequential random generator is a stateful pseudo-random generator (PRG)~\cite{cryptoeprint:2017/208}, 
which outputs a fixed-length string for each invocation, 
thus producing a sequence of pseudo-random strings through multiple invocations.
A SRG is said to be secure if its output is indistinguishable from a uniformly random sampled string.
SSKG are SRG with the additional property of being seekable,
meaning that they offer a convenient operation to compute the
element at a certain offset in the sequence from the initial one.
This operation is called \texttt{Seek} and runs in sublinear time.

SSKG are widely used in practice:
the logging service of the \texttt{systemd} system manager,
which is a core component in many Linux-based operating systems,
is using them to power fast verification of arbitrary log entries against tampering.
SSKG are provably (forward-)secure in the standard model, i.e.,
an adversary gets no advantage from learning the current output of the SSKG
while trying to compute the past outputs.
SSKG can be seen as time-efficient alternatives to hash chains (\cref{sc:DKR}),
when the \texttt{Seek} operation is required.
Marson et al.\ propose two constructions: one is based on number theory~\cite{ESORICS:MarPoe13}
and the other one uses a tree-based construction~\cite{ESORICS:MarPoe14}.
The latter suffers a logarithmic space overhead 
compared to the constant space required to store 
only the starting element of a hash chain.
However, in our implementation, we will be using the later tree-based version of SSKG~\cite{ESORICS:MarPoe14}
because it allows seeking also starting from any point in the sequence
instead of just from the initial element.
We discuss the practical details more in-depth in \cref{sc:ssf-sskg}.

\section{Continuous Group Key Agreement}\label{sc:CGKA}

A continuous group key agreement (CGKA) scheme~\cite{C:ACDT20}
allows a long-lived, dynamic, asynchronous group of users to agree 
continuously on shared symmetric secrets.
The shared group secret is recomputed both as users add (remove)
other members to (resp.\ from) the group, and when users periodically
refresh their private secret state. These operations can happen
asynchronously, allowing group members to refresh their 
state without requiring all other members to be online 
simultaneously.
CKGA allows a group of $n$ users to perform the above critical 
operations in $\log(n)$ time,
thus it can be used in practice with a large number of members.
The primitive guarantees post-compromise security (PCS) and forward secrecy (FS):
in case of a compromise in which a user's secret is leaked
to an adversary, the group keys should shortly become private
again through the ordinary protocol state refreshes; the past
group keys remain secure as well.

A first ends-to-ends\footnote{While in communication protocols between two entities (where an entity is commonly referred to as endpoint) we normally speak about end-to-end encryption, in case of a group setting we use the plural form for the term.}
encrypted (E2EE) primitive for an asynchronous group key exchange 
has been introduced by Cohn-Gordon et al.~\cite{CCS:CCGMM18}. The research was conducted to address the scenario of secure group messaging.
In such a case multiple users want to exchange messages, asynchronously
between them securely, which can be reduced to the problem
of asynchronously agreeing on a shared symmetric key.
The authors presented a way to achieve PCS in such cases.
To this end, they designed the Asynchronous Ratcheting Trees (ART),
which is internally using a Diffie-Hellman tree~\cite{10.1145/1368310.1368347} 
to represent the public and private secrets for each user
and derive a shared secret from them. In Diffie-Hellman
trees, the user keypair is stored in the leaves.
This primitive offers the operations to add and remove members
as well as refreshing a user's own secret efficiently in $\log(n)$ time.
It has been of significant interest in the industry and
has a dedicated working group by the IETF.
The ART construction has been replaced by the TreeKEM
construction proposed by Bhargavan et al.~\cite{TreeKEM}.
In TreeKEM, members are organised in a tree-shaped structure
as in ART, however, the keys stored in the tree for each group
can be any keypair supporting key encapsulation (KEM).
The key advantage of TreeKEM over ART is that
most operations are ``mergeable'':
any device receiving two
concurrent operations will be able to process and execute both of them,
instead of executing one and refusing the other.
Alwen et al.~\cite{C:ACDT20} slightly modified the TreeKEM construction
to achieve \textit{optimal} security in the context of
secure group messaging.
Further refinements for efficiency and security have 
been studied in recent years under different assumptions and threat models~\cite{TCC:ACJM20, SP:KPWKCCMYAP21, CCS:ACDT21, CCS:AHKM22, EC:AANKPPW22, C:AlwJosMul22, C:AlwMulTse23, IWSPA:KEONO23}.

Of particular interest is the family of CGKA protocols called admin-CGKA (A-CGKA).
In this declination of CGKA, a subgroup of the users are admins. 
Admins can perform additional operations that are otherwise disallowed, such as removing other members of the group.

As we will see in \cref{sc:SSF-scheme}, our secure shared folder (SSF) scheme uses A-CGKA as a building block
inside the GKP scheme (\cref{sc:gkp-scheme}).
The specific version of A-CGKA we use, called dual-CGKA, is composed of a CGKA within CGKA to manage the admin state, as proposed by
B{\'a}lbas et al.\!~\cite{USENIX:BalColVau23}.

\section{Messaging Layer Security}\label{sc:MLS}

As already discussed in \cref{sc:CGKA}, a major application of CGKA is in the field of group messaging.
The messaging layer security (MLS) protocol from IETF~\cite{rfc9420},
a standardised protocol for secure group messaging, is indeed 
built on top of CGKA. CGKA is used to agree on a shared symmetric key
to later derive message encryption keys.

CGKA instantiations rely on message exchange
between the members of the group to perform the protocol operations.
Therefore, it is not surprising that most of the implementations
of CGKA come from libraries that also implement the MLS protocol (\cref{sc:CGKA-implementations}).
The messages needed for the CGKA operations are referred to as ``control'' messages in the context of MLS.
CGKA operations are based on a proposal and commit mechanism,
where (multiple) users can propose (multiple) group state updates
through proposal messages and then commit to them through a commit message
sent by one member, referring to a list of previously shared proposals. 
The proposals are of different types and model the operation of adding or removing 
members as well as refreshing a user's secret state, as seen above (\cref{sc:CGKA}).
The commit messages are used to agree and finalise the state updates,
thus helping to synchronize the group state among the members.

The MLS protocol specification also offers an abstract overview of
the component and architecture of a system using it, 
where the concept of a ``delivery service''
(DS) is introduced. The DS is a facilitator component that is used to 
deliver the messages between the members of the group and helps in
synchronising the group state. The DS is assumed to reliably
send the messages to the group members in order, so that the
group is able to advance the CGKA cryptographic state reliably.
However, the DS is not part of CGKA itself,
but it is a necessary component to implement the scheme in practice.
Further, the MLS protocol offers a way to securely exchange ``application messages''
on top of CGKA. The application messages are just opaque payloads
without any predefined structure or semantic, in contrast to the
CGKA control messages.
Application messages are encrypted with a key derived from the shared group secret.

As we will see in \cref{ch:setup}, we use a library implementing MLS and not just CGKA in our implementation of SSF.
We will make use of some additional features from MLS that are not part of CGKA, such as the
aforementioned encrypted application messages,
and take inspiration from the system architecture of MLS for our architecture design.
We discuss the architecture in further details in \cref{ch:setup}.

\section{Secure Shared Folder Building Blocks}\label{sc:SSF}

In this section, we describe the
contributions from~\cite{GKP}
which are going to be used
in the SSF scheme.
We start by recalling the novel security notion for persistent
data called interval access security (IAC) (\cref{sc:iac}).
Then we describe the generalised dual-key regression and group key progression primitives
(\cref{sc:background-generalised-DKR,sc:gkp-scheme}).

\subsection{Novel Security Notion for Persistent Data: Interval Access Security}\label{sc:iac}

The SSF (\cref{sc:mental-model}) is a new primitive that targets a novel security 
notion for persistent data shared among a dynamic 
group of users: interval access control (IAC).
The new security notion aims to better and more naturally
capture the minimal security for shared persistent data.

In short, IAC requires that a user can only decrypt data that
is shared in the time frame in which the user is a member
of the group. For data in transit, the data is regarded as
ephemeral and therefore the creation and sharing of the data
itself is naturally bound to the point in time of the exchange,
and to its deletion. Persistent data on the other hand, naturally
outlive the moment in time in which it is shared.
Therefore, the lifetime of data at rest is decoupled from the
lifetime of the user access to the group exchanging the data.
Thus, a user that receives access to some data while being the
member of a group, will be able to still decrypt the data
after he loses membership, unless the data is re-encrypted
and the old ciphertext is securely deleted, because data persistence prevents the automatic expiration of the granted access.
In the opposite direction, access to data that has been
shared before the user joined the group, could also be granted
to new members.

\subsection{DKR for Unlimited Key Derivations with IAC}\label{sc:background-generalised-DKR}

The SSF construction uses a generalisation of DKR (\cref{sc:DKR}).
The main issue with plain DKR is that it comes with a limit
on the number of derivable keys, which is imposed by the
length of the backward chain. The generalisation aims to
remove this limit.
In the generalised DKR a user can derive a virtually infinite 
sequence of keys.
Each point in the key sequence is associated with one \textit{epoch}, 
which represent the (discrete) time in which the key sequence is advanced.
The term \textit{epoch} can be found also in CGKA and 
MLS~\cite{rfc9420} with similar meaning.
An epoch interval $[a, b]$ is constituted by the subsequence of keys
from epoch $a$ to epoch $b$. We call each key an \textit{epoch key}, and associate
the key with the epoch it belongs to.

We observe that a na\"ive generalisation of DKR allowing
infinite keys to be derived is easily achievable: we can keep the 
forward chain and just start a new backward chain after the 
current backward chain is fully released,
i.e. each $N$ epochs, where $N$ is a single backward chain maximum length.
The state thus grows over time, with the progression of epochs, as the 
user needs to store all the starting backward chains elements. 
Precisely, the space needed is $1 + epoch_{max} / N$, 
where $epoch_{max}$ is the current epoch.

However, by maintaining the same forward chain in the key progression,
a user who has access to epoch interval 
$[t_j, t_{j + c}]$ and $[t_i, t_{i + d}]$, 
where $t_{j + c} < t_i$ and $t_i - t_{j + c} < N$  
can simply derive the keys in the interval $[t_{j + c}, t_i]$, even if 
the user should not have access to them.
To solve this problem, \textit{blocks} can be added on top of DKR.
Blocks aim to restrict access to the keys in either forward, backwards
or both directions by rotating the chains:
\begin{itemize}
    \item A forward block starts a new forward chain, thus preventing an adversary which learnt an element from a prior forward chain from deriving subsequent elements in the collection of the forward chain. 
    \item A backward block starts a new backward chain, thus preventing an adversary which learnt a backward chain element proceeded by such block from deriving previous elements in the backward chain.
    \item A double or full block starts both a new forward and backward chain, thus applying both restrictions.
    \item Finally, an empty block can be used to avoid any chain rotation if the chains are not completely released.
\end{itemize} 
In the example above, starting a new forward chain, i.e. adding a forward block,
at epoch $t_i$ would prevent the user that was removed and re-added from deriving the keys in between 
the intervals. Observe that the space complexity to hold the cryptographic
state has a lower bound defined by the maximum length of a backward chain,
but in this case, it grows in the number of blocks. In asymptotic
complexity, the space used is still linear in the number of epochs.\footnote{As a user can add only one block at each epoch.}
Practically speaking, however, inserting a block at each derivation
should be avoided in space-constrained settings.


We will now recall the full generalised DKR scheme from~\cite{GKP}, and we will call it simply DKR from now on. 
A DKR scheme global state is composed of:
\begin{itemize}
    \item A max epoch $epoch_{max}$, which is the current epoch to which the DKR has progressed.
    \item A set of forward (backward) chain elements is needed to derive the epoch keys from 0 to $epoch_{max}$, each of them paired with the epoch they belong to.
    \item A maximum size $N$ for the backward and forward chains before a chain rotation is required.
\end{itemize}

We give an informal description of the operations that are provided in DKR:
\begin{itemize}
    \item \texttt{Init}: initialise the state of the DKR.
    \item \texttt{Progress}: on input the global state st and a block, progress to the next epoch ($epoch_{max} + 1$) creating new chains accordingly to the block and maximum size $N$. Returns the modified state.
    \item \texttt{GetInt}: on input the global state and an epoch interval $[l, r]$ returns an ``interval state'' which gives access to a sub-interval of the global epoch key interval $[0, epoch_{max}]$. We point out that an interval state is a way to export a partial DKR state to other users.
    \item \texttt{CreateExt}: on input the global state st and an epoch interval $[r + 1, b]$, returns
    an extension for these epochs, which can be used to extend an existing interval state $[l, r]$, allowing for a compact representation of the state exported as interval state $[l, b]$.
    \item \texttt{ProcExt}: on input a compatible interval state and extension, returns the
    extended interval state.
    \item \texttt{GetKey}: on input an interval state and an epoch, returns the epoch key corresponding to the epoch or error if the epoch is not in the interval.
\end{itemize}

The \cref{fig:dkr-new-chains} provides a visual representation of the 
state after a \texttt{Progress} operation is executed with different blocks.

\begin{figure}[t]
	\centering
	\includegraphics[width=\textwidth]{figures/dkr_new_chains.pdf}
	\caption{
        From~\cite{GKP}.
		Example of how a progress flag at epoch 4 affects the keys derivable from interval states $(f{0}, b_{5})$ and $(f_{6}, b_{0})$.
		From left-to-right, top-to-bottom: none, new forward chain (forward block), new backward chain (backward block), both new chains~(full block). 
        \label{fig:dkr-new-chains}}
\end{figure}

An instantiation of the DKR scheme, called D[S, F] is proposed in~\cite{GKP},
using SSKG and double-PRF as building blocks. The implementation
details are provided in \cref{sc:DKR-implementation}.
The DKR scheme is used to achieve IAC in GKP (\cref{sc:gkp-scheme}).


\subsection{Group Key Progression Scheme}\label{sc:gkp-scheme}

To implement our SSF construction, we use the group key 
progression (GKP) primitive from~\cite{GKP} and its
instantiation GRaPPA.
This primitive and its instantiation builds on top of the previously presented DKR
(\cref{sc:background-generalised-DKR}) and dual-CGKA (\cref{sc:CGKA}) primitives
and takes inspiration from them.
As in CGKA, GKP is used among an asynchronous dynamic group of users
to agree on a sequence of shared group secrets. Those secrets
are derived from a shared state, which is maintained by a
subgroup of the users, called admins. Each key is associated with
an epoch, exactly as in DKR. GKP epochs model
discrete time, in which the group state is changed, through
additions, deletions of users or key rotations as in CGKA.
Only users are allowed to add or remove users from the group,
and their cryptographic state might be bigger than the one of
non-admin members.

We informally recall the syntax and operations of GKP from~\cite{GKP}.
As in CGKA and MLS (\cref{sc:MLS}), GKP implicitly rely on a delivery service
to distribute messages among the group members with ordering guarantees.
These messages, as in CGKA, are needed to perform the protocol operations.
The operations of GKP are:
\begin{itemize}
    \item \texttt{InitUser}: initialise the state of a new user.
    \item \texttt{Create}: on input the user state, creates a group owned by the calling user and outputs an updated user state. The creator is an admin of the group.
    \item \texttt{ExecCtrl}: on input the user state, optional arguments, and a command type execute the corresponding action, eventually modifying the state. The list of commands for an admin is:
    \begin{itemize}
        \item \texttt{Add}: add a user to the group with access from the current epoch.
        \item \texttt{Rem}: remove a non-admin user from the group.
        \item \texttt{RotKeys}: rotate the key material of the entire group.
        \item \texttt{AddAdm}: add an existing member to the admin group and share the complete admin state.
        \item \texttt{RemAdm}: remove an admin user from the admin group.
        \item \texttt{UpdAdm}: refresh the state of the calling admin. 
    \end{itemize}
    The commands for a non-admin user are:
    \begin{itemize}
        \item \texttt{UpdUser}: refresh the state of the calling user.
    \end{itemize}
    Each of the command above produces a control message and/or a welcome message, to be sent to the other members of the group through the delivery service.
    \item \texttt{ProcCtrl}: on input the user state and a control message, process the control message, evolve the epoch accordingly and return the updated user status.
    \item \texttt{JoinCtrl}: on input the user state and a welcome message, process the welcome message to join the group. Return the updated user state.
    \item \texttt{GetEpochKey}: on input the user state and an epoch, derive the corresponding epoch key and return it.
\end{itemize}

In a correct GKP, all parties, i.e. members and joining users, which process a correct
sequence of welcome and control messages, should be able to derive the same keys
for each epoch they are given access to. The mathematical syntax assumes
no global state is stored outside the clients, however, this assumption is challenging
in practice as we detail in \cref{ch:ssf},
especially refer to \cref{ssc:GKP-client-middleware} and \cref{ssc:delivery-service}.

The construction GRaPPA instantiates GKP using dual-CGKA~\cite{USENIX:BalColVau23}
and DKR \cite{GKP} as building blocks to achieve IAC (\cref{sc:iac}) for the shared group epoch keys.

IAC can be enforced by properly adding blocks in DKR at membership
changes, to prevent new members from deriving past secrets (as in the previous detailed example)
as well as old members to derive future secrets. See \cref{fig:gkp-iac}.

\begin{figure}[t]
	\centering
	\includegraphics[width=\textwidth]{figures/gkp-meta-figure.pdf}
	\caption{
        From~\cite{GKP}.
        Group key progression (GKP) example built on dual key regression (DKR) to achieve interval access control
        (IAC) for shared persistent data.
        \label{fig:gkp-iac}}
\end{figure}

The implementation details are provided in \cref{sc:GRaPPA-implementation}.
We provide an overview of the state of GRaPPA users, which we derive from the protocol description and pseudocode in~\cite{GKP}. 
Note that the cryptographic state of an admin contains the cryptographic state of a member as described below.
Members maintain the following cryptographic state:
\begin{itemize}
    \item A user identity.
    \item The state of a member CGKA group (from dual-CGKA), $CGKA_M$.
    \item The interval state $int$ to which they have access to (exported from D[S, F] construction by admins, see \cref{sc:DKR}).
\end{itemize}
Admins additionally store and have access to:
\begin{itemize}
    \item The state of an admin CGKA group (from dual-CGKA), $CGKA_A$.
    \item The DKR state instead of just the interval state, where the current epoch $epoch_{max}$ is the current epoch of the group in GRaPPA.
\end{itemize}


We note that we simplify our description here by not explicitly mentioning the current epoch as part of the state of a user:
for admins, this is implicitly part of the DKR state, while for members it is implicitly part of the interval state.
In contrast, each CGKA group state includes a current epoch of the group,
which diverges from the GRaPPA epoch, as it is determined by changes in
the CGKA group state composition and by user secret refreshes internally in the
CGKA group itself only. State management is a critical aspect 
of the implementation, and will be discussed in detail in \cref{sc:state-sync-rollbacks}.


\chapter{System Design}\label{ch:setup}

In this chapter we discuss the overall system design of the project.
First, we provide an architectural overview
of the system and the various components. \cref{sc:architectural-overview}.
We then discuss the requirements of the system from the user point of view
(\cref{sc:real-user}), from the developer point of view
(\cref{sc:developer}) and from the cryptographic point of view
(\cref{sc:abstract-to-real}). We highlight the importance
of taking into account all point of views and stress out the 
relevance of considering the real-world
user for the choice of technologies in the implementation. 

After laying out the requirements, we discuss the various
components of the system, detailing for each of them
the technologies we use. In-depth technical
discussion and technology surveys are provided
to motivate our final choice when required, and we believe
that this is of great interest for the practicioners
that want to implement cryptographic systems with 
a similar technology stack.\footnote{The term ``technology stack'' (or tech stack) generally refers to the set of technologies used together to build a software product.}
However, the reader should feel free to safely skip
them, as they are not strictly necessary to understand 
the rest of this work.

We only discuss common
pieces to both the baseline and the SSF implementation
in this chapter. Most of the system is
shared between the two implementations, to allow
for easy comparison and benchmarking.
The specifics of each implementation are instead discussed in \cref{ch:baseline,ch:ssf}.

\section{Architecture Overview}\label{sc:architectural-overview}

In figure \cref{fig:architecture} we provide an high level overview of the system architecture.
We use different colours to highlight different components of the system, grouping
by frontend, backend, relational databases and cloud components.

\paragraph{Client} The SSF client is a web application that
runs in the browser. The client is used by the user to
interact with the system, and performs the cryptographic
operations needed to access, store and retrieve files 
from the shared folders the user has access to.
Although in the \cref{sc:SSF-scheme} we only model one folder,
in practice the system should support multiple ones.
We imagine the final version of the user interface (UI) to
be similar to the one of Google Drive or Dropbox, where
the user can see the shared folders and navigate inside each
of them to see list of files stored in each of them.
The UI has been kept out of scope from this project, due to
time constraints (\cref{sc:client-overview}).

\paragraph{PKI} The system needs a PKI to manage the identities
of its users. We assume a corporate environment, where
an internal identity provider is available. The PKI server
is a simple certificate authority (CA) which is trusted
by the clients, and issues certificates to users of the system.
The CA certificate is generated once and embedded in the client
code to allow for the verification of other users certificates.
All the certificates are stored in a relational MySQL database \texttt{pki}.

\paragraph{SSF Gateway and DS}
The SSF Gateway server is the main backend component of the system.
The same server will be used also to implement a delivery service (DS), see \cref{sc:MLS} and \cref{ssc:delivery-service}.
These are two different components in abstract, but in practical terms, 
we will see that they share some state (\cref{ssc:delivery-service}).
The SSF Gateway, or simply Gateway, is responsible for the
creation of the shared folders (\cref{sc:SSF-scheme}) and the management of user access
to them. The shared folders are stored in a relational MySQL
database \texttt{ds}, together with the users access control lists (ACLs).
Through the Gateway, files can be uploaded and downloaded 
from the cloud storage provider, which is hidden from the client,
as can be seen in \cref{fig:architecture}.
We can imagine, that the Gateway would constitute the main
server of a company offering a service for secure file sharing.

\begin{figure}
    \centering
    \includegraphics[width=0.8\textwidth]{figures/architecture.png}
    \caption{System Architecture Overview: The main components of the system are shown with the dependencies between them.}
    \label{fig:architecture}
\end{figure}

\section{High Level Practical Requirements}\label{sc:requirements}

After the initial very high level overview of the system, we
now move on to discuss its requirements instead and continue with a more
detailed explanation of the components of the system in the following sections.

\subsection{Starting with the User in Mind}\label{sc:real-user}

A goal of this thesis (\cref{sc:focus-of-this-thesis})
is to create an MVP. To this end, we start by asking ourselves,
as users of the system, what do we want from it?
\footnote{The study of the use cases and situations from a product 
point of view in industry is normally done through so-called ``user storties''.
User stories as sentences that try to summarise a workflow from the
point of view of a user. They have the following structure:
``As [a user persona] I want [to perform this action] so that [I can accomplish this goal]''.}
Our exploration is limited in time therefore we want to simplify
our requirements to the minimum necessary to run the system
in a real-world setting, 
but still with a customer-centric approach
and leave the possibility to iterate over the initial implementation~\cite{ries2011startup}.\footnote{To be customer-centric is a famous Amazon core approach. Indeed, Amazon has a leadership principle called ``customer obsession''~\cite{AmazonLeadershipPrinciples}.}
We stress out that we do not want to develop just a proof-of-concept of the protocol.

As users, we expect to be able to access 
cloud systems from any device that can navigate online.
This is indeed a core minimum requirement for any modern
cloud storage solution that could make it to the market.
We also think this is a challenge that is worth exploring
in the context of SSF to understand if the underlying
schemes (\cref{ssc:GKP-client-middleware} \cref{sc:background-generalised-DKR})
are actually feasible and applicable for real use cases.
Normally, such ubiquitous access to cloud storage solutions,
is through a web interface.
This in turns means that we need to run our code in the
browser of the user (at least a part of it).

Other user expectations include updates and changes to uploaded files.
Further, we expect our system to provide multi tenancy,
meaning that we handle multiple groups of users at the same time.\footnote{Note that each shared folder is in one to one relationship with the group of users that have access to it.} 
Another side of the multi tenancy aspect, is that a 
single user could be part of multiple such groups at once. 
However, we point out that the SSF scheme (\cref{sc:SSF-scheme}) 
itself is designed for a single shared folder at the time.
Indeed, the underlying primitive GKP (\cref{sc:gkp-scheme}) focus 
on one group at the time in an isolated execution environment.

We highlight that all major file sharing solutions available today meet the above requirements.
Therefore, we claim that these requirements and expectations are expected to be met by any serious attempt to create a new cloud storage solution, including the implementation of the SSF scheme.

\subsection{Cryptography: from Math to Real Execution Environments}\label{sc:abstract-to-real}

Before writing any code, we need to choose the right 
programming language that targets the execution environment
we want to run the code in.

As seen in \cref{ch:background}, the SSF scheme (\cref{sc:SSF-scheme})
uses some advanced cryptographic primitives:
\begin{itemize}
    \item continuous group key agreement (CGKA) (\cref{sc:CGKA})
    \item seekable sequential random generators (SSKG) (\cref{sc:SSKG})
    \item dual-key regression (DKR) (\cref{sc:background-generalised-DKR})
    \item group key progression (GKP) (\cref{sc:gkp-scheme})
\end{itemize}
Implementing these primitives from scratch could in itself
be a complex, time-consuming and error-prone task, especially for CGKA.
It is therefore important to use libraries when available
to deliver reliable implementations.

In our execution environment 
we need support for all cryptographic operations
needed both by the primitives above and by simpler
cryptography which is normally exploited and assumed to exist. 
To be more precise we would need:
\begin{itemize}
    \item Secure random number generation.
    \item Availability and constant time execution of the cryptographic operations underlying the constructions we implement, both for the baseline and the SSF scheme.
    \item Memory safety.
\end{itemize}
The above requirements are needed to avoid security
issues in the implementation. For example, a non-constant
time execution of the cryptographic operations could
lead to timing attacks. 
However, during our implementation work, 
we found that in major runtime
environments, such as the browser, 
support for the cryptographic operations normally
used to instantiate cryptographic primitives is not always available.
This might lead to the usage of custom implementation,
where such guarantees are impossible to provide because
of the lack of low level control on the execution environment.
We detail such findings throughout the rest of the thesis,
and summarise them among other engineering issues in \cref{ch:gaps}.

\subsection{What does the Developer need?}\label{sc:developer}

As developers, we face the challenge of implementing a full
system in a very short time frame and with limited resources.\footnote{To be more precise, the whole implementation has been conducted by only one person in less than six months.}
We want to reduce the possibility of errors and bugs
as well as the complexity in maintaining the integration between
different components.
Sometimes, just a small change in the protocol can lead to
hundreds of lines of changes in the code.
Even more, we want to be able to easily prototype, i.e., 
test out different ideas coming from the theoretical side
or different engineering solutions.
However, changing requirements during the implementation
and updating the system accordingly can take up months of work
if the setup is not properly done to allow development agility.
Further, we would like to easily benchmark the implementations.



\section{The PKI}\label{sc:PKI}

This section presents the PKI server implementation from the architecture overview (\cref{sc:architectural-overview}).

Recalling from the description given in \cref{sc:mental-model},
we need a Public Key Infrastructure (PKI) to implement the SSF scheme.
Papers in cryptography normally assume the existence of a PKI,
which is solving the problem of assigning identities.
Since we want to be able to easily develop the code, and the PKI itself
is not the core problem we are focusing on, we want to rely on existing
solutions.
A common standard way to manage identities are X509 certificates~\cite{rfc5280}.
To implement this solution, we need to have a Certificate Authority (CA)
to distribute the certificates.
Our MVP imaginary target use-case is company or organization. 
Thus, we can assume to have an internal identity provider.
Building on top of X509 certificates, we use transport
layer security (TLS) and mutual TLS (mTLS) to 
secure communication between the clients and the server(s).
While TLS only verifies the server certificate,
in mTLS the client also provides a certificate to the server
for authentication.
The PKI server is using TLS, while the SSF Gateway (\cref{sc:ssf-proxy-server}) is using mTLS.
This way, we greatly simplify development. Indeed, we 
avoid introducing any other Authentication
protocol such as OAuth2 or usage of JWT tokens.
We highlight that this PKI server implementation is not meant for production use.


\paragraph{Implementation Details} 
The PKI server is implemented in Rust, using \texttt{Rocket}
framework for the server and \texttt{sqlx} driver to interact
with the MySQL database. 
For local development, 
the database runs in a Docker container, which allows
for easy setup and teardown.\footnote{Docker is a virtualization technology} 
It is important to have the ability
to easily reset the database to a fresh state 
while developing and testing the system.
The commands to start and stop the container are provided
in the top level \texttt{README.md} of the project.

The code is available in the \texttt{pki} crate
of this project, in \texttt{services/pki} folder.
The server exposes a simple API
to issue certificates to users. 
A client can send a certificate signing request (CSR)
to the server and receive a signed certificate in return.
The code to construct the CSR and issuing certificates, as well as 
verify them is available in the Rust crate \texttt{common},
which is shared among servers and clients (\cref{sc:client-overview}).
We provide unit tests for the library and expose simple bindings (\cref{sc:browser-runtimes}) to be
called from JavaScript in the client (\cref{sc:client-overview}), check the \texttt{README.md}
of the crate for details. All the exposed functionalities are stateless.
When issuing a certificate, validation is not performed
on the user identity, as this is out of scope for the MVP.

The certificates are stored in MySQL ``pki'' database, and the CA
certificate is copied and embedded in the client code to allow
for local verification of other users' certificates.
The server exposes an endpoint to fetch the
CA certificate, which can be used by clients to
refresh the embedded CA certificate.
Further, an endpoint to fetch users' certificates providing the
email of the target user is available.
The description of how to compile and start the server is
provided in the \texttt{README.md} file of the \texttt{pki} crate.
The server is exposed locally at \texttt{\url{https://localhost:8000}}



\paragraph{OpenAPI Specification}
The server API is available in OpenAPI
specification format~\cite{OpenAPISurvey} 
which is generated automatically from the code using
\texttt{utoipa}. Each server endpoint code is
annotated through the library annotations, with the description
of the endpoint, parameters etc. \texttt{utoipa} is then
able to generate the OpenAPI specification in YAML format.
We store the generated file in \texttt{openapi/pki-openapi.yaml}.
We provide an executable written in Rust to perform the generation
through \texttt{utoipa} and regenerate the \texttt{.yaml}
file when the server endpoints are modified.
The OpenAPI specification is then used to generate the
code in TypeScript, which is used in the client to interact
with the PKI server (\cref{sc:client-overview}), thus spearing
the developer from writing thousands of lines of code and keeping them up to date, 
as well as remove reducing the possibility of errors in the client-server interaction. 
The guide on how to use the executable can be found in the \texttt{README.md}
of the crate.
While the server is running, the OpenAPI specification can be
accessed nicely through the Swagger UI from the browser, 
by visiting the page \texttt{\url{http://localhost:8000/swagger/UI}}.

\section{Cloud Storage: from an Abstract Model to Real Systems}\label{sc:cloud-storage}

This section provides the necessary background on cloud storage,
aiming to provide the reader with a first understanding of the challenges
we encounter when dealing with real cloud storage system
as compare to the abstract model we can find in the GKP scheme (\cref{sc:gkp-scheme})~\cite{GKP}.
We start with motivating some of the assumptions that are made
GKP scheme around the cloud storage (\cref{scc:cloud-storage-assumptions}), and then we discuss why
other assumptions are too simplistic and need to be reconsidered.
The discussion of the gaps will also be expanded in more details in the next sections,
where we will use the cloud storage provider to store the shared folders' contents.
The discussion is particularly relevant for the implementation of the
SSF Gateway server (\cref{sc:ssf-proxy-server}).

\subsection{Assumptions}\label{scc:cloud-storage-assumptions}
In the cryptographic research, cloud storage systems are abstracted away by assuming
that some operations are available to write and read data to a virtually
infinite storage system always available.
This is a well-founded assumption, as a cloud storage provider
will normally provision new hardware as needed (in advance) to accommodate the
increased capacity request~\cite{AzureBlobStorage}.

Deletion of files is not assumed to be secure, meaning that it is not guaranteed to happen faithfully and completely,
even if the storage provider is not malicious.
We motivate this assumption with the following facts:
\begin{itemize}
    \item Cloud providers state in their Service Level Agreement (SLA) that
    with very high probability the service is up and running for more than a certain percentage of time or a refund is paid out to the clients.\footnote{e.g.\ Amazon Simple Storage Service (Amazon S3) starts to pay a refund to clients when the monthly uptime percentage is less than 99.9\%}
    The SLA also assures the durability of content uploaded is at least a certain percentage.\footnote{As an example again, Amazon S3 is designed for 99.999999999\% durability}
    \item To meet the above SLA, cloud providers replicate the content. The content is replicated at least (normally) 3 times, and possibly in different geographical zones, to protect against widespread failures~\cite{AzureBlobStorage}.
    \item The content is automatically monitored and re-replicated in case some storage devices are failing.
    \item It is well known that in most case deleting a file from a disk doesn't delete the content but only marks that disk space as free without zeroing out the bits.
    \item The replicated storage could be eventual-consistent, 
    meaning that in case of a network partition the two 
    parts of the system continue to work independently. 
    In this setting, if while the network is partitioned 
    a side of the storage receives a deletion operation 
    for a certain file, upon restoration,
    the nodes still owning the data would try to replicate 
    them to the other side.
    Therefore, instead of deleting the information, 
    the distributed data store creates a (usually temporary) 
    tombstone record to keep track and eventually perform 
    the deletion on all other nodes as well upon reconciliation.
\end{itemize}

While we show that these assumptions are well-founded,
we highlight the following issues when working with
real cloud storage systems, which will be discussed in the
relative sections:
\begin{itemize}
    \item Cloud storage technologies are of many types.
    Different providers can differ in the way they
    offer similar services.
    When setting up the project we need to choose the right
    technology to use. We also need a way to
    abstract away the specific vendor for the storage
    to assure that our solution is easily portable.
    We introduce and discuss the technology of choice in (\cref{ssc:object-storage})
    \item The cloud storage system is not accessible freely
    by clients. Access to cloud resources is
    granted by cloud providers upon registration and payment
    for the service. A system implementing the SSF scheme
    needs to deal with this aspect, and its implications
    on the system design and in terms of security (\cref{sc:cloud-storage-access-and-billing}).
    \item Abstracting the cloud storage system as read/write
    operations on a virtually infinite storage system does
    not capture the complexity of concurrency and consistency
    issues that arise in real-world systems. Although this
    might not be relevant for security in the SSF scheme,
    it is relevant for the correctness of the system and
    security properties normally expected by users (\cref{sc:ssf-file-changes-sync}).
\end{itemize}

\subsection{Object Storage}\label{ssc:object-storage}
As we need to store the shared folders' contents in the cloud,
we check the available storage solutions.
Cloud providers offer different storage solutions,
each with different characteristics. The main file
storage solutions on the market can be divided into
file systems, object storage and databases.

Databases are not needed in this case,
as we do not want to analyse the data stored in
the cloud. Files are encrypted by the client
before the upload.

File systems offering could be used to implement
our shared folders. However, the SSF scheme does not
assume any hierarchy inside a shared folder.
Also, file systems API are rather more complex
than the abstraction that is considered by the
theoretical construction. File storage is also offered
at more expensive prices than the alternative object storage.

Object storage solutions provide
very simple API that allows to store and retrieve
objects in a flat namespace.
The name object storage derives from the fact that this type
of storage does not handle files and file hierarchies 
as we would expect in a file system. Instead,
files are grouped together as plain list of objects,
inside a top level namespace, called a bucket.\footnote{The name is borrowed from Amazon S3, the first such system.}
This type of storage has all the characteristics that are assumed
for the GKP scheme (\cref{sc:gkp-scheme}) as well as for the SSF scheme (\cref{sc:ssf-scheme}).
Stored files are called objects and have the following characteristics:
\begin{itemize}
    \item An object can be big (e.g.\ in Amazon S3 up to 5 GB in a single chunk).
    \item Objects generally can only be written all at once and update or delete them creates new versions of the entire data. A special case are append-only objects, which can be extended with new data without rewriting the whole object each time, but still disallow in-place modifications of the existing content.
\end{itemize} 

We notice that we do not require the ability to update the content of the files in-place,
as the file is always sent encrypted by the client. Therefore, the server has no visibility 
on the content.
Object storage solutions are normally cheaper than the alternatives
and are designed to be highly available and durable.
The cost is really important when considering the implementation
of a real-world system.

Examples of object storage solutions are Amazon Web Services (AWS)
Simple Storage Service (Amazon S3), Microsoft Azure Blob Storage, Google Cloud Storage (GCS).

In our implementation we will use Amazon S3, as it is the most
popular object storage solution, and it is widely used in the industry.
During development, to avoid incurring costs, we will use LocalStack~\cite{LocalStack},
to emulate Amazon Web Services locally on our machine. The availability
of such a development tool was also determining in the choice
of the cloud storage provider. LocalStack
runs as a Docker container.
To avoid vendor lock-in, we will abstract the cloud storage provider
and allow for multiple storage solutions to be used.\footnote{Vendor lock-in indicates the condition where a customer is dependent on a vendor for products and services, and cannot move to another vendor without substantial costs.}
This is discussed in \cref{sc:ssf-proxy-server}.


\subsection{How do users access shared data?}\label{sc:cloud-storage-access-and-billing}
Cloud storage providers are normally accessed through an
identity system. The clients register to the cloud provider.
Through their accounts they can store and load data.
Normally only the account owning the data itself can access it.
However, some storage solutions allow for world-wide public access 
to the data store owned by one account.\footnote{This is the case of Amazon S3 and all other aforementioned object storage solution.} 
This is not recommended for sensitive data.
A recent empirical study from K. Izhikevich et al.~\cite{izhikevich2023using},
shows that there is a strong interest
in compromising publicly accessible data storages,
especially when relatable to commercial entities.
Indeed they found a correlation between storages named after companies and likelihood of being discovered and attacked.
Hundreds of IP addresses attempt to download, delete, or upload objects including malicious shell scripts. 
Furthermore, actors scan public data stores within 40 minutes of deployment and upload unsolicited content within 10 days.

In case instead where each client have its own account
and we keep the data private,
some practical questions arise:
\begin{itemize}
    \item How to share data between clients?
    \item How to manage the access control?
    \item How to manage the billing?
\end{itemize}
Further, our system should be able to integrate with multiple
cloud storage providers. We do not want to force the users to
create accounts on a specific provider, rather this should be
transparent to them.
\nd{I am not sure if I should talk about the following in a later section, where I present the tech stack, or maybe add an ``overwiew of the architecture'' section before the tech stack? }
To this end, we abstract the cloud storage provider itself
and create a server component, called Gateway, which will manage
the access to the cloud storage provider, acting as a proxy
between clients and the cloud storage provider(s).
The Gateway do not only solve the practical issues above, but also
allows for user-expected (non cryptographically enforced) 
security properties of the system to hold.
For example, by performing access control, the Gateway can ensure
that only members of the group can access the data in their
specific shared folder. Although data is E2EE, it is still
good practice to disallow non members from freely download the data,
considering also that the read operations on the storage are also billed.
Further, by disallowing arbitrary writes, the Gateway
protects the data from being overwritten by a malicious external user,
thus protecting from DDOS attacks.
Although availability is out of scope in the SSF scheme,
it is practical required in any real-world system.


\section{Allow the User to Interact with the System: the Client(s)}\label{sc:client-overview}

This section discusses the client side of the system,
called SSF client in the high level architecture (\cref{sc:architectural-overview}).

\paragraph{The Scope of the MVP Client}
As seen in \cref{sc:real-user}, 
we want our client to ultimately be a web application,
running in a browser.
However, for a first development iteration, we want to
remove the development burden of building a UI, which is
a time-consuming task. We instead want to focus on the
implementation of the cryptographic constructions (\cref{ch:background}). 
Still, we want to implement them is such a way so that we
will be able to just run the same code inside the browser,
which is the final target execution environment.
The browser client setup is showcased
in the \texttt{www} folder. Interested practicioners
should check the respective \texttt{README.md} file
for the instructions on how to compile and run the code.

\paragraph{Development Agility} 
To test out the implementation (\cref{sc:developer}), 
and provide the users of our system
with a first MVP version, we develop a
command line interface (CLI) client.
Internally, the CLI will use the cryptographic constructions'
implementations. This part of the code will be tested to be executable
inside the browser to ensure portability and allow a subsequent fast
migration to the final UI. The code for the CLI
can be found in the \texttt{ssf-client} folder.

\paragraph{Code Portability}
Since we want E2EE guarantees (\cref{sc:SSF-scheme}), the protocol execution needs to happen
in the client device of the user.
This implies that we need to investigate the various options
in terms of runtime support in the browser (and in the CLI runtime) 
to check if we can also execute all required cryptographic 
operations (\cref{sc:abstract-to-real}).
Our code portability requirements brought us to interesting
findings on runtime compatibilities,
detailed in \cref{sc:Web-Crypto-API-implementations:-non-standard-behaviours}
and \cref{sc:MLS-enhancements}.
These subtle differences in the runtime environments
make the implementation of portable and secure 
cryptographic software a challenging, or impossible, 
task.

\paragraph{Execution Runtimes} 
Refer to \cref{sc:browser-runtimes} for an in-depth survey about browser runtimes.
We do not only explore the browser setting, but also explore
portability of the code from and outside the browser.
Libraries written in other languages could be ported and used in the implementation.
In short, two runtime environments are available in the browser:
JavaScript (JS), mostly used in web development, and WebAssembly (Wasm),
a newer execution platform, normally the right choice to achieve better performance
and to port code from other languages to the web. Wasm is normally
used in combination with JS, as it cannot interact directly with
the web page as well as with the browser APIs directly. Furthermore, 
as mentioned above, we also want to use our code to first build a CLI.
Node.js is a natural choice to run JS and Wasm outside the browser,
for our portability needs, as it is based on the same 
JS execution engine as Chrome, V8.
Instead of developing directly in JS, we implement in TypeScript (TS),
a statically typed superset of JS, to avoid common JS pitfalls.
For a full, in-depth explanation see \cref{sc:browser-runtimes}.

\paragraph{Cryptographic Support and Libraries}
In \cref{sc:webcrypto-api} we describe the cryptographic API
available in the browser and its limitations. Since we need to use
such API to perform the cryptographic operations, we will use JS
as the main runtime environment.
We also notice that the same API is also implemented and supported in Node.js.
We survey the available 
implementation to date of CGKA 
(\cref{sc:CGKA}) and MLS (\cref{sc:MLS})
to the best of our knowledge (\cref{sc:CGKA-implementations}).
To be able to use the best available
and browser compatible implementation of this cryptographic
primitive we need to integrate Rust code compiled to Wasm, which is
called from JS in our client code. We remind the reader that
Rust is a memory-safe language.
Thanks to its memory safety guarantees and performance,
as well as its expressive type system which prevents many errors already at compile time,
it is a good choice to write cryptographic code.
The library of choice is mls-rs~\cite{AWSMLSGroup}, developed by AWS Lab, 
and open source on GitHub.\footnote{AWS Lab is a research branch of Amazon Web Services (AWS). GitHub is a famous git-based source code sharing platform. The term ``open source'' refer to the fact that the code is publicly available and, as we will see, anyone can read and modify it. Normally, the code is distributed under a licence, so usage might be restricted.}
The library is licenced under MIT and Apache version 2.0, both
permissive licences that allow for unrestricted commercial use of the code.

\paragraph{Certificate Validation}
As we are already using the Rust to Wasm toolchain, 
we will also use some Rust code shared
with our servers to check certificate validity in the client.
This code is provided in \texttt{src/common} as a Rust library
with both native and Wasm compilation targets (\cref{sc:PKI}).
This reduces the amount of code duplication and interoperability
issues between the client and the server. Interested practicioners
should check the \texttt{README.md} file for the
instructions on how to compile the code and the configuration
in \texttt{Cargo.toml} to allow both native and Wasm compilation
targets. We use conditional
compilation to configure how dependencies are imported
given a compilation target. To allow for local verification
of other users certificates,
we embed the CA certificate in the client code.
For testing purposes, we generate also the servers' certificates
from our testing CA.
We also install the servers' certificates in the client code,
to allow for the verification of the servers' identity
in the Node.js CLI, by including them in our HTTPS calls.
See the \texttt{ssf-client/src/protocol/authentication.ts} file.
To be able to call the servers from the browser client,
we need to install the servers' certificates in the OS
certificate store, so that the browser, while connecting 
using the HTTPS protocol, can verify the servers' identity.

\paragraph{Code Generation}
To interact with the servers, we need to write code to
call all the endpoints they expose.
Instead of writing the code manually, we use the OpenAPI
specification generated from the server annotated code (\cref{sc:PKI})
to generate the TypeScript code to call the endpoints.
We have tried multiple generators, and we
decided to use \texttt{@hey-api/openapi-ts}~\cite{OpenAPITs}.
This generator allows generating clients backed by
different HTTP libraries, we rely on \texttt{Axios}~\cite{OpenAPIAxios}, 
to abstract the HTTP calls and easily port the code
between browser and Node.js clients.
Another motivation for using this generator is that it
correctly generates the TS types for the request parameters
and the responses, thus speeding up the development
and enable fast iteration on the client code when a server
endpoint is modified.
We use the generator to create clients both for the PKI server and the SSF Gateway server.
The generated code is created under the subfolder \texttt{gen/clients}.


In the reminder of the chapter we will discuss various topics
related to the client implementation, more in detail.
As already pointed out before, \cref{sc:browser-runtimes}, \cref{sc:webcrypto-api} \cref{sc:CGKA-implementations}
are meant to provide the reader with detailed information 
on the technologies used in the client implementation.
The \cref{sc:CLI} will provide a detailed description of the CLI client,
the code architecture, the abstraction layers, unit and
integration tests setup, the list of commands and an idea of how
to use them.

\subsection{A Deep Dive in Browser Runtimes}\label{sc:browser-runtimes}

JavaScript (JS) is the primarily runtime available in modern browsers.
JS is a managed language, meaning that the programmer
does not have low level control on the allocation and
deallocation of memory. Rather, the language runtime support does it.
In more detail, the heap-allocated memory
is tracked by a garbage collector (GC), which regularly
checks for unreachable objects and frees the memory allocated
to them. We recall that this doesn't solve memory leaks
problems. Indeed, some memory is constantly leaked
between the activations of the GC. Also, the GC
pauses the execution of the program to perform its work,
and can therefore cause timing issues. However, garbage-collected
languages are normally regarded as memory safe, since
the programmer does not need to deal with memory pointers
and their management.


JS is a dynamically
typed language: the types of variables are not checked
statically. This can lead to bugs that are not caught
until the execution of the program.
To mitigate this issue, TypeScript~\cite{bierman2014understanding} (TS)
commonly replaces JavaScript in new or large codebases.
TS is transpiled to JS, it's a superset of JS
\footnote{Any JS program is also a TS program.}
and is statically typed.
Also, other programming languages such as
Kotlin~\cite{KotlinToJs} can be transpiled to JS.

The three major execution engines for JS are V8~\cite{V8} (Chrome/Chromium),
SpiderMonkey~\cite{SpiderMonkey} (Firefox) and JavaScriptCore~\cite{JavaScriptCore} (Safari).\footnote{Note that Edge is now based on Chromium and therefore
also uses V8.}
Among them, V8 is the underlying engine used in other
common JS execution environments on the server side, 
namely Node.js~\cite{NodeJS} and Deno~\cite{Deno}.


WebAssembly (Wasm)~\cite{Haas2017,WasmSpecification} is an alternative runtime
inside browsers.
The Wasm virtual machine (VM) is available in all major
browsers and can be used to run code either written directly in Wasm
or compiled from other languages.
\footnote{Wasm is also supported in Node.js on the server side, however the
code loading process is slightly different.} 
The latter is the
most common case, with source languages like C/C++, Rust,
Kotlin, Go, etc. C/C++ and Rust are low level languages
which allow for fine-grained control of the memory
and do not rely on GC. Emscripten~\cite{Zakai2011} is the primary 
tool for compiling C/C++ to Wasm, using Clang compiler,
which is based on the Low Level Virtual Machine
(LLVM) architecture~\cite{LLVM2004}.
Rust\footnote{Rustc compiler internally also uses LLVM.} also supports compilation for whole application to Wasm
through Emscripten, using the \texttt{wasm32-unknown-emscripten}
compilation target.
However, currently the preferred way is to use
the \texttt{wasm32-unknown-unknown} target, which
compiles to Wasm directly without the need of Emscripten
and produces smaller binaries.\footnote{Normally,
Emscripten is used to port existing application instead 
of building libraries. It indeed 
provides a large standard library, 
containing things such as TCP sockets, 
file I/O, multithreading, openGL etc. that are needed in
standalone applications.
Rust is a new programming language compared to
old C and C++ therefore not many applications were yet 
written before Wasm was created. 
Thus, the preferred usage of Rust code compiled to Wasm
is to write libraries with bindings exposing the
compiled Wasm module to JS in the Browser.}
The Rust to Wasm standard tool chain includes 
wasm-pack~\cite{WasmPack} and wasm-bindgen~\cite{WasmBindgen}.
By using these, the compilation creates a Wasm 
module together with the JS bindings, the ``glue'' code
needed for interoperability, exporting the functionalities, 
so that are easily callable from JS. The TS type declarations 
for the JS generated code are also created.
While C/C++ and Rust compile down to native code, Kotlin normally executes on the
Java Virtual Machine (JVM). It is a high level language with GC,
but it can be compiled to Wasm thanks to the WasmGC proposal
and its support in common execution environments~\cite{WasmGCProposal, WasmGCinV8}.
Go similarly is a garbage collected language~\cite{GoGarbageCollector}.
However, the compilation to Wasm~\cite{GOWasm} 
includes a GC in the compiled code itself, because WasmGC support doesn't
provide certain assurances that Go code expects from its own GC.\footnote{Specifically, the CG should not move memory around while cleaning the unreachable objects.}
For sake of completeness, we mention also the AssemblyScript~\cite{AssemblyScript} language, 
which is a subset of TS that statically compiles ahead of time to Wasm.
Being a subset of TS, AssemblyScript opens up possibilities for
interoperability and sharing of code between the two languages.
It got attention and traction in the community as web developers 
can write in a familiar syntax and easily compile optimised Wasm.

\subsection{Cryptography in Browsers}\label{sc:webcrypto-api}

There are several libraries in JS that provide cryptographic
operations.
However, those libraries present a lot of issues:
\begin{itemize}
    \item Ensure constant time execution is hard. JS engines are constantly changing and tuning how they optimise code, leading to a variety of ever-changing expectations for how the code will actually run
    \item In JS all numbers are 64-bit double precision floating point as specified in the IEEE 754 standard. This means that representation of integers is exact only until $2^{53} - 1$.
    \item There is no source of randomness suitable for cryptographic applications.
    \item Implementors are skilled JS programmers but not skilled cryptographers. Developers maintaining such libraries could make mistakes, as well as developers using such libraries may not understand the implications in terms of security. Further, as mentioned in \cref{sc:abstract-to-real}, it is better to re-use basic primitive implementation that are likely more robust. 
\end{itemize}

The Web Crypto API, a World Wide Web Consortium (W3C) standard~\cite{WebCryptoAPISpecification}, 
provides basic cryptographic support in browsers. This API should
always be used when cryptographic operations are needed in the browser
environment.

The specification has been implemented by all major browsers
and is accessible from JS. 
Node.js and Deno for server-side JavaScript also implement the specification~\cite{NodeJsWebCryptoAPI, DenoWebCryptoAPI}.
The goal of the specification
is to provide a common interface to the underlying 
cryptographic primitives. Also, it provides calls to
a secure source of randomness. 
The API is asynchronous, and it is implemented by each
browser providing native support. This means that the
operations are executed in constant time and can
utilize the hardware acceleration and advanced
security mechanism otherwise unavailable to JS, as well as
more precise arithmetic.\footnote{For example, Chromium is internally calling BoringSSL for the cryptographic operations~\cite{ChromiumWebCryptoAPIImplementation}}
However, we note that the specification itself doesn't mandate
constant time execution of the operations.
Overall, the specification should aim to 
remove the need of aforementioned cryptographic JS libraries,
but support is missing for some new standard cryptographic objects. 
For example, for elliptic curves based cryptography, currently
the API supports P-256, P-386 and P-512 curves~\cite{WebCryptoAPICurvesSupport}.
Howerver, the ``secure curves''~\cite{WebCryptoAPISecureCurvesDraft,WebCryptoAPISecureCurvesExplainer}
are not yet part of the standard, although already recommended by
the Crypto Forum Research Group (CFRG) from the Internet Research
Task Force (IRTF) in 2016 and 2017~\cite{RFC7748IRTF, RFC8032IRTF}
and made part of the Federal Information Processing Standard (FIPS) in 2023~\cite{SecureCurvesNIST}.

When compiling cryptographic code to Wasm in a browser environment,
all cryptographic operations should be done through bindings to
the Web Crypto API. Wasm itself doesn't provide any constant time
guarantee. However, some research were made to add such semantic
to the language~\cite{CTWasm, gu2023constanttimewasmtimerealtime}
and a Wasm constant-time proposal exists in the Wasm specification
repository~\cite{WasmCTProposal}.

\subsection{CGKA Implementations}\label{sc:CGKA-implementations}

CGKA is a major component of the SSF scheme (\cref{sc:SSF}).
It's also a core component in MLS as seen in \cref{sc:CGKA}.
Indeed it is normally implemented as part of libraries developing
the full MLS specification.
To the best of our knowledge, the open source available libraries
providing MLS/CGKA are:
\begin{itemize}
    \item OpenMLS, available in multiple languages but not production-ready.
    \item Java BouncyCastle includes a CGKA only library.
    \item AWS Lab's Rust library ``mls-rs'', a full implementation of MLS sponsored by Amazon Web Services (AWS). 
\end{itemize}

Other minor implementation are available, but are mostly broken or outdated.
Thanks to the support for Wasm builds, mls-rs can be used in the browser.
Out of all the options available, mls-rs seemed the most promising:
we note that the library is developed by some authors of the MLS IETF
specification itself and is also integrated in the Android Open Source Project
code base.\footnote{\url{https://android.googlesource.com/platform/external/rust/crates/mls-rs/}}
The library provides crypto agility, meaning that the cryptographic
primitives can be easily swapped out for new ones and support for multiple cipher suites is available.
For the Wasm build, the library uses bindings to the Web Crypto API 
(\cref{sc:webcrypto-api}) which act as a cryptographic operations' provider.
As described in the relevant section, 
this way the library assure a safe execution of the underlying
algorithms and a source of entropy inside the Browser. 
The bindings for the Web APIs are available in Rust 
through the ``web-sys'' crate,
which is procedurally generated from WebIDL language
used in the API specifications~\cite{WebSys}, assuring
that they are always up-to-date and correct.\footnote{Crate is the name for a Rust library.}\footnote{WebIDL is the interface description language used to define the Web APIs, describing the data types, interfaces, properties and methods and all other components that make up the API itself.} 

Furthermore, while the GKP scheme (\cref{sc:gkp-scheme})
assumes only the usage of CGKA without MLS (and thus the security proof
is simplified), we note that in practice MLS itself is needed.
More details are given in \cref{ch:ssf}.

In summary, we use mls-rs library to get an implementation
of CGKA (and MLS), compiling the library to Wasm to use it both
in the browser client and in the Node.js CLI client. The library is
safe to use, as the cryptographic operations are executed on top of the Web Crypto API.


\subsection{CLI}\label{CLI}

The CLI client is a command line interface tool running in Node.js.
It is meant to be used for testing purposes and to provide a
first MVP client for the system.




\section{Get Our Hands Dirty: The Tech Stack}

We give a complete overview of the tech stack we use both
for the baseline and the SSF implementation.
We discuss both the client and the server side of the project.
We recommend interested practicioners to look at the
project repositories for more details on the various
configurations.
For each library or component, we also provide a markdown
\textit{README.md} file containing instructions on how to
setup the required dependencies. We also provide commands
to run the code and test it.
Given that strive we for code portability, our setup
is quite complex and showcase in details
different targets and configurations.
The project repository is organised as a monorepo,
meaning that all the code is contained in a single
git repository. Also, the Rust code is kept
together in a Cargo Workspace for easier management.%
\footnote{Cargo is the package manager of Rust.
For a detailed description of a Workspace: \url{https://doc.rust-lang.org/cargo/reference/workspaces.html}}
Overall, the project repository
could be taken as a starting point for new projects that
intend to use a similar tech stack.

\subsection{Servers}\label{ssc:servers}

Servers are developed in Rust. Rust is a rather new mainstream language
and it is primarily intended to be used for low level
programming. However the unique type system of Rust
allows for a very safe and performant code, which doesn't
require GC but still provides memory safety.
Writing the servers in Rust allow us to gain confidence
with the language itself and avoid lots of bugs thanks to the
aforementioned properties.
We also want to leverage the Rust cryptographic
ecosystems, which is very rich. Indeed, the combination of
safety and low level control makes Rust a very good choice
for cryptographic code. In particular, we want to take
advantage of libraries like:
\begin{itemize}
    \item Ring, a ``safe, fast, small cryptographic library using Rust with BoringSSL's cryptography primitives''~\cite{Ring}.
    \item Rustls ``a modern TLS library'' which also can use internally Ring as a crypto provider. Rustls ``provides no unsafe features or obsolete cryptography by default''~\cite{Rustls}.
    \item webpki a crate ``to validate Web PKI (TLS/SSL) certificates''~\cite{WebpkiCrate}.
\end{itemize}
Also, Ring supports compilation to Wasm, which comes handy
when we want to share some of the cryptographic code
between servers and clients. In more detail, when targeting the
browser, we can enable the feature flag \texttt{wasm32\_unknown\_unknown\_js}.
In this way the necessary calls to the Web APIs
to perform some cryptographic operations that would
otherwise require a specific operating environment,
e.g. secure random number generation (\cref{sc:webcrypto-api}).
Furthermore, these libraries have received an independent security audit 
from Cure53, sponsored by the Cloud Native Computing Foundation (CNCF),
Indeed Linkerd, a project under the umbrella of the CNCF, is using the same libraries~\cite{RustlsAudit}.
The audit found this cryptographic stack to be exceptionally high in quality.

Web development frameworks are however not as well established
as for example Spring in Java. Some of them support quite a wide
set of features though. 
We tried multiple ones, Warp~\cite{Warp}, 
Actix~\cite{Actix} and Rocket~\cite{Rocket}.
We ultimately pick Rocket as it is the most feature rich
and has a very good documentation and easily integrates with
SQL database drivers crates like sqlx and diesel.
We use sqlx to reduce the cognitive load, as it has a very
low level and straight-forward API to write SQL queries.
Further, to reduce the amount of client code and time to update
integration between servers and clients, we make use of OpenAPI~\cite{OpenAPISurvey}.
OpenAPI (OAS) is a specification to describe APIs. It supports
multiple tools to generate stub client code or mocked servers from
the specification itself.
Using the utopia crate, we annotate the routes of our servers
to allow automatic generation of the OpenAPI specification as 
a Yaml file. This way we will be able to generate the TypeScript
clients for the servers (\cref{ssc:clients}). Unfortunately,
the crate didn't support the latest specification (version 3.1), 
which supports mTLS at the time of writing.
\footnote{See this Github issue in the project repository: \url{https://github.com/juhaku/utoipa/issues/531}}
Support for version 3.1 of the specification is now available,
\footnote{This is a breaking change: \url{https://github.com/juhaku/utoipa/pull/981}}
but mTLS is still missing.\nd{can be part of Future Work to advance the library version}


We have two servers, both in the \texttt{/services} subfolder.
Under \texttt{/services/pki} we implement the PKI server, a simple certificate authority (CA).
It has minimal set of features needed to issue certificates
and validate them.\footnote{As implementing a full compliant CA server is out of scope for this thesis, we left some items in the backlog for improving our base testing implementation.}
The certificate are signed with the
self-signed root certificate of the CA.
The connections to the PKI are secured with TLS.
The PKI server also uses a MySQL database to store the certificates that were issued to clients.
Docker, a virtualization technology, is used during testing. 
With Docker, we can easily spin up a clean MySQL instance running 
locally in our machine as a container and tear it down when we no longer
need it. Each time we start a new container, the database is loaded
with the initial configuration. 
Client certificates use emails of the issuing client to bind the certificate to
a user identity.
The email is represented in the certificate as a \textit{subject alt name} as specified in~\cite{rfc5280}.
The client certificates are generated using the
\texttt{rcgen} crate, which is part of aforementioned Rustls. 
Parsing is performed through
the \texttt{X509-parser} crate. Some of the code
to generate the client certificate request and to
verify the identity of the client in the client certificate
is shared with the clients. 
The crate \texttt{/common} contains this shared code.
It can be included as a Rust library or
compiled to Wasm and the npm package can be then
imported. To power the Wasm build we allow
conditional compilation of subdependency ring as
described above.
The relevant configuration can be found in
\texttt{/common/Cargo.toml}.

The Gateway server mentioned in \cref{sc:cloud-storage}
is implemented with the same stack. It is keeping
a MySQL database to store users registered in the system and
the state of each shared folder, i.e. the current set of users
that can access it. 
It also connects to the cloud storage provider. 
We use Amazon S3 as the storage solution (\cref{ssc:object-storage}).
However, we use the \texttt{object-store} crate as an abstraction layer
that allows our server to easily switch to another cloud storage,
like Azure Blob Storage or Google Cloud Storage, by just changing
some configuration. The library also handles concurrent writes
to an object through optimistic concurrency control.
This means that when trying to update an object,
the server also sends to the storage the ETag (version)
of the version the update is made upon.
If the version is different, the cloud storage will reject the update.
However, Amazon S3 does not support this behaviour as an atomic operation. 
The server itself needs to read the version,
check and then write the new object.
Therefore, the library simulates this mechanism through the usage
of a locking mechanism, where the locks are written in a
Amazon DynamoDB table.
To avoid billing issues and simplify local development and testing,
we use LocalStack. LocalStack is a virtualized simulation
of the AWS cloud stack. It provides implementation both for Amazon S3
APIs and DynamoDB.
Connections to the Gateway server are secured with mTLS.
This way the server performs also authentication of the clients.
As the Gateway maintains the shared folder state and has visibility
of the group of users that can access a certain shared folder, 
it can also enforce access control.
We imagine the Gateway server as the server component of an
imaginary company that provides SSF as a service to its clients.


\subsection{Clients}\label{scc:clients}

The client code is written in TypeScript and Rust compiled to Wasm.
In particular, Rust is present for the X.509 certificate parsing and
verification that is shared with the server-side.
The CGKA library of choice is ``mls-rs'' from AWS-Lab (\cref{sc:CGKA-implementations}).
These technologies can all run in the browser. Specifically,
the resulting JavaScript and Wasm code can be bundled
together via Webpack, a common tool to prepare
the code for the browser.
The project includes a folder showcasing the setup with
Webpack.
However, to simplify developing and allow fast iterations and easier benchmarking,
we use Node.js as execution environment at first.
We develop therefore a command line interface (CLI) to test the code.
This way, we avoid the creation of a user
interface (UI)
\footnote{The UI is considered out of scope for this thesis given the short time frame, but it is a natural extension to the project.}
while still testing out the core protocol. 
As noted in \cref{sc:browser-runtimes}
Node.js internally uses V8, which allows us to assume
similar results in terms of performance and memory usage
when running the code in Chrome. It also provides
an implementation of the Web Crypto API.
However, we needed to patch the ``mls-rs'' library
to allow compatibility with Node.js.\footnote{We contribute our changes to the main project repository: \url{https://github.com/awslabs/mls-rs/pull/189}}
We only change how the library is binding the Web Crypto API.
We provider a JS snippet to load the Web Crypto API from
\texttt{node:crypto} instead of the \texttt{window.crypto} global
object from the browser. All the Rust code remains
unchanged. The library contains also another JS snippet
to use \texttt{Date.now()} function, which is written in
ES6 syntax, which Node.js doesn't support.
We therefore add a commonJS compliant re-implementation
of the same snippet which is used
when compiling for Node.js.
To test the resulting JavaScript and Wasm code, we use the Jest~\cite{Jest}
framework, targeting Node.js as an execution environment.\footnote{Jest can also run partially simulating the browser environment using \texttt{js-dom}.}
For the Rust code that is compiled to Wasm, we make use of
``wasm-bindgen-test'' to test the resulting Wasm code in 
Node.js, as well as in headless browsers.



\section{SSF Proxy Server}\label{sc:ssf-proxy-server}


\section{SSF File Changes Sync}\label{sc:ssf-file-changes-sync}






\chapter{Baseline}\label{ch:baseline}

In this chapter, we describe the baseline implementation.
We first describe the mental model
associated with this simpler version
of the protocol (\cref{sc:baseline-mental-model}).
Then, we detail the motivation for this work (\cref{sc:baseline-motivations}).
We end this chapter with a description of the
implementation, providing the details
that are more relevant (\cref{sc:baseline-protocol}).
We highlight that, even though we use well-established
cryptographic primitives, we found it
challenging to implement the baseline correctly
with the limited support provided by the
Web Crypto API.
The cryptographic expertise of both supervisors
of this thesis was really helpful to ensure that the
workarounds we use are secure.

\section{Mental Model}\label{sc:baseline-mental-model}
The baseline is a simple version of the SSF scheme.
As in the SSF scheme, the baseline goal is to provide users, organised in groups, with
the ability to share files among members of the group in
a secure way.
Users are identified through a PKI.
A ``shared folder'' is a collection of files that are shared 
among a group of users. 
The group membership can change.
The file content 
should be accessible only to the users participating 
in the folder. Users use E2EE to ensure the confidentiality
of file contents.
Storage space for the files is
provided by a cloud storage provider. 
Differently from the SSF scheme, advanced security guarantees
like IAC (\cref{sc:iac}) and the admin role (\cref{sc:mental-model}) are not part of the baseline.
Since admins are not part of the baseline, users can only remove themselves from a group, but they cannot remove other users from the group.
The baseline does not provide any mechanism to rotate the group key.

\section{Motivation}\label{sc:baseline-motivations}

The main motivation for the baseline implementation is
to create a simple client to test the infrastructure which 
will be used by the SSF implementation as well.
With this first client, we avoid adding the cryptographic complexity of the SSF scheme, and we can focus
on the server components and client-server communication.

A secondary future use of the baseline is to provide a
na\"ive implementation of a file sharing system 
for performance comparison with the SSF implementation (\cref{sc:future-work}). 

\section{Cryptographic Protocol and Key Hierarchy}\label{sc:baseline-protocol-hierarcy}

The baseline uses simple cryptographic primitives, instantiated with the following algorithms:
\begin{itemize}
    \item Symmetric encryption based on Advanced Encryption Standard Galois/Counter mode (AES-GCM) with 256-bit keys.
    \item Message authentication code, HMAC, using SHA-256 hashing algorithm. 
    \item Key derivation function, HKDF, using the SHA-256 hashing algorithm.
    \item Elliptic curve integrated encryption scheme (ECIES), a hybrid encryption scheme based on:
    \begin{itemize}
        \item Asymmetric Key Exchange based on elliptic curve Diffie Hellman (ECDH), using P-256 curve.
        \item Key encapsulation mechanism (KEM), based on the above ECDH and AES-GCM.    
    \end{itemize}
\end{itemize}

We now describe the key hierarchy used to encrypt files and their metadata:
\begin{itemize}
    \item Each file is encrypted with a randomly sampled symmetric key, called a file key.
    \item Each folder is associated with a randomly sampled symmetric key, called folder key. The folder key is shared among all group members.
    \item The file key and its metadata (name, authors, etc.), which we call collectively the file's ``private data'', are encrypted under the folder key.
\end{itemize}

The folder key sharing protocol is based on ECIES where the folder key
is encrypted under the long-term public key of each group member,
extracted from the user's certificate that can be fetched
from the PKI server (\cref{sc:PKI}).
The key material and file metadata encrypted as described above is
then stored in a metadata file inside the shared folder in the cloud.
The file structure is given by the \texttt{Metadata} TS interface
detailing the following content:
\begin{itemize}
    \item \texttt{folderKeysByUser}, storing for each user the folder key encrypted through ECIES, with all required parameters to perform decryption.
    \item \texttt{fileMetadatas}, for each file id (\cref{sc:ssf-proxy-server}) store its encrypted private data, i.e. the file metadata and the file encryption key.
\end{itemize}

Although the cryptographic primitives used in the baseline are
basic blocks in the cryptographic research literature,
and are standardised by the IETF, the implementation of
ECIES using the Web Crypto API (\cref{sc:webcrypto-api}) was challenging.
We document the issues and the workarounds in \cref{sc:implement-ecies}.

\subsection{Implement ECIES with the Web Crypto API}\label{sc:implement-ecies}

ECIES is not provided as a building block out-of-the-box by the Web Crypto API.
To implement ECIES, we need to first implement 
the KDF and KEM on top of the ECDH, AES-GCM and HKDF algorithms provided
by the Web Crypto API.

The file \texttt{ssf-client/protocol/kemKdf.ts} contains the code for both the KDF and KEM.
The difficulties are rooted in the fact that
in the Web Crypto API, the \texttt{CryptoKey}
object, which is an abstraction used in the API on the byte array containing the key material,
is always designated to perform a specific algorithm and
a limited set of operations.
Trying to use a \texttt{CryptoKey} object for a different algorithm or operation
will result in an error.

Specifically, the library does not allow performing 
HKDF to obtain a new HKDF designated key.
Therefore, instead of separating the implementation
of the KEM and KDF, we implemented them in one single pass.
During the derivation of the symmetric key,
We concatenate all the labels that are used, both
during the KDF and the KEM. This is not optimal,
as we cannot re-use the KDF or the KEM operations
separately, but with this workaround we do not need
to rely on more complicated techniques 
to bypass the limitations of the Web Crypto API,
as done in \cref{sc:ssf-sskg}.

We remind the reader that the KEM and KDF are used to
implement ECIES, which ultimately is used to
share the folder key with new group members.

\section{Implementation Details}\label{sc:baseline-protocol}

The file \texttt{ssf-client/protocol/baseline.ts} collects all the code specific to the baseline.
Unit and integration tests for the baseline are in the subfolder \texttt{test}.
All cryptographic modules that can be found in
\texttt{ssf-client/protocol} apart from ECIES are re-used in the SSF scheme,
and unit tests with 100\% code coverage are included in the
\texttt{test} subfolder.

\subsection{Metadata Synchronization}\label{sc:metadata-synchronization}

As mentioned in \cref{sc:ssf-proxy-server}, the
SSF Proxy server provides concurrency control
when updating the files in the cloud storage (\cref{sc:ssf-file-changes-sync}).
Since the key material is all stored encrypted in a special metadata file
in the cloud
the server needs to synchronize the metadata file updates
to ensure a consistent and ordered cryptographic state of the folder,
i.e., when two or more update requests targeting the same version of the metadata file are received, only one is
accepted, and the others are rejected, so that the
clients fetch the updated state before proceeding
with the changes, and no modifications would be lost.

The server exposes an endpoint to upload a new file
to the cloud storage. The request contains the file
content to upload, which is sent encrypted by the client
(\cref{sc:baseline-protocol-hierarcy}). The client
also needs to provide:
\begin{itemize}
    \item The ETag (version) of the metadata file on which the update is based.
    \item The new metadata content to be stored in the shared folder metadata location.
\end{itemize}
Requests where the ETag does not match the current version
of the metadata file are rejected. Given the
synchronization mechanism described in \cref{sc:ssf-file-changes-sync},
all clients will operate on consistent data, and
will not lose updates.

The first creation of the metadata file instead happens
at shared folder creation, where the server will
accept the first metadata file content upload
without requiring any ETag. A baseline client
creating a shared folder will initialise the metadata
file with the folder key encrypted under its own
public key.

The synchronization mechanism implemented
for the baseline will be re-used in the SSF scheme
implementation, see \cref{sc:ssf-file-encryption}.

\chapter{Secure Shared Folder}\label{ch:ssf}

In this chapter, we describe in detail the Secure Shared Folder (SSF) scheme (\cref{sc:SSF-scheme}) and its implementation.
We detail the cryptographic implementation and usages of supporting
libraries, as well as their implications in terms of runtime execution
and the challenges we encountered (\cref{sc:ssf-sskg}, \cref{sc:ssf-double-prf}, \cref{sc:DKR-implementation}, \cref{sc:GRaPPA-implementation}). 
For each of them, we describe how the implementation diverges from
the paper's construction pseudocode and the reasons behind them.
While implementing these underlying primitives indeed, we also discovered 
and fixed a few major issues, which caused failures in the 
synchronization of the cryptographic state among clients (\cref{sc:GRaPPA-bugs}).

We also discuss the SSF Gateway modifications 
(\cref{ssc:delivery-service})
together with the SSF scheme state synchronization, persistency and rollbacks
strategies (\cref{sc:state-sync-rollbacks}). In this context,
we explain why it would be beneficial to model precisely the
interactions between clients and servers in cryptographic
primitives that are used in a distributed setting and highlight
gaps created by loose abstractions of such communication.

We briefly describe how we perform the upload and encryption of files
and their metadata and key material (\cref{sc:ssf-file-encryption}).

We provide a detailed description of some issues we encountered
while using the Web Crypto API (\cref{sc:Web-Crypto-API-implementations:-non-standard-behaviours}),
and propose a change to the API specification to solve them.
Also, we propose enhancements to the MLS library 
(\cref{sc:MLS-enhancements}) and we detail
what features MLS implementations should provide to better support
the implementation of new cryptographic primitives on top of
CGKA and/or MLS.

Finally, we provide our enhancement proposals to the primitives
from~\cite{GKP}, 
such that they can better model the different client entities with respect
to their cryptographic state (\cref{sc:DKR-enhancements}).


\section{The SSF scheme}\label{sc:SSF-scheme}

Using the GKP primitive (\cref{sc:gkp-scheme}), 
we can informally describe the Secure Shared Folder (SSF) scheme.
As discussed in \cref{sc:mental-model}, we want to create 
a shared folder, containing E2EE files, which are shared
among a dynamic group of users that interacts asynchronously, where
some group members are admins, and provide interval access control security (\cref{sc:iac}) 
for the shared files. 
All members should be able to
upload files to the folder, which are stored
in a cloud storage solution. 
Admins, as in GKP, can manage the
memberships and can advance the group's shared secret.

Intuitively, the SSF scheme provides operations to:
\begin{itemize}
    \item Upload and encrypt files in the shared folder for all users that
    are currently members.
    \item Download and decrypt files that the user has been previously granted access to. 
    \item List the files in the shared folder.
    \item Refresh the private state of the members.
    \item Add and remove users from the shared folder (only admins).
    \item Grant and revoke admin privileges to members of the shared folder (only admins).
    \item Rotate the shared group secret (only admins).
\end{itemize}

The SSF scheme applies the GKP primitive to manage a sequence
of shared epoch group secrets, which in the context of SSF are called
``folder keys''. 
Each file is encrypted under a randomly sampled symmetric file key.
The file keys, chosen at upload time, are encrypted under the current
folder key. Together with the file keys, sensible metadata of the files
are also encrypted under the current folder key.
The ciphertext containing the file keys and the metadata is stored
in a special file in the folder inside the cloud storage provider,
to offload the space from the client devices to the server.
The implementation details are given in the remainder of the chapter.

\section{Threat Model}\label{sc:threat-model}

The SSF scheme is designed to protect the confidentiality and integrity
of the files stored in the shared folder against 
the cloud storage provider and other server components (such as SSF Gateway server), 
which are considered honest but curious.
An honest but curious server follows the protocol specification
but tries to learn as much as possible from the data it processes.
We remind that file deletion is not safe (\cref{scc:cloud-storage-assumptions}).
The scheme also protects against compromise of non-admin
members of the group, and provide IAC security in case
of a member compromise (\cref{sc:iac}).
Malicious active members, can be purged from the group by the admins.
As in GKP, the scheme does not protect against a malicious active admin~\cite{GKP}.


\section{SSKG implementation: TreeSSKG}\label{sc:ssf-sskg}

We start in this section and following ones with describing the
implementation of all primitives used internally in the SSF scheme.

The DKR construction uses SSKG, which is introduced in \cref{sc:SSKG}.
However, there is no browser-compatible implementation available to the best of our knowledge.\footnote{We found a reference Go implementation~\cite{SSKGGo}, which was used as reference together with the pseudocode in~\cite{ESORICS:MarPoe14}. However, the Go implementation, even if compiled to Wasm, would not use the Web Crypto API, and would thus be insecure. Also, compiling Go to Wasm is not recommended as noted in \cref{sc:browser-runtimes}.}
Therefore, we re-implement the tree version of SSKG, whose pseudocode is present in~\cite{ESORICS:MarPoe14}, using TypeScript.
The choice of the tree-based (\cref{sc:SSKG}) implementation is motivated both by its efficiency
compared to the numerical construction and because it only uses common cryptographic
primitives, such as hash functions, PRGs or block ciphers, which are supported by the Web Crypto API.
More importantly for the practical use case, it provides \texttt{SuperSeek}
functionality, which is a generalisation on top of the \texttt{Seek} procedure.
While \texttt{Seek} can only calculate an output starting from the initial state, 
\texttt{SuperSeek} can start the forward derivation from an arbitrary point of the sequence.
This allows the usage of SSKG inside the GRaPPA construction,
as we can just share the state of the SSKG chains from and until
the epochs we want to give access to. 
We pay more in terms of additional bytes sent over the wire, each SSKG
state in the tree-based construction requires up to $O(log(n))$,
where $n$ is the maximum number of elements in the sequence generated by the SSKG.

The subfolder \texttt{ssf-client/src/protocol/sskg/} of the project contains the code for the SSKG module.
The file \texttt{sskg.ts} specifies the functionalities exported by an instance of the object.
The file \texttt{treeSSKg.ts} contains the implementation of the tree-based SSKG
as a TypeScript class \texttt{TreeSSKG}. The unit tests are in \texttt{test/treeSSKG.test.ts} file,
completely covering the code, with some additional randomized tests for further validation of the code.
The \texttt{TreeSSKG} class maintains the state internally in fields:
\begin{itemize}
    \item A read-only \texttt{name}, used to identify the SSKG instance and helpful in debugging.
    \item A read-only \texttt{totalNumberOfEpochs}, a positive integer representing the total number of elements derivable from this TreeSSKG instance.
    \item A mutable \texttt{stack} field, a stack data structure, i.e.\ a JS array, storing the internal state of the SSKG as detailed in~\cite{ESORICS:MarPoe14}.
    The stack is used as a more convenient and performant way of 
    representing a pre-order traversal on the tree structure on 
    which the SSKG is based on. Each element of the stack is a tuple 
    of the form $[s, h]$, where $s$ is an element in the pseudo-random 
    sequence, stored as a byte array of type \texttt{ArrayBuffer}, 
    while $h$ is the height of the node storing $s$ in the tree.
    The tree structure is implicitly represented by the stack,
    where the top element is always the next element that would
    be explored by the traversal on the tree.
\end{itemize}

The implementation required the following cryptographic operations:
\begin{itemize}
    \item Generate a starting element (seed) for the SSKG.
    \item A pseudo-random function (PRF) to derive the next element in the sequence. 
\end{itemize}

This seed is generated using the Web Crypto API \texttt{subtle.generateKey} function 
to create an HMAC SHA-256 symmetric key which is then exported to raw bytes using \texttt{exportKey},
and stored in the \texttt{stack}.

The Web Crypto API provides the necessary cryptographic primitives
through the \texttt{subtle} object, with the functions
\texttt{generateKey} and \texttt{deriveKey},
which support HMAC and HKDF algorithms.
Recall that the Web Crypto API representation for a
key, is made through the \texttt{CryptoKey} JS object,
which is a wrapper around the actual key material.
This object contains information about the key, such as
whether the key is symmetric or asymmetric, in the latter
case if it is private, the algorithm for which the key is used
and whether it can be exported or not to e.g. raw bytes.
To implement the PRF, we need to chain HKDF calls (implemented by \texttt{deriveKey}).
However, a call to \texttt{deriveKey} cannot produce a \texttt{CryptoKey}
designated for HKDF, thus we cannot reuse the output for a new
key derivation directly. 
To avoid the problem, we add an intermediate step,
where we perform HKDF to derive first an HMAC key,
then we use the bytes of the HMAC exported through
\texttt{exportKey}, by re-import them
using \texttt{importKey}
into an HKDF \texttt{CryptoKey} object.
Thus, we can chain the HKDF key derivations and work around the limitations
of the Web Crypto API.
This also explains why the initial seed is generated as an HMAC SHA-256 key.
We note that HKDF internally uses HMAC,
justifying our choice of using HMAC-designated
keys when implementing the PRF.
To ensure that the bits of the keys are not truncated,
we fix the hash function to SHA-256 in all cryptographic
operations that require a hash function.
When invoking any API call where an HKDF or HMAC algorithm is involved,
we therefore always specify that the hash function is SHA-256 (see \texttt{ssf-client/src/protocol/commonCrypto.ts}). 

The method \texttt{GetKey}
is used to retrieve a \texttt{CryptoKey} designed for HKDF.
This is done by first applying the PRF, with a \texttt{key} label, and then importing the bytes obtained from the PRF
through \texttt{importKey}. 

In our implementation we also expose two additional methods:
\texttt{getRawKey} and \texttt{clone} which are not present 
in the SSKG primitive. 
\texttt{getRawKey} method is equal to \texttt{GetKey}
but does not perform the final import of the key, thus it 
returns the current element of the sequence as raw bytes
in an \texttt{ArrayBuffer}. 
This utility will be used in~\cref{sc:ssf-double-prf}.
The \texttt{clone} method, as the name intuitively suggests, 
is used to create a new instance
of the \texttt{TreeSSKG} class with a deep copy of the state.\footnote{In object-oriented programming, when copying an object we normally distinguish between shallow and deep copy. 
A shallow copy is a copy of the object where the fields are 
copied by reference, while a deep copy is a copy where the 
fields are copied by value. 
In our case, we will need a deep copy of the state, 
as the state is mutable, and we want to avoid side effects 
when manipulating the copy, for example, to keep access to the current element before seeking the instance.} 
Refer to \cref{sc:DKR-implementation} for its usages inside DKR. 
Also, it has extensively been used in our unit tests, to check the equivalence
of state updates performed through the different SSKG operations.

\paragraph{Serialization} The code also provides utilities for serializing and deserializing 
the object to persist the state in the client's browser storage
as well as sending it to the other members of the group.
Since the internal state contains raw bytes, we use the concise binary 
object representation (CBOR)~\cite{rfc8949} standard to encode it. 
CBOR supports natively the encoding of byte arrays, which is not the case
for other encodings, such as the widely supported JSON.

\subsection{Asynchronous classes initialization in TypeScript}\label{pg:async-classes-init}
We highlight that all the code using the Web Crypto API is 
asynchronous by design to avoid blocking
the main thread of the browser. Blocking the JS main thread
would result in a really poor user experience, as the browser
would be unresponsive while waiting for the cryptographic
operations to complete. This introduces some complexity
when writing stateful classes like \texttt{TreeSSKG},
or later the implementations of DKR (\cref{sc:DKR-implementation})
and GRaPPA (\cref{sc:GRaPPA-implementation}), as the state
of the object is not immediately available after the constructor
is invoked. In TypeScript, one way of elegantly handling
asynchronous constructors, which are not supported, is to make the constructor private
and instance fields that require asynchronous initialization
private. Then expose a static ``factory'' method to create an instance of the class.
This method will return a \texttt{Promise} containing the object.
The \texttt{Promise} will be resolved after:
\begin{enumerate}
    \item The constructor has 
    completed its execution, where all synchronous available state
    is initialized, and an instance of the object has been allocated,
    but the fields requiring asynchronous initialization are still
    empty.
    \item The remaining empty private fields are also filled with the missing asynchronous state.
\end{enumerate}
Keeping the constructor private prevents any client code from instantiating the object directly, thus protecting developers from partial object state creation errors.
We use this pattern in all TypeScript classes that require asynchronous initialization.

\section{Sign using HMAC as a double-PRF}\label{sc:ssf-double-prf}
In \cref{sc:DPRF} we recall the notion of double-PRF security.
As we have seen, under the assumption of fixed key length, HMAC 
can be used to instantiate a double-PRF secure construction.
In GRaPPA the generation of shared keys is delegated to the
DKR construction (\cref{sc:DKR-implementation}), which 
internally combines elements of fixed
length from two chains to derive a key (\cref{sc:DKR}).
In our implementation, elements generated from any \texttt{TreeSSKG}
always have a fixed length of 256 bits, so we can use HMAC as a
double-PRF, to 
combine backward and forward chain elements 
(one per HMAC input) to
produce a DKR key, which will be returned by GRaPPA through
the \texttt{GetKey} operation.

The code for the double-PRF implementation can be found in the TS file {\texttt{ssf-client/src/protocol/doubleprf/}}.
The implementation accepts two byte arrays $k_1$ and $k_2$ 
as \texttt{ArrayBuffer} in input and execute the {\texttt{subtle.sign}} 
function of the Web Crypto API. 
Unfortunately again, the Web Crypto API does not provide a direct way to
use HMAC as a function that works on raw bytes. We instead
need to first import $k_1$ in a \texttt{CryptoKey} object
designated for HMAC-SHA-256 and sign algorithm,
then use the \texttt{subtle.sign} function to ``sign''
(that is equivalent to computing HMAC) $k_2$.
The resulting \texttt{ArrayBuffer} containing the
256 bits signature is then imported back as
a \texttt{CryptoKey} object, designated for HKDF.
We checked the underlying implementations of the Web Crypto API
both for Chrome and Node.js and made sure the bits
from both $k_1$ and $k_2$ are fully used, and that
\texttt{subtle.sign} correctly calls HMAC($k_1$, $k_2$).

This is a careful workaround around the limitations
of Web Crypto API, which however should be revised
in the future, when multi-input key ``combiners''
will be available in the API.
Again we rely on the fact that the hash function is
fixed to SHA-256 in all cryptographic operations
to ensure that the bits of the keys are fully utilised
and not truncated. We specify useful constants
in a common place: \texttt{ssf-client/src/protocol/commonCrypto.ts}.

\section{DKR implementation: KaPPA}\label{sc:DKR-implementation}

In \texttt{ssf-client/src/protocol/key-progression/} we provide
the implementation and tests for the DKR primitive instantiation
D[F, S]~\cite{GKP}. As specified in the D[F, S] construction, we use
SSKG (\cref{sc:ssf-sskg}) to implement efficiently hash-chains supporting seek capabilities from any element of the sequence. 
We only use the methods exported by the
\texttt{SSKG} interface to implement the operations from the primitive. 

\paragraph{Types Design}
\texttt{dkr.ts} contains the 
type definitions for double-key regression, resembling
the DKR syntax. We also provide the types for the various
concepts used in the DKR primitive (\cref{sc:background-generalised-DKR}) in the same file:
\begin{itemize}
    \item An epoch is just a plain \texttt{number}, aliased as \texttt{Epoch} for readability. We recall that JS numbers are always 64-bit double precision floating point (\cref{sc:webcrypto-api}), however epochs are always positive integers (see \cref{sc:gap-type-safety-of-opaque-byte-arrays}).
    \footnote{For the curious practitioner: the TypeScript type-system allows the usage of union types, e.g. ``\texttt{type positiveUntil3} $ = 1 | 2 | 3; $''. However, the type checker in the TS compiler has an upper limit on the number of elements in a union type. We could generate a type to express a range of positive integers up to a given number through metaprogramming. Again however, the TS compiler limits the recursion depth to 999 (in the version used in this project). Another option would be to write a conditional type that checks if the string representation of the number contains a dot and/or a minus sign, and returns a type \texttt{never} in these cases, the number literal otherwise. The problem with each of the options presented is that in practical terms either we hit a compiler limit or create a type which becomes unusable in practice.}
    \item An epoch interval $[l, r]$ is represented as a JS object with properties \texttt{left} and \texttt{right}, where both properties are of type \texttt{Epoch}.
    \item The DKR blocks are expressed as an enum type \texttt{BlockType}.
    \item Each forward chain is represented as a tuple of two elements $[e, ssgk]$ called \texttt{ForwardChain}, where $sskg$ is an \texttt{SSKG} object (\cref{sc:ssf-sskg}) and $e$ is the DKR \texttt{Epoch} at which we released the first element of $sskg$.
    \item Each backward chain is a tuple of three elements $[e, sskg, N]$ called \texttt{BackwardChain}, where $sskg$ is an \texttt{SSKG} object (\cref{sc:ssf-sskg}) and $e$ is the DKR \texttt{Epoch} at which we release the first element (in backward, i.e., reverse, order) of $sskg$, similarly to the forward chains, while the number $N$ specify an upper limit on the number of elements that can be generated by $sskg$. Recall that $sskg$ maintains internally a read-only property \texttt{totalNumberOfEpochs}. It follows that $N \leq$ \texttt{totalNumberOfEpochs}.
    \item We also provide a type to model an ``interval state'', which we call \texttt{DoubleChainsInterval}. An interval state is a slice extracted from a full DKR state, which gives access to forward and backward elements in a given \texttt{EpochInterval}. The property \texttt{forwardChainsInterval} (resp. \texttt{backwardChainsInterval}) contains the slices (as JS arrays) of the forward (resp. backward) chains, where the first (resp. last) chain are shrunk (\cref{sc:DKR}). We notice that both intervals and extensions of the DKR syntax map to this type.
\end{itemize}

\paragraph{Object-oriented Design}
The class \texttt{KaPPA} in {\texttt{kappa.ts}} implements the
D[F, S] construction. The state comprises:
\begin{itemize}
    \item \texttt{maxEpoch}, of type \texttt{Epoch}, representing the current epoch of the DKR state. This means for all epochs from 0 to \texttt{maxEpoch} we can get a key.
    \item \texttt{forwardChains}, a JS array of \texttt{ForwardChain}.
    \item \texttt{backwardChains}, a JS array of \texttt{BackwardChain}.
    \item A read-only \texttt{maximumIntervalLengthWithoutBlocks}, a positive integer representing the maximum number of elements that can be generated by an SSKG used internally.
    Indeed, when creating a \texttt{TreeSSKG} instance, we set the \texttt{totalNumberOfEpochs} to this value (\cref{sc:ssf-sskg}).
\end{itemize}

As mentioned above, numbers are 64-bit double-precision floating point in JS,
therefore the maximum epoch to which we can progress in a DKR instance before
loosing precision in the epoch representation is $2^{53} - 1$.
If we progress the DKR state every second, thus creating a new key
every second, this would correspond to keys for
a time span of more than 285 million years. 
Therefore, we can safely
assume that the epoch space is large enough for any practical use case.
Further, we highlight that all calls to the Web Crypto API are embedded 
inside the \texttt{TreeSSKG} instances that made up the chains in \texttt{KaPPA}. 

Recall also that the operations of the DKR primitive are:
Init, Progress, GetInt, CreateExt, ProcExt, GetKey. Further, we recall
that the operation GetKey is defined for both a full DKR state and
an interval state.
While the mathematical
description of D[F, S] only partially distinguishes between
a full DKR state and an interval state, we have chosen to model
this distinction more clearly in the implementation.
Precisely, we map an instance of \texttt{KaPPA} to a user with
read/write access to the DKR state, while a read-only
user will never construct a \texttt{KaPPA} instance, but will
only maintain and query interval states of type
\texttt{DoubleChainsInterval}. Even if the distinction does not
provide any additional cryptographic guarantee, it serves the purpose of
making the code better to understand and use, as well
as helping in the identification of potential bugs in the implementation
or errors and enhancements in the primitives (as explored in \cref{sc:correcting-primitives}).

The state distinction is reflected on how the operations are attached
to the \texttt{KaPPA} class. All operations that either require
to access the full DKR state or to update it, are instance
methods of the \texttt{KaPPA} class. We call these ``admin methods'',
as they will be used only by admins of the folders in GRaPPA (\cref{sc:GRaPPA-implementation}).
\texttt{Progress}, \texttt{GetInt} and \texttt{CreateExt} 
are all ``instance methods''.
The only exception is \texttt{Init}, which is a ``static method'' 
of the class for the motivations explained in \cref{pg:async-classes-init}.
Diverging from the DKR syntax, instance methods do not take the DKR state in input
as parameter, as we can access the instance's private state directly,
better modelling the fact that DKR is stateful.
Further, \texttt{Progress} does not return the modified state as output, 
but instead it modifies the state of the \texttt{KaPPA} instance in memory.
Modelling the primitive as a class with private state also
allows us to easily keep multiple different instances
in memory, which is practically useful for multi-tenancy, i.e., 
for a client handling multiple shared folders with GRaPPA.
The ProcExt and GetKey operations are instead static methods,
thus usable without constructing a \texttt{KaPPA} instance,
and are the only methods that can be used by members in GRaPPA
which do not have write access to the DKR state (\cref{sc:CGKA-implementations}).
GetKey is also defined as an instance method for convenience,
but internally it will first construct an interval state and
then call the static GetKey method on it, as described in the
DKR syntax.

We highlight that each time we need to seek a \texttt{TreeSSKG}
instance stored inside \texttt{KaPPA} forward or backward chains,
with the execution of \texttt{Progress} method, which is the only method
that modifies the internal state, we always clone the \texttt{TreeSSKG}
instance. This is needed to avoid side effects
deriving from the mutable state of the \texttt{TreeSSKG} instances. 

\paragraph{Serialization} As seen in SSKG, our implementation also provides utilities
to serialize and deserialize the state of the DKR object
using CBOR encoding, as we will ultimately need to
store the state in the client storage.
The upper bound on the cost of serializing the state is determined
by the number of chains (\texttt{TreeSSKG} instances),
as we need to recursively serialize the state of SSKG first.
Since we need to support also member state serialization and deserialization,
we provide static methods to serialize and deserialize interval
states of type \texttt{DoubleChainsInterval}.

\paragraph{KaPPA time efficiency}

Regarding the internals and efficiency of the D[F, S] construct
implementation \texttt{KaPPA}, we translated the helper functions 
GetFChains and GetBChains to get
slices of forward and backward chains into a modified binary search,
to avoid the linear search in the original pseudocode.
Notice that the chains are stored ordered by the epoch of instantiation,
and recall the definition of \texttt{ForwardChain} and \texttt{BackwardChain}:
we can see that the first element of both tuples is the epoch.
Therefore we have written a generic binary search function that,
given an epoch, returns the index in the array of the forward (resp. backward)
chains of type \texttt{ForwardChain} (resp. \texttt{BackwardChain})
where the first tuple element is equal to the input epoch. If the epoch is not
present, the function returns the index of the element with the 
closest epoch that is smaller than the input epoch.
As all \texttt{KaPPA} instance methods, i.e., admin operations,
are relying on these helper functions, each of them has a time
complexity of $O(log(n))$, where $n$ is the number of forward (resp. backward)
chains in the \texttt{KaPPA} instance.

\section{MLS implementation: mls-rs in JavaScript}\label{sc:js-bindings-for-mls}

To build the GRaPPA protocol, we need a browser-compatible implementation
of MLS (\cref{scc:clients}).
In this section, we detail how we use and export the functionalities from mls-rs
into JS. We also give an overview of the internal state management
that is relevant for later discussions on state synchronization and resiliency
against errors (\cref{sc:state-sync-rollbacks}).
Recall from \cref{ch:setup} that mls-rs is a Rust library, which we
compile to Wasm to execute in the browser.
Wasm runtime has a separate memory and runs in parallel
to the JS main thread in a separate execution environment, however, communication
is possible across boundaries. Some restrictions apply to the data types
that can be transferred between the two environments, as Wasm natively supports
only numeric types and arrays. Thus, the general strategy 
is to minimize sending objects or any other complex data structures 
while writing our bindings.\footnote{The term ``bindings'' refers to code that allows calling Wasm functions from JS.}

\paragraph{Compilation and Dependency Management}
As we are using \texttt{wasm-pack} to compile mls-rs to Wasm, we automatically
get a ready-to-use JS module, including the type definitions for TypeScript,
which can be installed as a NPM dependency. The \texttt{wasm-pack} also 
provides the JS shim to allow sending strings and work with the Wasm linear
memory. Further, we can compile asynchronous Rust code to Wasm, as support is also
provided by the toolchain. For details, see the documentation in~\cite{WasmBindgen}

The code for exposing the mls-rs functionalities is provided in the Rust library crate
\texttt{ssf/}. We detail how to compile the code in the \texttt{README.md}
as well as how to run the unit tests written with \texttt{wasm-bindgen-test}.
The tests successfully execute both in Chrome browser and Node.js.
The crates from AWS mls-rs library are included in \texttt{Cargo.toml}
file, where we specify our branch of the repository which 
includes the changes we made to the library to fix the compatibility issues
with Node.js (\cref{sc:MLS-enhancements}). From the library, we use the following crates:
\begin{itemize}
    \item \texttt{mls-rs}: the main library for the MLS protocol, exporting the Rust client and state storage in-memory providers.
    \item \texttt{mls-rs-crypto-webcrypto}: the cryptographic provider implementation for Wasm target. This is based on the Web Crypto API.
    \item We further explicitly specify the \texttt{mls-rs-core} crate because we want to use our branch of the library and enable the compilation options needed for compatibility.
\end{itemize}

For practitioners, we highlight that
the correct compilation of the library to Wasm is possible only by enabling the
(undocumented) compiler flag \texttt{rustflags = "--cfg mls\_build\_async"}. This option can be added
to the \texttt{Cargo.toml} under the \texttt{[build]} stanza.

\paragraph{Error Handling}
The mls-rs library uses an internal error type to handle errors, as generally
done in Rust. The \texttt{MlsError} enum represents all the possible errors generated
by the library, with their description.
In the JS bindings written in \texttt{ssf/lib.rs} every function returns a Rust
\texttt{Result} type, expressing either a successful result or an error.
The error is always converted to a string, the description of the error itself,
before sending it to JS. 
Note that since all the functions we export are asynchronous,
the \texttt{Result} is converted to a JS \texttt{Promise} in the JS shim.
A \texttt{Promise} object in JS represents the eventual completion (or failure) 
of an asynchronous operation, and its resulting value or error.
When calling the bindings from our TypeScript client implementation,
we can therefore easily handle the errors,
as if the code was natively written in JS.

\paragraph{Application Messages}
The main entry point of the library is the \texttt{Client} struct, which
is an object providing functionality to create a MLS (internally CGKA)
group and evolve its state. Further, the \texttt{Client} allows the creation
and encryption of application messages (\cref{sc:MLS}). While in the pseudocode 
of GRaPPA there is no explicit mention to MLS protocol, but only to CGKA,
notice that while executing the operations of the GRaPPA protocol,
parts of the control message are encrypted through Authenticated Encryption
with Associated Data (AEAD) using the CGKA shared epoch secret.
Instead of manually implementing this encryption, we just use the
support for application messages provided by MLS. This difference highlights
that although for demonstration purposes the MLS protocol is kept out of scope,
practically the GRaPPA protocol is built on top of MLS and not only CGKA, 
and further motivate our choice to use an MLS library.

\paragraph{Key Packages}
The \texttt{Client} object provides also functionalities to create
key packages, which are the cryptographic material needed to invite
a new member to the group. A key package can be created through the provided
bindings and returned as a \texttt{Uint8Array}, the serialization and
deserialization of the key package (when processing it) is handled by the 
library itself. 


\paragraph{State Management}
A \texttt{Client} object can handle multiple MLS groups.
An abstraction layer on top of the storage is provided by the library,
thus allowing for different storage providers to be implemented and used.
The core library provides an in-memory storage provider.
The \texttt{mls-rs-provider-sqlite} crate implements a storage provider
on top of the relational database engine SQLite. However, this is not compatible
with the Wasm target, as SQLite is not available in the browser environment.

We resort on the in-memory storage provider, which is not persisted
across browser sessions. We propose an implementation of a browser-compatible
storage provider in~\cref{sc:MLS-enhancements}, but leave it out of scope
for this project, because of the time constraints.
When using the in-memory storage provider a further problem arises:
since we are making calls from the JS runtime to the Wasm runtime,
the Wasm memory is freed after a call ends.
Therefore, the state is lost after the portion of code running in Wasm
returns to JS.
We solve this problem by instantiating the storage provider-related objects
in a lazy-initialised global map addressed by user identity, which is kept in
the Wasm memory across calls. This hack is not ideal and only temporary,
until we implement a browser-compatible persistency solution.
Also, to avoid losing the changes in the group state between calls to
exported functions, we need to write the state back to the storage provider
after each operation that modifies it.

On a related note, recall that the MLS protocol is using a proposal
and commit mechanism to evolve the state of the group (\cref{sc:MLS}).
The \texttt{Client} object can be used to get a handle on the \texttt{Group}
object, which is the main interface to interact with a CGKA group.
The \texttt{Group} object provides functions to create proposals,
commit them, and apply the state changes. Further, it also provides
functions to encrypt and decrypt application messages.
Creating a commit message, which is the final step
in the proposal and commit mechanism, does not update the \texttt{Group}
current state until a call to apply the changes is made on the object.
In between the creation of the commit message and the
application of the changes, the \texttt{Group} object maintains
the pending commit in a staging area~\cite{AWSMLSGroup}.
This staging area is also persisted in the storage provider.
However, the modifications to the cryptographic state of the group
deriving from the pending commit are not taken into consideration
when encrypting application messages, i.e., the encryption
is done using the current state of the group only, and not the pending commit.
This limitation has effects on the state synchronization and rollbacks
as described in \cref{sc:state-sync-rollbacks}.


\section{Delivery Service implementation}\label{ssc:delivery-service}

The delivery service (DS) is a new component of our system needed 
to support the usage of MLS (\cref{sc:MLS}).
The DS is responsible for delivering messages to clients in order.

We implement the abstract functionalities of the DS inside our existing server, the SSF Gateway
server. As the SSF Gateway server is already responsible for the creation of folders,
authentication of users, and the access control of users to folders, 
it is a natural choice to extend its functionalities to also provide APIs
for the DS.
To this end, we modify and extend the SSF Gateway server and its SQL database as follows.
We add new endpoints to receive, fetch and acknowledge
GRaPPA controls messages, as well as key packages.
We store in the ``ds'' MySQL database the new entities, and use the
DB transaction support to synchronize multiple clients executing
the GRaPPA protocol. Furthermore, we use the server as a broadcaster to
deliver the messages to the clients in order. The ordering guarantee
is provided by our usage of MySQL.

\paragraph{Additional Database Entities}
We extend the SQL database ``ds'' with three different tables:
\begin{itemize}
    \item \texttt{key\_packages} storing the key packages (see \cref{sc:js-bindings-for-mls}) generated by the MLS clients for later retrieval by a different user which wants to send an invitation to join a folder.
    \item \texttt{pending\_group\_messages} storing the first part of a GRaPPA control messages of type \texttt{Proposal} that are pending and need to be delivered to the clients (\cref{sc:GRaPPA-implementation}).
    \item \texttt{application\_messages} storing the second part of a GRaPPA control messages of type \texttt{ApplicationMessageForPendingProposals}, that are pending and need to be delivered to the clients (\cref{sc:GRaPPA-implementation}).
\end{itemize}

All the data from the client is sent as a byte array, and stored in each table
directly by the server. The server is not aware of the content of the messages
and does not need to be.

The \texttt{key\_packages} table has the following columns:
\begin{itemize}
    \item \texttt{key\_package\_id} as a primary key, a unique autoincremented identifier for the key package.
    \item \texttt{key\_package} as a \texttt{BLOB}, storing the key package.
    \item \texttt{user\_email} a foreign key to \texttt{users.user\_email}, storing the email of the user that created the key package. A \texttt{CASCADE DELETE} constraint is set to delete the key package when the referenced user entity is deleted to automatically clean up the database.
\end{itemize}
Key packages do not belong to a folder, but just to a user, and are used to
add a new user to a folder in GRaPPA, since the user needs to be added to the
member CGKA group (\cref{sc:gkp-scheme}).

The \texttt{pending\_group\_messages} table has the following columns:
\begin{itemize}
    \item \texttt{message\_id} as a primary key, a unique \texttt{INTEGER} autoincremented for the message.
    \item \texttt{payload} as a \texttt{BLOB}, storing the message.
    \item \texttt{folder\_id} a foreign key to \texttt{folders.folder\_id}, storing the unique identifier of the folder to which the message belongs. A \texttt{CASCADE DELETE} constraint is set to delete the message when the referenced folder entity is deleted to automatically clean up the database.
    \item \texttt{user\_email} a foreign key to \texttt{users.user\_email}, storing the email of the user to which the message has to be delivered. A \texttt{CASCADE DELETE} constraint is set to delete the message when the referenced user entity is deleted to automatically clean up the database.
\end{itemize}
Since we store messages by \texttt{user\_email} we are replicating
the \texttt{payload} content for each user that is part of the folder 
referenced by \texttt{folder\_id}. This will be changed and was done as a first
development iteration to simplify the implementation. Ideally, this table should be
normalized, and a new table containing only the \texttt{message\_id}
and the \texttt{payload} should be created to store the content of the messages
independently from the receiving users.

The last table is effectively modelling an ordered queue of messages:
the messages are globally ordered by insertion time through the \texttt{message\_id},
so for each user and folder, the subset of messages belonging to that
user and folder are also ordered by insertion time.
We call a queue the ordered subset of messages belonging to a user and folder in the table.
We will see in the following paragraph about the API how the server guarantees consistent state updates
of the clients.

The \texttt{application\_messages} table has the following columns:
\begin{itemize}
    \item \texttt{id} as a primary key, a unique \texttt{INTEGER} autoincremented for the application message.
    \item \texttt{message\_id} a foreign key to \texttt{pending\_group\_messages.message\_id}, storing the unique identifier of the message which is completed by this application message. A \texttt{CASCADE DELETE} constraint is set to delete the application message when the referenced pending group message entity is deleted to automatically clean up the database.
    \item \texttt{payload} as a \texttt{BLOB}, storing the message.
\end{itemize}

This table stores the second part of the GRaPPA control messages, which reference 
the first part stored in the \texttt{pending\_group\_messages} table.
Similarly, we simplify and speed up development by replicating the
payload for each foreign key to a pending group message.

\paragraph{DS APIs}
We extend the SSF Gateway server with the following new endpoints to support the DS functionalities:
\begin{itemize}
    \item \texttt{POST /users/keys}: upload a key package.
    \item \texttt{GET /users/<folder\_id>/keys}: fetch and delete a key package given an existing folder id. The folder id parameter is needed for access control, to verify that the user requesting the key package of another user is part of the corresponding folder. The server will return an error in case the user for which the key is requested is already part of the given folder.
    \item \texttt{POST /folders/<folder\_id>/proposals}: upload the first part of a GRaPPA control message for the specified folder. The server checks that the sender user identity is part of the folder, and that the user's pending message queue for the folder is empty. The upload will transactionally\footnote{In databases terminology a transaction is a sequence of operations which are grouped together. The operations can either be all executed successfully and the resulting change is persisted, or none takes effect. A transaction is ``committed'' when all operations are completed or ``aborted'' if any failed.} check and update the database, inserting the message in the \texttt{pending\_group\_messages} table, replicated for each user in the folder by the server through the \texttt{users\_folders} table, excluding the sender user.
    \item \texttt{PATCH /folders/<folder\_id>/proposals}: upload the second part of a GRaPPA control message for the specified folder and message ids sent in the request body. The server performs the usual access control checks, and transactionally checks and inserts the message in the \texttt{application\_messages} table, replicated for each message id in the request body. This is a PATCH HTTP request, as logically it is patching the first part of the GRaPPA control message with the second part. Furthermore, a notification containing the folder id is sent to the clients receiving the message, to signal that they should try to fetch new GRaPPA control messages from the server. 
    \item \texttt{GET /folders/<folder\_id>/proposals}: fetch the first GRaPPA complete control message for the specified folder from the user's queue. The server checks that the user is part of the folder, and returns the first pending message together with the corresponding application message if available.
    \item \texttt{DELETE /folders/<folder\_id>/proposals/<message\_id>}: delete the GRaPPA complete control message for the specified folder and sender user. The server deletes both related entries, transactionally from the tables \texttt{pending\_group\_messages} and \texttt{application\_messages}.
    \item \texttt{PATCH /v2/folders/<folder\_id>}: given the folder id, share the corresponding folder with the user specified in the request body. The body further includes the bytes of the first part of the GRaPPA control message which invites the new member to the folder. Since this operation requires a different DB transaction to be performed compared to the original version of the endpoint, we defined a v2 version of the endpoint to also avoid regressions in the baseline implementation.
\end{itemize}

Notice that, the GKP scheme and the GRaPPA construction do not capture the
interaction with the server, and do not assume the server
to hold any state. However, in our implementation we
need to take both things into account. 
The DS state, which is made up of the pending messages, reflects the fact that
the cryptographic state of the clients for which the queues are non-empty
is out of sync with respect to other group members.
We will discuss the implications in \cref{ssc:GKP-client-middleware} and \cref{sc:state-sync-rollbacks}.

\section{GKP Implementation: GRaPPA}\label{sc:GRaPPA-implementation}

The group key progression (GKP) primitive (\cref{sc:SSF}) is instantiated in the
GRaPPA construction, of which the pseudocode is provided in~\cite{GKP}.
We implement GRaPPA in TypeScript, similarly to the DKR primitive.
The code can be found in \texttt{ssf-client/src/protocol/group-key-progression/},
including the unit tests in \texttt{test/grappa.test.ts}.

\paragraph{Types Design}
As in the other primitive implementations,
we provide the TS types for the GKP entities used 
in the GRaPPA construction (\cref{sc:gkp-scheme}) inside the \texttt{gkp.ts} file:
\begin{itemize}
    \item We model the state of both members and admins as TS interfaces. 
    TS interfaces are extendable, so we factor out the common
    properties to keep our code DRY in the type definitions.\footnote{DRY stands for Don't Repeat Yourself, a software development principle that aims to reduce repetitions in code.}
    Having shared properties factored out also make them always
    available to the type checker while writing common code paths for both members and admins.
    In the \texttt{BaseState}, the \texttt{cgkaMemberGroupId} property stores the unique identifier of the group to which all members belong to. We will use the unique folder id returned by the SSF Gateway server at folder creation (\cref{sc:ssf-proxy-server}). This is represented as an opaque \texttt{Uint8Array}. 
    The interface \texttt{MemberState} extending the \texttt{BaseState} adds:
    \begin{itemize}
        \item A \texttt{role} property, a literal string \texttt{member}.
        \item An \texttt{interval} property, of type \texttt{DoubleChainsInterval}, storing the interval state to which the member has been given access to (\cref{sc:DKR-implementation}).
    \end{itemize}
    The admin state modelled in \texttt{AdminState}, also extending \texttt{BaseState}, adds:
    \begin{itemize}
        \item A \texttt{role} property, a literal string \texttt{admin}.
        \item A \texttt{cgkaAdminGroupId} property, of type \texttt{Uint8Array}, storing the unique identifier of the group to which all admins belong to. We will deterministically construct this identifier concatenating a prefix \texttt{ADMIN-} with the \texttt{cgkaMemberGroupId} 
        \item A \texttt{dkr} property, holding an instance of \texttt{KaPPA} (\cref{sc:DKR-implementation}).
    \end{itemize}
    Finally, a type \texttt{ClientState} represents either a member or an admin state, and will be used
    inside the \texttt{GRaPPA} class.
    \footnote{As the \texttt{MemberState} and \texttt{AdminState} interfaces 
    extend the \texttt{BaseState} interface and they both declare 
    a \texttt{role} property, after a check for the value of role in the code,
    the compiler will be able to infer which properties are available in the object, 
    providing type safety and avoiding runtime errors. This type system feature is called discriminated union.~\cite{TSDisciminatedUnions}.}
    
    \item We define the types corresponding to each command from GKP (\cref{sc:gkp-scheme}),
    and we make a clear distinction between commands that target another user and commands that
    target the user executing them.\footnote{To avoid bloating the description, we forward to the code for the details. We again make use of discriminated union modelling each command with its type.} 
    Note that the arguments to \texttt{ExecCtrl} all take
    either zero or one argument, which is the target user identifier.
    The type \texttt{ControlCommand} is a union of all the possible types of commands.

    \item The \texttt{Proposal} type partially represents the messages exchanged in GRaPPA 
    to carry out the various commands. The name of this type reflects the fact
    that a proposal could be rejected by the system (\cref{sc:state-sync-rollbacks}).
    Each proposal embeds a property \texttt{cmd}
    specifying the exact type of the command and its arguments if any.
    Then, depending on the \texttt{cmd}, the proposal will additionally contain:
    \begin{itemize}
        \item \texttt{memberControlMsg}, a \texttt{Uint8Array} containing the encrypted member CGKA control message (all). Corresponds to the variable $T_M$ in the pseudocode.
        \item \texttt{memberWelcomeMsg}, a \texttt{Uint8Array} containing the CGKA welcome message for a new joining member (\texttt{Add}). Corresponds to variable $W_M$ in the pseudocode.
        \item \texttt{adminControlMsg}, a \texttt{Uint8Array} containing the encrypted admin CGKA control message (\texttt{AddAdm}, \texttt{RemAdm}, \texttt{UpdAdm}, \texttt{Rem}, \texttt{RotKeys}). Corresponds to the variable $T_A$ in the pseudocode.
        \item \texttt{adminWelcomeMsg}, a \texttt{Uint8Array} containing the CGKA welcome message for a member who is granted admin privileges (\texttt{AddAdm}). Corresponds to variable $W_A$ in the pseudocode.
    \end{itemize}

    \item The \texttt{AcceptedProposal} type is used to model a \texttt{Proposal}
    that has been accepted by the delivery service,
    and is therefore considered applied in all clients' states, meaning that
    the changes specified in the proposal fields, which are CGKA proposals (\cref{sc:CGKA}),
    have been committed inside the respective CGKA state (\cref{sc:js-bindings-for-mls}, \cref{sc:state-sync-rollbacks}).
    An \texttt{AcceptedProposal} is a \texttt{Proposal} with the additional \texttt{messageId} field of type \texttt{number},
    which is a unique global identifier of the message in the system, and is added by the DS if the proposal is accepted.

    \item \texttt{ApplicationMessageForPendingProposals} is the type modelling a partial message exchanged in GRaPPA
    to carry out the various commands. The name of this type reflects that the partial message is composed of 
    MLS application messages (\cref{sc:MLS}). This message is sent to the delivery service
    with references to the \texttt{AcceptedProposal} that are completed by this application
    message, meaning that they constitute together the full control message (or welcome message) as described in
    GRaPPA~\cite{GKP}. The references are a list of \texttt{messageId}, i.e.\ an array of \texttt{number}.

\end{itemize}

The reader should notice the difference between the modelling of message exchange
in the GRaPPA construction and implementation. This
difference can be seen already in the types, where the \texttt{Proposal} type is only 
partially modelling the return values from the operations of 
GRaPPA pseudocode, representing only the CGKA control messages. 
As described in \cref{sc:js-bindings-for-mls},
the library mls-rs does not allow a stashed change in the internal 
cryptographic group secret of CGKA that has not been fully applied
to be used to encrypt an application message,
until the committed stashed proposals are applied,
i.e.\ the state is advanced and there is no possibility to rollback.
Thus, we need to split proposals and commits from what we call 
GRaPPA application messages (\cref{sc:state-sync-rollbacks}, \cref{ssc:delivery-service}).

\paragraph{Object-oriented Design}
Inside \texttt{gkp.ts}, we also provide the interface exposing the functionalities 
of the \texttt{GKP} primitive (\cref{sc:gkp-scheme}). 
The procedures \texttt{JoinCtrl} and \texttt{InitUser} are however not instance
methods of the class, but static methods. This is because one is used to create
a GRaPPA client to actively create a new group, while the other is used to
join an existing group to which the target user is invited. 

The TS class \texttt{GRaPPA} implements GRaPPA and maintains the state internally, similarly to the \texttt{KaPPA} class.
The class holds the following state:
\begin{itemize}
    \item \texttt{uid}: the \texttt{Uint8Array} representing the unique identifier of the user, this corresponds to the \texttt{user\_email} in the public certificate from the PKI and with which the user registers itself to the SSF Gateway.
    \item \texttt{middleware}: an instance of the \texttt{GKPClientMiddleware} class, which is used to abstract away calls to the backend. It is injected in the constructor of the class, it is useful mostly for testing purposes (\cref{ssc:GKP-client-middleware}).
    \item \texttt{state}: an object of \texttt{ClientState} type, which is either a \texttt{MemberState} or an \texttt{AdminState} depending on the role of the user in the group. As this field depends on the user being part of a group, it is \texttt{undefined} until either \texttt{JoinCtrl} or \texttt{Create} are successfully executed. 
\end{itemize}

Similarly to what is happening in \texttt{KaPPA}, \texttt{GRaPPA} methods changing the state
do not return the new state, but instead apply the changes to the internal one.
Each method, when executed successfully, will also try to serialize the state to persistent storage before completing (\cref{ssc:GKP-persistent-storage}).


\subsection{GKP Client Middleware}\label{ssc:GKP-client-middleware}
One of the major differences between GRaPPA and the implementation in \texttt{GRaPPA}
is the real interaction with a server, which is interleaved in the protocol execution (\cref{sc:state-sync-rollbacks}).

We abstract away this interaction in \texttt{GKPClientMiddleware} type, also
defined in \texttt{gkp.ts}. The middleware is used to handle serialization of
\texttt{Proposal} and \texttt{ApplicationMessageForPendingProposals} objects
to byte arrays using CBOR and to send them to the server using the autogenerated
APIs from the OpenAPI specification of the SSF Gateway server (\cref{sc:ssf-proxy-server}).
This enhances the readability of the \texttt{GRaPPA} code, as well as maintainability,
as the code for the communication with the server is kept in a single place.
We implement the middleware in the \texttt{dsMiddleware.ts} file.
A mock implementation is provided in the \texttt{inMemoryMiddleware.ts} file, 
and it is used to test the \texttt{GRaPPA} class in isolation from the server.
We highlight that the serialization code for the message is kept separate
from the implementation of GRaPPA, as the \texttt{GRaPPA} class should ideally only
deal with cryptographic operations.

\subsection{GKP Persistent Storage Abstraction}\label{ssc:GKP-persistent-storage}
Another difference between the GRaPPA construction and its implementation
is the need to persist the state of the client
to ensure that a change is not lost in case of a crash
or any other event that would wipe out the volatile memory.
This, similar to database transactions, has to be done
in a precise way, to ensure the possibility of recovering a
consistent state of the client in case of errors (see also \cref{sc:state-sync-rollbacks}). 

As in \cref{ssc:GKP-client-middleware} we abstract away the interaction
with the persistent storage layer to keep our
\texttt{GRaPPA} class clean and focused on the cryptographic operations.
The persistency layer is implemented in the \texttt{storage.ts} file.
A TS interface \texttt{GKPStorage} together with the types expressing
the format of the serialized state are provided. 
For the serialization and
deserialization we first recursively operate on each
internal complex property, like \texttt{dkr}, deferring to
the specific serialization and deserialization methods of the
respective classes. Then we serialize the state to a byte array
using CBOR. In the Node.js CLI client, we store
the serialized state in a file, while in the browser client
we would need an Indexed DB~\cite{IndexedDBAPI} compatible implementation of the
aforementioned \texttt{GKPStorage} interface, which was left
as future work.

Remind that although we are able to serialize
the state kept in the \texttt{GRaPPA} class,
the state of the MLS client running in Wasm is not persisted
in long term storage.
Although this is a blocker for the deployability of the system,
we still structure the code in such a way that this part of
the system is easy to extend in the future, allowing us to
have a fully-functional system.

\subsection{State Synchronization and Transactions}\label{sc:state-sync-rollbacks}

Global state synchronization and state rollbacks are a crucial part of the \texttt{GRaPPA} implementation:
we need to maintain the consistency of the group state across all clients, which is 
key for a client own ability to continue to operate and access the folder in case of
errors or rejection of a proposal by the DS.

Ideally, the middleware calls should not be interleaved with the cryptographic operations.
We would like to have a unique GRaPPA control message, including both parts we model in our types, 
to be sent to the server
at once and then delivered to all recipients.
However, as we have seen in \cref{sc:js-bindings-for-mls},
since we are calling the mls-rs library across process boundaries, i.e.\ from JS
runtime into Wasm runtime, we cannot avoid writing the changes to the state
in the persistent storage.
Further, the DS, hence the SSF Gateway server, needs to ensure that messages are delivered in
a global order to all clients, and has the responsibility to 
keep the client synchronized.

Let's consider the case where a client starts executing a \texttt{ExecCtrl} operation:
the client needs first to check if the operation is accepted by the DS,
and then apply the changes to the state. The only trusted source of truth
on the synchronization status of the system among clients of a folder is
the DS, as its DB is acting as a global transactionally consistent state.
Now if the client proposing a change to the group state tries to send it
to the DS, and if the DS rejects the proposal, the client should roll back the changes as well.
Since there is no way to roll back the changes with the in-memory storage provider
of the mls-rs library, therefore we need to split the GRaPPA control messages
in two parts, as already explained when describing the types in \cref{sc:GRaPPA-implementation}
and the DS API in \cref{ssc:delivery-service}.
Now that we have motivated this choice, the GRaPPA \texttt{ExecCtrl} protocol is modified as follows:
\begin{enumerate}
    \item Depending on the operation, the client proposes and commits the changes internally to the CGKA group state, some of these operations involve both member and admin CGKA groups. The pending commit(s) is (are) written to the in-memory storage (inside Wasm).
    \item The client sends the first part of the GRaPPA control message (of type \texttt{Proposal}) to the server, through the \texttt{POST} proposal endpoint. In case the operation creating the proposal is the addition of a new member to the folder, use the v2 \texttt{PATCH} folders endpoint (\cref{ssc:delivery-service}).
    \item The DS processes the \texttt{POST} proposal request (or v2 \texttt{PATCH} folders request) and answers according to whether the pending message queue for the sender in the targeted folder is empty or not:
    \begin{itemize}
        \item accepts the message and send back all the \texttt{messageId} created by the DB to the client;
        \item rejects the message and sends back an error to the client.
    \end{itemize}
    \item In case an error is received, the client discards the pending commit(s), thus rolling back the changes to the CGKA group state(s). The client will then try to sync the outdated local state, by fetching the latest GRaPPA control messages from the DS until \texttt{Not Found} error is returned, and returns an error to the user instructing to re-try the operation.
    \item If the \texttt{POST} proposal call is successful, the client applies the changes to the state. 
    \item Then it computes the application message and encrypts it through a call to the JS bindings. At this point, the encryption will be performed with the updated CGKA epoch secret. 
    \item Send the second part of the GRaPPA control message to the DS. Retry with an exponential backoff in case of errors.
    \item Finalise the operation, by also serializing and persist the eventual changes to the internal \texttt{dkr} object.
\end{enumerate}

\section{SSF Implementation: File Encryption Management}\label{sc:ssf-file-encryption}

In the SSF system, file encryption is performed by the client before
uploading the file to the server (\cref{sc:SSF-scheme}).
The client offloads the storage of the file keys and file metadata
to the cloud as in the baseline (\cref{sc:metadata-synchronization}).
We call ``private data'' the encryption key and metadata of each file.
Each folder thus has an associated metadata file, and all private data
is encrypted in inside the metadata file.

The structure of the metadata file is different from the one
described for the baseline, as here we are using GRaPPA epoch
keys to encrypt the private data by epoch.
The type \texttt{Metadata} in \texttt{ssf-client/src/protocol/ssf.ts}
models the content of the file. 
The metadata file is structured as follows:
\begin{itemize}
    \item \texttt{epochByFileId}: a map from file id to epoch.
    \item \texttt{fileMetadatasByEpoch}: A map from epoch to an internal map from file id to the respective file's encrypted private data.
\end{itemize}

Each file's private data is encrypted separately to allow for
faster encryption and decryption when reading or writing 
one single file.
The encryption is using the current epoch key obtained from
\texttt{GRaPPA} \texttt{GetKey} method. We use AES-GCM
symmetric encryption, adding the file id, which is known to
the server, as additional authenticated data.

To write a file to the server, the client will follow the same
synchronization protocol as described in \cref{sc:ssf-file-changes-sync}, 
and call the same SSF Gateway endpoint.
To read a file from the server given a file id, the client will
first fetch the metadata file from the server.
Then it will search in constant time through \texttt{epochByFileId} 
in which epoch the file was uploaded.
Finally, it will follow the references in the \texttt{fileMetadatasByEpoch}
to retrieve the encrypted private data of the file and decrypt it.

An alternative solution we have considered, but discarded in this
first iteration, is to split the metadata file into multiple files,
one per GRaPPA epoch.
The structure of the file itself would be simpler,
as the epoch would be implicit in the file name.
However, we would need to store separately an additional metadata header
file, to keep track of which epoch is encrypting which file's private data,
corresponding to the \texttt{epochByFileId} map.
Splitting the metadata file would allow for faster concurrent
reads and sharding. Only the metadata corresponding
to the current epoch would be modified.
We would also reduce bandwidth consumption and reduce the cost of cloud storage usage.
To implement this solution, we would also need changes to the
SSF Gateway server. We note that the synchronization protocol to
upload a new file to the cloud would remain the same,
where the metadata header file would be used to keep track
of the last version.
This is an optimisation for long-lived folders with many 
key rotations and group membership updates,
therefore we left it out of the MVP.

\section{Web Crypto API: non-standard behaviours}\label{sc:Web-Crypto-API-implementations:-non-standard-behaviours}

In this section, we describe a gap in the Web Crypto API specification and
implementation which creates a portability issue affecting the mls-rs library
and, in general, the deployability of advanced cryptographic primitives
in the browser environment.

AWS mls-rs library implementors made the library available to Wasm compilation
targets. The support is experimental and involves for now only the implementation
of the cryptographic operations needed to execute MLS on top of the Web Crypto API.
\footnote{For practitioners: the library abstracts the underlying cryptographic code in Rust traits which are implemented for different runtime environments/cryptographic providers. The one used for the Wasm target is implemented in the crate \texttt{mls-rs-crypto-webcrypto}.}
While looking at the history commits of the library, we found out that
apparently the support for Wasm builds was added only for Chrome runtime.
Indeed, integration tests are available
which prove the compatibility and avoid regressions in Chrome.
However, as seen in \cref{ch:setup} we use Node.js to develop our client
CLI to avoid dealing with a UI in our first implementation.
Thanks to our multi-runtime setup, we discovered the portability issue
hidden in the library and the implementations of the Web Crypto API.

When using elliptic-curve crypto primitives, the mls-rs library is 
calculating the public key from the bytes of a DER-encoded~\cite{Kaliski2002ALG} private key by passing the bytes of the key to the Web Crypto API \texttt{importKey}
call~\cite{WebCryptoAPIImportKey}. In Chrome, the library BoringSSL~\cite{BoringSSL} underlying the Web Crypto API implementation, calculates
the public part of the key which is appended to the bytes provided in input.
However, this behaviour is non-standard and is not supported in other
major Web Crypto API implementations:
\begin{itemize}
    \item In Node.js, the \texttt{importKey} call will just import the key into a \texttt{CryptoKey} JS object, without any error~\cite{WebCryptoAPICryptoKey}. However, the bytes, corresponding to the public key field, are not modified. The mls-rs library has a check to detect the missing public key part and will then throw a runtime error. No documentation is provided on this issue, and we discovered it by debugging the library compiled in Wasm, which is not a trivial task. In the end, we resorted to reading the source code. Although the documentation states that the cryptographic support is only experimental, it would be beneficial to add more details on this problem.
    \item In Safari, the \texttt{importKey} call will throw an error.
    \item In Firefox, the \texttt{importKey} call will throw an error.
\end{itemize}

The only case we could develop a workaround for is Node.js. Since in this runtime
the public key is not calculated but at the same time the operation does not fail,
after importing the private key we can force the calculation of the public key with an additional
export/import operation to \texttt{jwk}~\cite{JsonWebKey} format back to \texttt{pkcs8}~\cite{rfc5958}.
Indeed, by performing this \texttt{exportKey} call to produce the \texttt{jwk}
representation of the key, the curves coordinates $x$ and $y$ are calculated.
Therefore, importing back the key in \texttt{pkcs8} format will produce the correct
public key. The code is published in a pull request to the mls-rs library~\cite{AWSNodeJSCodeContributions}.

Note that the mls-rs uses the Web Crypto API in a way conforming to the
specification, as the definition of the DER-encoded private key does not
restrict to the usage of elliptic curves private key info which must 
contain the public key field, but only should~\cite{rfc5915}:
\begin{quote}
    publicKey contains the elliptic curve public key associated with
    the private key in question.  The format of the public key is
    specified in Section 2.2 of [RFC5480].  Though the ASN.1 indicates
    publicKey is OPTIONAL, implementations that conform to this
    document SHOULD always include the publicKey field.
\end{quote}

Notice also that mls-rs implementation use case seems legitimate,
as the public key calculation would require non-constant time code for elliptic curve manipulation
to be executed to compute the public key portion from the secret portion of the key. 
Therefore, we would propose
to change the Web Crypto API \texttt{importKey} specification to
explicitly require the public key calculation, so that all
implementations are aligned with the same behaviour. In this way,
the API would support more advanced cryptographic operations on top
of the existing support for elliptic curves cryptography, such as the ones
required while trying to compute a key pair from a symmetric private key.

\section{mls-rs: Wasm build Enhancements}\label{sc:MLS-enhancements}

While working with the MLS library we found out that the Wasm target
is not fully production-ready:
\begin{itemize}
    \item The library is not fully portable to all major browsers (\cref{sc:Web-Crypto-API-implementations:-non-standard-behaviours}).
    \item Node.js is not supported. We add support for the compiled Wasm to run in Node.js by adding bindings to the Web Crypto API implementation from \texttt{node:crypto} module of Node.js~\cite{NodeJsWebCryptoAPI}. We adapt the existing JS inline code included in the library to be compatible with the JS module resolution mechanism and syntax of Node.js~\cite{AWSNodeJSCodeContributions}.\footnote{For practitioners: JS modules evolved over the years, Node.js is using CommonJS modules, which are the standard outside browser runtimes. These modules are also used in the npm ecosystem (the package manager for JS). The syntax of CommonJS is using plain JS objects and functions (require) to perform imports and exports. Browsers use currently ES modules, which were added in 2015 to the JS standard and widely adopted in 2020. ES modules use the \texttt{import} and \texttt{export} keywords to perform the same operations.}
    \item The client state is not persisted and is only kept in memory for now. We propose a strategy to persist the client-side state in browsers using the Indexed Database API~\cite{MlsRsWebStorageProvider, IndexedDBAPI}.
    \item The X.509 certificates are not supported to manage identities of the users. This imposes the usage of Basic Credentials, which lowers the security of the implementation as those credentials are not validated while performing group operations. We propose to add support for X.509 certificates in the library~\cite{MlsRsX509Certificates}. Our proposal aims to support the use case where the CA certificate is embedded in the client code or where the CA certificate is loaded at runtime through a different mechanism. This use case could be applied for example in a corporate environment.
\end{itemize}

A further possible enhancement, which would help in
using the library to build advanced protocols such as GRaPPA,
is to add support to encrypt application messages
using the epoch secret deriving from a pending commit (\cref{sc:GRaPPA-implementation}).

\section{Correcting the primitives}\label{sc:correcting-primitives}

During the implementation and testing of the GRaPPA constructions,
we found issues in the pseudocode descriptions as provided
in the original manuscript of \cite{GKP}.
In \cref{sc:GRaPPA-bugs} we provide the full explanations of the bugs, 
and the corrections we applied. 
We also discuss the implications of our corrections. 

The bugs we highlight in the following sections
derive from the underlying primitive not being expressive and
precise enough to model the interactions between distributed clients
playing either the role of admin or member.
In the generalized DKR primitive, the read/write (admin) 
and read-only (member) capabilities (see \cref{sc:DKR-implementation})
are not modelled at all in the mathematical definitions of the operations.
We explore the implications and attempt to correct
the primitive in~\cref{sc:DKR-enhancements}.

\subsection{GRaPPA bugs}\label{sc:GRaPPA-bugs}

While implementing the GRaPPA construction, we found two bugs.
The bugs are both leading to the same issue: the state of the
sender admin and receiving admins can go out of sync after a certain
sequence of operations is executed.

\paragraph{Bug 1:} The admin operations \texttt{Rem}, \texttt{RemAdm} and
\texttt{RotKeys} require to advance the DKR state from the current epoch
$e_{c}$
to the next epoch $e_{r}$ and add blocks, meaning that new chains are created, 
either a forward, backward or both.
When an admin receives messages containing such operations and related
state updates, it will execute the \texttt{ProcCtrlAdmin} procedure.
In this procedure, the receiver admin will reconcile the local DKR state
with the state update received from the sender admin.
We note that the receiving admin at the time of receiving the message
has its local state synced up to epoch $e_{c}$.
While the \texttt{RemAdm} and \texttt{RotKeys} operations are sending 
the full DKR state, \texttt{Rem} is just sending an extension of the current state. 
This extension was however ignored by the receiving admin, 
which would just get out of sync. We adjusted the \texttt{ProcCtrlAdmin}
procedure to also handle the case where an extension is sent,
and process it. To this end, the receiving admin calculates the
interval of its complete local DKR state, from epoch 0 to the current
local epoch $e_c$. Then it will extend the interval with the extension
received from the sender admin, thus having access to the state up to
the epoch $e_r$. Finally, the resulting interval is used to override the
local DKR state of the admin. 
We notice that the sender admin might still
have access to a bigger state than the receivers,
as receiving admins are forced to shorten their last backward chain
to epoch $e_r$ to correctly process the extension.
However, the admin group will still be able to progress forward
to a new epoch $e_{r+1}$, and maintain all the local views of the DKR state
synchronized. We distinguish two cases, depending on which admin is
executing the next operation:

\begin{itemize}
    \item If the next operation is performed by the same sender admin,
the state of all other admins will be updated with a new extension or with
a full state update, depending on the operation.

\item When the next operation is performed by a different sender admin,
then a new backward chain will be started. The new initial element will 
be sent in the extension and the admin still holding the larger state
will also shrink its state to the epoch $e_r$. 
Then the extension will be processed, so the new initial element
will be added to start a new backward chain, and the state will be
synchronized up to epoch $e_{r+1}$. All other receiving admins will
also process the extension and will be able to progress to the new epoch,
following the same procedure, with the detail that the last backward chain
has already been shortened to epoch $e_r$ in the previous operation.

\end{itemize}

The implications of this na\"ive approach, where admins use the same
mechanism of members to extend their local state, is that in case admins
alternate in performing operations, at each operation a new backward chain
will be started, resulting in an average space complexity of $O(n)$ 
for the initial elements.
Another possible solution would be to send the DKR state also
in case of a \texttt{Rem} operation. This would instead 
increase the bandwidth usage.

\paragraph{Bug 2:} The second bug is caused by an optimization,
with which admins spare on the bandwidth usage while sending the new DKR state
from a sender admin to receiver admins.
An admin performing the \texttt{Add}, \texttt{UpdAdm} or \texttt{AddAdm} operations needs to advance the global epoch of the group and only release a new forward chain element.
This is performed by calling the \texttt{progress} procedure on the current
DKR state. No block is needed, as we only need to disallow
the new member of the group from generating keys corresponding
to epochs before the one in which he joined. Therefore, the receiving
admins would need only to call progress on their local copy of the DKR
global state and should be able to calculate the new forward chain element.
However, recalling the generalisation of the DKR construction,
we might be exactly at the epoch corresponding to a maximum chain length,
either for the latest forward chain, the latest backward chain or both.
In this case, the sender admin would need to generate a new random initial 
element for the new chain(s). If also the receivers randomly sample new 
initial elements in their local state, all admins would be out of sync.
The fix we have implemented is to also send an extension in those cases,
with the same implications as seen above for the other bug.


\subsection{DKR enhancements}\label{sc:DKR-enhancements}

In the discussion on the implications of the adjustments to GRaPPA in ~\cref{sc:GRaPPA-bugs},
we have seen how the admins can get out of sync.
We highlight that the current na\"ive solution still
has some drawbacks, as the state of all admins is not
fully synchronized, where an admin maintains a larger state
comprising all the latest backward chain elements other admins
lost access to.

The key point we want to stress is that the DKR primitive
does not capture the distinction between a full DKR state,
with read/write access, and a partial DKR state, with read-only access
together with the associated operations.
We claim that this distinction is crucial to understanding the discrepancy existing between the state of two clients with different capabilities,
namely admins and members, and provide a more detailed specification
of the two compared to the one provided in the original manuscript of \cite{GKP} (\cref{sc:background-generalised-DKR}):

\begin{itemize}
    \item A DKR full state, which we will simply call DKR state,
    comprises: 
    the current epoch $e_{max}$, 
    the complete list of backward chains
    and the complete list of forward chains,
    from epoch $0$ to epoch $e_{max}$ and 
    a parameter $N$ indicating the maximum length of a chain.
    Each chain is stored from its initial element. 
    In particular, the latest backward chain
    is stored from the initial element of the element sequence order,
    which corresponds to the latest possible element which will be released.
    The state thus contains all the elements to derive the state up to
    the current $e_{max}$ epoch and possibly beyond. 
    In case the latest forward chain is not fully used, i.e.,
    the latest $N$-element was not yet released, and the backward
    chain is not also fully used, i.e., the element
    corresponding to $e_{max}$ is not the initial element of the chain,
    it is possible to compute new elements of both
    chains, which can be used to derive keys for the next epochs.
    \item An interval state is a subset of the DKR state, which
    allows deriving keys for the epochs represented in the interval.
    It comprises: the epoch interval with the starting ($e_{left}$) and ending ($e_{right}$) 
    epochs, with the constraint $0 \leq e_{left} \leq e_{right} \leq e_{max}$,
    where $e_{max}$ is the current epoch of the DKR state from which the interval
    state is extracted. Also, it includes the slices of the backward
    and forward chains corresponding to the epochs in the interval.
    More precisely, the forward (respectively backward) chain from which the
    element for the epoch $e_{left}$ (respectively $e_{right}$) can be derived
    is shrunk to that element, disallowing the derivation of any key outside the epoch interval.
\end{itemize}

When we consider a shared DKR state among multiple clients,
as in the implementation of GRaPPA, it becomes clear
that an operation to extend a copy of the DKR state is
missing in the scheme.
We therefore propose to add the following operation to the DKR
primitive, of which we provide an informal description:

\begin{itemize}
    \item \texttt{CreateFExt($st$, $l$)}, on input the DKR state $st$ and an epoch $l$,
    returns a full-extension $fext$ or error. The full-extension is an interval state
    starting at epoch $l$ and ending at the current $e_{max}$ epoch of the DKR state
    $st$, similarly to an extension. However, in a full extension, we do not 
    specify the ending epoch, as the latest backward chain included in the interval
    could give access to state beyond the current epoch. Practically speaking,
    the full extension is an extension where the latest backward chain is not shrunk.
    We further highlight that full extensions always comprise the latest current epoch,
    as they serve the specific purpose of extending the state of a replica of the DKR state.
    Also, a full extension contains the current epoch $e_{max}$.

    \item \texttt{ProcessFExt($st$, $fext$)}, on input the DKR state $st$ and a full-extension $fext$,
    returns the updated DKR state $st'$ or error. The procedure processes the full-extension
    $fext$ on the provided $st$ as the existing \texttt{ProcExt} operation processes an extension $ext$
    on a provided interval state $int$, with the additional
    operation of setting the state current epoch $st.e_{max}$ to the current epoch contained
    in the full-extension $fext.e_{max}$.
    
\end{itemize}

\paragraph{Solving the Bugs in GRaPPA}
Equipped with the new full-extension entity, its creation and processing operations,
the enhanced DKR primitive can be now used in the GKP instantiation
GRaPPA to solve the bugs from \cref{sc:GRaPPA-bugs}.
Instead of applying the na\"ive solution, where the admins extend their local state as members, or they always send each other the full state, the admins can now send full extensions among them for all operations that require progressing the DKR state,
i.e.\ all admin operations.
This will allow them to keep their local copy of
the DKR state synchronized, with an optimal constant bandwidth 
consumption, thus serving the purpose of correcting the 
protocol and clarify how to minimize bandwidth consumption.

%\paragraph{Optimise Transactional Storage of DKR}

%With the new operations \texttt{CreateFExt} and \texttt{ProcessFExt},
%we can optimise the transaction storage of the DKR state in a client.
%This use case is not only present inside GRaPPA, but in any practical
%usage of DKR alone. However, a related discussion can be found in \cref{sc:state-sync-rollbacks}.

%Let's imagine the case in which a client 

\chapter{Engineering Gaps}\label{ch:gaps}

This chapter summarise the engineering gaps (\cref{sc:summary-of-contributions})
we encountered during the implementation of the system,
both for the baseline and the SSF scheme implementation.
The engineering gaps highlight the practical issues
and differences between the theoretical model and 
the real-world implementation.

For each gap, we refer to the relevant sections to guide the reader
to the details given in the overall content of this document.
Looking in retrospective, these gaps are main starting
points for further reflections on the current
state of real-world cryptography and lesson learnt.

We close the chapter by presenting the future work (\cref{sc:future-work}).

\section{On the Cryptographic Ecosystem of the Browser}\label{sc:gap-webcrypto-api}

The Web Crypto API (\cref{sc:webcrypto-api})
has been cited during this work
multiple times. 
The API has many limitations 
(\cref{sc:baseline-protocol}, \cref{sc:ssf-sskg}, \cref{sc:ssf-double-prf}, \cref{sc:Web-Crypto-API-implementations:-non-standard-behaviours}),
when compared to the cryptographic libraries available
in other languages and runtimes.
Writing the same code just for a Desktop application
would have been way easier.
Although the API design tries to avoid developers from
wrongly using cryptographic primitives, it lacks
the flexibility to allow advanced use cases.

We think this API should
be enhanced to provide more building blocks
for advanced use cases, to allow high quality
cryptography to be shipped in the browser to
end users. We claim this is not a small issue,
as most of today services are nowadays accessible
from the browser and security is a raising concern.
While the cryptographic research is advancing and new
schemes are proposed, those advancements remain hard to
deploy in browser runtimes, where they can 
protect millions of users.

\section{On Code Portability and Heterogeneous Device Capabilities}\label{sc:gap-code-portability}

In this work, we have also tried to showcase the
portability of the cryptographic primitives
between browsers and desktop runtimes.
We found Node.js to be an alternative runtime supporting
the same Cryptographic APIs as the browser, although
with some differences (\cref{sc:Web-Crypto-API-implementations:-non-standard-behaviours}).

We have used WebAssembly to have access to
library code otherwise unavailable
in the browser, like the mls-rs library
(\cref{sc:CGKA-implementations}). 

WebAssembly is a new raising technology, which could
allow for more portability and reuse of the code on
heterogeneous platforms and devices. 
If cryptographic primitives
are integrated in the WebAssembly runtime,
through dedicated instructions guaranteeing
certain requirements for the security of cryptographic
code (\cref{sc:abstract-to-real}, (\cref{sc:browser-runtimes}, \cref{sc:webcrypto-api})),
code portability issues could be mitigated.
While in the theoretical model the primitives are
mathematical objects with defined properties,
these properties might not hold at runtime in the
implementation, especially across different runtime,
implementations and devices.
Wasm could be an abstraction layer providing the
same cryptographic primitives across different
runtimes.

\section{On the Design and Implementation Efforts of Cryptography}\label{sc:gap-crypto-primitives-design-implementation}

Designing cryptographic primitives is a complex task,
which involves proving the security of the scheme. The prototyping phase, where ideas are gathered, 
and the scheme is designed is usually done through 
discussions and whiteboard sessions.

The implementation of such ideas can instead take months.
It requires setting up all the development environment,
required dependencies (\cref{ch:setup}), and writing thousands of lines of code.
The final version of this project 
is more than 18600 lines of code, excluding code that
is generated through our tooling (\cref{sc:PKI}, \cref{sc:client-overview}), 
accounting for around other 6000 lines,
for a total of more than 25000 lines of code.
The complete git repository, including
also package manager files and documentation, 
is more than 53000 lines.
When we compare with the pseudocode provided in the manuscripts,
it is clear that the implementation effort, especially
when targeting a real-world scenario, requires much
more effort. 

Also note that every change requires
changing many lines of code. Many parts of the code
were rewritten during the implementation to try out
different approaches or because changes were made to the
design of the primitives (see also \cref{sc:collaboration-crypto-se}). 
Ability to prototype with
code requires a lot of engineering effort and
expertise. Normally indeed, only proof-of-concept
implementations are created in research,
distancing the design of new primitives from their
real-world applications.

Primitives targeting complex systems with complex interactions
between different components, like the GKP scheme (\cref{sc:gkp-scheme}), are the
ones most likely to suffer from this gap between design
and implementation.

\subsection{Implementing Assumptions}\label{sc:implementing-assumptions}

An important aspect of the effort required to implement
cryptographic primitives is the need to implement also
all the dependencies, which are normally assumed to exist
in the mathematical model. For example, a PKI is normally
assumed to handle identities, but we needed to construct a server for it (\cref{sc:PKI}).
Another such example is the delivery service for MLS and
GKP, which is just described by its properties in both schemes,
requiring a server to be implemented as well (\cref{ssc:delivery-service}).

\section{On the Non-Cryptographic Guarantees}\label{sc:gap-non-crypto-guarantees}

While the security proofs of GKP~\cite{GKP} consider availability out of
scope, the real-world expectation for the SSF scheme
is that the system should be
available, and the persisted data should be present in 
a folder until deletion is ordered by a user with
legitimate access.

In the implementation this is all handled by the SSF Proxy server
(\cref{sc:ssf-proxy-server}).
The server assure authentication and access control
to the data, thus for example protecting from actors 
external to the shared folder to delete its files (\cref{sc:cloud-storage-access-and-billing}). 

\section{On Client Execution and State Management}\label{sc:gap-execution-multi-tenancy-state-management}

The description of the operations in the cryptographic
constructions are usually abstracting away many details,
in particular related to the state management.
The pseudocode of D[F, S] and GRaPPA~\cite{GKP} (\cref{sc:background-generalised-DKR}, \cref{sc:gkp-scheme}) 
does not specify how the state of the client is persisted 
and how the client can recover from a crash or transient error.
The procedures of the protocols are usually described as
mathematical functions, thus, not holding any state across
invocations.

In our implementation we have to deal with the above issues,
to guarantee that the client state is always correct,
so that if an operation is started is either completed
or rolled back to the previous correct state.
Borrowing from database terminology, we say that the client
state must be handled ``transactionally'', to ensure that
a client will always be able to continue updating its state
correctly and consistently.

We remind the reader that some stateful cryptographic implementations do not
allow rolling back to a previous state after an operation
that changes the internal state has been executed (\cref{sc:js-bindings-for-mls}).
This creates issues that can only be solved by changing
the details of how the procedures are executed (\cref{sc:state-sync-rollbacks}).

We highlight that the transactional state management problem
is particularly interviewed with the server synchronization
mechanism (\cref{sc:gap-synchronization-server-state}).


\section{On Synchronization and Server State}\label{sc:gap-synchronization-server-state}

Although in the scheme and related constructions the description
does specify that the server should not hold any state,
this is only true in terms of cryptographic state, i.e., for
the key material.

The SSF Gateway server provides the clients with a way to order and synchronize
their concurrent operations on the actual shared cloud storage (\cref{sc:ssf-file-changes-sync}).
We remind the reader that 
the server is used also as an MLS/GRaPPA delivery service (\cref{ssc:delivery-service}).
Therefore, in the implementation, for a correct execution of the protocols, 
we need the server
to keep track of a portion of the state of the groups, 
shared-folders and clients (\cref{ssc:delivery-service}, \cref{sc:state-sync-rollbacks}).


\section{On the Abstract Modelling of Cloud Storage}\label{sc:gap-abstract-cloud-storage}

As discussed in \cref{sc:cloud-storage} and more in details in 
\cref{scc:cloud-storage-assumptions},
\cref{sc:cloud-storage-access-and-billing}
and \cref{sc:ssf-proxy-server} the cloud storage modelling
as read/write operations on virtually infinite storage
is simplification that removes multiple practical problems. 
In reality, implementors need to
deal with the details of each cloud provider, the billing attribution
and the choice of the actual technology to use among
the different available options.
Also, a production-ready
implementation should allow for different cloud storage
providers to be used, with a simple configuration change.

The above points required the implementation of the
SSF Gateway server (\cref{sc:ssf-proxy-server}).
The Gateway acts as an abstraction layer between clients
and cloud providers, thus closing the gap
between the abstract model and the real-world providers.

\section{On Performances}\label{sc:gap-performance}

Performance is only studied in the context of the
cryptographic scheme itself, looking at the time and
space complexity of the primitives.
However, since the correct execution of the protocol
could be dependent on server components for the synchronization
of the state, like for the GKP or SSF scheme, the performance
can be greatly affected by the server implementation.

Protocols where multiple clients need to maintain
a global shared state, like the GKP scheme, can especially
suffer this problem. The complexity of the client
execution can become negligible compared to a server
becoming the bottleneck of the entire system.
The design of primitives which do not require 
global state synchronization to advance the local
client state should be preferred
for real-world applications, as they are easier 
and more efficiently scalable.

%We observe that GKP (hence also SSF) suffer from the CAP problem (\cref{sc:ssf-file-changes-sync}).\footnote{The CAP theorem states that it is impossible for a distributed system to simultaneously provide more than two out of the following three guarantees: Consistency, Availability, Partition tolerance.}
%However, this is a latent issue in the scheme,
%as availability is kept out of scope 
%(\cref{sc:gap-non-crypto-guarantees}). 


\section{On the Design of new Cryptographic Primitives: a Feedback Loop}\label{sc:collaboration-crypto-se}

Recalling the discussion of the bugs and the enhancements 
proposals in the implementation of the GRaPPA construction
(\cref{sc:GRaPPA-bugs}, \cref{sc:DKR-enhancements})
we want to stress the importance for
cryptographers to work closely with software engineers
when designing new cryptographic primitives.

First, software engineers can provide a different, more
practical, point of view to the problems the primitive
wants to address, and provide background knowledge
on the current state-of-the-art for the application
domains the primitive can be applied to.
Second, an implementation targeting a real-world scenario
can uncover many issues in the original design, which could be
overlooked in the mathematical model or by a toy implementation.
As in other software, many problems arise only when
we consider the system at scale, or if we want certain
user expectations to be met.

We think that the design of new cryptographic primitives and
the implementation of the constructions in real-world
setting should go hand in hand. 
In our case, the implementation, 
initially guided by the pseudocode 
description, has discovered bugs in the constructions which
ultimately were caused by a gap in the mathematical model.
The enhancements we are proposing are actively discussed
with the authors of the original manuscript, and we hope
to see them included in the next version.
This is a very clear
example, among many others during this work which was conducted 
with continuous back and forward discussions, 
where the implementation details entailed 
an in-depth analysis of the implications on the model and 
assumptions taken in the design of the primitive.
In the end, a feedback loop between cryptographers and engineers 
is established leading to
a better analysis of the primitive and its applications,
and a more robust and secure implementation.


\section{On the Type-Safety of Opaque Types}\label{sc:gap-type-safety-of-opaque-byte-arrays}
The attentive reader must have realised that most of the data either in the state
or in the message exchange in our implementations is represented as opaque \texttt{Uint8Array}.
We point specifically to the implementation of the cryptographic primitives:
\cref{sc:ssf-sskg}, \cref{sc:DKR-implementation}, \cref{sc:js-bindings-for-mls}, \cref{sc:GRaPPA-implementation}.

This is usually the case for cryptographic heavy software written in any language, 
and can easily cause bugs in the implementation, as the type checker
would gladly accept any \texttt{Uint8Array} passed to a function, or as object 
property and so on even if the contents are meant for a different usage. 
During the implementation we encountered this problem multiple times,
where calling a function with the wrong order of the parameter, where
multiple \texttt{Uint8Array} are passed as arguments, can cause bugs
very hard to fix without long debugging sessions. Note that it can be
really easy to make such an error, especially while writing thousands
of lines of code and, in this case, without anyone else reviewing the code.
We believe the discussion of this practical issue can be of great interest for other
engineers encountering this problem, as well as it highlights yet another
difficulty in writing and deploying correct cryptographic software.

The na\"ive solution currently adopted in the code (for time constraints reasons)
is to always use a wrapper object containing all the parameters of the function
as properties. In this way, we need to give a name to each parameter
in the function declaration as well as to the arguments at function-call
time, as the object needs to be constructed. This will make clear which
opaque byte array is passed to which argument, and clarify the intent.
With this pattern is also possible to perform runtime validation
while creating the object with the arguments if required.
We highlight that this methodology also helps in a team effort, to make
code reviews easier. Further, this solution applies as well to JavaScript,
as it does rely on plain JS objects to convey the semantic. It is also
easy to translate this technique to any other language supporting objects.
The downside however, is that at runtime an object is created, which adds 
an overhead in terms of memory allocation.

A TypeScript-specific solution which avoids the runtime overhead, is to use
so-called ``branded types''~\cite{vanderkam2019effective, goldberg2022learning}. 
With branded types, we can construct two 
different types of the same underlying \texttt{Uint8Array} only
at compile time. After the code is type-checked and transpiled down to JS,
we will just obtain the same code we would have written without branded types.
Branded types are thus type-safe opaque types. We leave an in-depth explanation
of the technique to the cited literature. We plan to refactor the code base
to use branded types to better specify the semantic of primitive types,
like \texttt{Uint8Array}, \texttt{string} (for example storing PEM certificates),
\texttt{number} (positive numbers, integers, etc.).

Type-safety is a strong requirement when writing
cryptographic software, and we want to point out that especially for cryptographic
software, the semantic of the data should be conveyed by the type,
to enhance readability, maintainability and reduce the number of
bugs in fairly complex crypto systems.
In the future, we
might hope to see the rise of new programming languages
or extensions of existing ones to enable the native usage
of opaque types with an associated semantic for the reasons 
explained above.


\section{Future Work}\label{sc:future-work}

One of the initial practical goals of this work
was to implement an MVP of our secure folder shared folder
real-world system. At the end of this thesis,
we can say we have \textbf{almost} achieved this goal.
We refer here to the SSF scheme implementation.
As we have seen in the previous chapters,
the current implementation suffers multiple issues,
which we summarise in the following points as future work:
\begin{itemize}
    \item Implement the changes to DKR and GKP after the discussion with the original authors of the primitives is finalised (\cref{sc:DKR-enhancements}).
    \item Maintain the state of the client between multiple user sessions (\cref{CLI}, \cref{sc:js-bindings-for-mls}). We remind that this includes the completetion of two separate tasks:
    \begin{itemize}
        \item Implement a browser-compatible storage layer for the mls-rs library (\cref{sc:MLS-enhancements}).
        \item Write a browser-compatible implementation of \texttt{GKPStorage} (\cref{ssc:GKP-persistent-storage}).
    \end{itemize}
    \item Enhance application message encryption capabilities of mls-rs (\cref{sc:MLS-enhancements}).
    \item Consequently, refactor the resiliency protocol in the GRaPPA implementation, by sending only one control message as in the original design (\cref{sc:state-sync-rollbacks}).
    \item Add X.509 certificate support to the MLS client (\cref{sc:MLS-enhancements}), to substitute the basic credentials with proper PKI support.
    \item Research a workaround to avoid compatibility problems between different browsers' implementations of the Web Crypto API or propose a change of the API specification to the W3C  (\cref{sc:Web-Crypto-API-implementations:-non-standard-behaviours}). We recall that some needed cryptographic operations are currently not supported in all major browsers, specifically Safari and Firefox. 
    \item Optimise the storage of files' private data (\cref{sc:ssf-file-encryption}).
    \item Partially refactor the types in the client implementation to use branded types (\cref{sc:gap-type-safety-of-opaque-byte-arrays}).
\end{itemize}

Further optimisations and enhancements can be done
to the server components, particularly we note that
the PKI server should be replaced with a real CA
implementation. The SSF Gateway server could be optimise
for performance, but this is not a priority of the
next development iteration.

Finally, continue to study the applications of the GKP scheme in
real-world scenarios, and improve the primitives.


\appendix

\chapter{Appendix}\label{ch:sample-appendix}

Here is a link to \cref{ch:sample-chapter}.

\backmatter

\bibliographystyle{plain}
\bibliography{../cryptobib/abbrev3,../cryptobib/crypto,refs}

\includepdf[pages={-}]{declaration-originality.pdf}

\end{document}
