\newif\ifdraft
\draftfalse

\documentclass[E]{BAMASA}
\usepackage{xspace}
\usepackage{natbib}
\usepackage[normalem]{ulem} %for \sout strikeout

\renewcommand{\asst}{Matilda Backendal, David Balbás, Miro Haller\\
Matteo Scarlata}

\usepackage[usenames,dvipsnames]{xcolor}

\usepackage[colorlinks=true,linkcolor=Blue,urlcolor=Blue,citecolor=Blue]{hyperref}


%
% draft-mode macros
%
\ifdraft
\newcommand{\TODO}[1]{{\color{Red}\textbf{TODO:} #1}}
\newcommand{\new}[1]{{\color{OliveGreen}#1}}
\newcommand{\old}[1]{{\color{RawSienna}\ifmmode\text{\sout{\ensuremath{#1}}}\else\sout{#1}\fi}}
\newcommand{\replace}[2]{\old{#1}\new{#2}}
\newcommand{\mb}[1]{{\color{Green}{\textbf{mb:} #1}}}
\newcommand{\nd}[1]{{\color{Orange}{\textbf{nd:} #1}}}
\fi


\begin{document}

%In the line below, use the "\MA" flag for a Master's thesis or the "\BA" flag for a Bachelor's thesis. 
\MA{April 15, 2024}{October 15, 2024}%
{Nicola Dardanis}%
{Implementing a Secure File Sharing System}


\intro
In the Internet era, data creation and collection has reached new peak levels.
More and more personal data is communicated through, stored at and analyzed by services such as messaging apps, social media and online file storage.
Concurrently, a concern about privacy for all this outsourced data has sailed up as an important topic, both legally, politically and for private individuals.
Control of what data is collected, where it is stored and how it is used is largely outside the control of the data owners.
Nonetheless, the functionality provided by these services is too beneficial for most companies and consumers to abstain from.
The consequence is that even highly sensitive data, such as for example medical records, whistle-blower material and regime-critical protester plans, end up on the servers of online application service providers.

Fortunately, cryptography has simultaneously made giant leaps forward in securing the confidentiality and authenticity of data in transit against strong adversaries.
End-to-end encryption (E2EE) is now the norm for Internet browsing (via TLS) and increasingly also for messaging (with apps such as WhatsApp and Signal being end-to-end encrypted by default).
Somewhat surprisingly, services that offer outsourced data storage, such as cloud storage and collaborative file editing platforms, still lag behind.
One of the explanations might be the complexity that arises due to the persistence of data, which makes it difficult to use ephemeral key material to achieve strong security guarantees such as forward secrecy (FS) and post-compromise security (PCS).
Another is the lack of a formal security models for even basic E2E security of outsourced data storage supporting functionality such as file sharing between users.
In particular, the number of potential end-points arising from file sharing increases the complexity of E2EE cloud storage compared to single client settings.

This complexity also exists in messaging, as showcased by the fact that protocols for secure two-party messaging (such as e.g.\ the Signal protocol) have been around for quite some time, but a protocol for E2EE group chats was only very recently standardized~\cite{rfc9420}.
The newly standardized group messaging protocol is called ``messaging layer security'' (MLS).
One of the main motivations for MLS was to make E2E security for messaging in groups of size $n$ more efficient than through the naïve construction of $n^2$ two-party channels, while still retaining the same high-security guarantees---including forward secrecy and post-compromise security---that we expect from modern secure messaging protocols.

At its core, MLS uses a primitive called continuous group key agreement (CGKA) to generate and exchange group keys.
The group key is only available to current group members, and when the group membership changes (causing a so called ``epoch'' shift), the group key is updated such that only the members of the new group get the new key.
CGKA comes in many different variants, with different syntaxes and associated functionalities.
In particular, recent CGKA schemes provide restrictions on which parties can perform certain group membership operations (such as adding and removing users);
this is known as administrated CGKA (A-CGKA)~\cite{USENIX:BalColVau23}.

In this project, we will explore the possibilities for more advanced security guarantees for file sharing systems in the E2EE setting.
In particular, we will aim to tackle the conflict between the required functionality (including persistent data access, and flexible group and access management) and strong security guarantees such as FS and PCS.
Our initial attempt at a solution, which we call the ``secure shared folder'' (SSF) scheme, combines the recent advancements of group messaging from the MLS standard with a form of key ratcheting known as key regression~\cite{NDSS:FuKamKoh06}.

In MLS, the CGKA group key is used by an application layer to provide the functionality required for a group messaging protocol.
In SSF, the CGKA group key is instead used at the top of a key hierarchy for file encryption.
Key regression is additionally embedded in the hierarchy to permit persistent access to files uploaded in earlier epochs. 
For group management, the SSF scheme relies on the security guarantees of administrated CGKA.


\desc
The main aim of the project is to implement a proof of concept of the SSF scheme.
The implementation should be conducted following the best engineering practices 
and should guarantee the security expressed in the theoretical design.
The core deliverable of the project is a minimum viable product (MVP) implementation with the main features of the scheme.
Developing the MVP additionally requires the choice of a suitable programming platform and language, and in particular, a specification of the ecosystem in which the final implementation should be supported.

The project also consists of benchmarking the SSF implementation.
Hence, a preliminary task is to develop a suitable baseline to benchmark the SSF system against.
For example, the baseline may consist of a na\"ive construction of a shared folder system, which provides the same main features as the MVP, but not necessarily the same security guarantees.
The baseline implementation can also take inspiration from mainstream file sharing systems available on the market, to ensure that the final scheme supports all functionality required in practice.

After benchmarking the initial artifact, optimisations will be explored. 
Both the SSF MVP and the baseline implementation should initially provide a command line interface (CLI) from which the functionality can be accessed by a user.
If time permits, extensions such as a graphical interface and functionalities commonly needed to ease the use of file sharing systems will be researched and implemented.

A secondary aim of the project is to study the real-world requirements on a file sharing system in various settings.
For example, a company might have very different functionality and security demands than a group of protesters.
Since the SSF scheme will provide stronger security guarantees than existing file sharing systems, it is likely that trade-offs between functionality and security will be necessary.
Hence, another valuable extension to the project is to survey the needs of the possible users of the SSF system, how these are (not) fulfilled by existing systems, and to define the feature changes to the theoretical design that would be needed to better meet the demands a certain target audience.


\tasks
\begin{workpackage}{Background research and literature review}\label{wp:background}
Review the cryptographic primitives used by the theoretical construction. A non-exhaustive list of literature review includes the following papers and documentations:
\begin{itemize}
	\item Key Regression: Enabling Efficient Key Distribution for Secure Distributed Storage, Fu et al.~\cite{NDSS:FuKamKoh06}
	    
    \item On Ends-to-Ends Encryption: Asynchronous Group Messaging with Strong Security Guarantees, Cohn-Gordon et al.~\cite{CCS:CCGMM18}
    
    \item On the Insider Security of MLS, Alwen et al.~\cite{CCS:AHKM22}
    \item The Messaging Layer Security (MLS) Protocol, Barnes et al.~\cite{rfc9420}
    
    \item Continuous group key agreement with active security, Alwen et al.~\cite{TCC:ACJM20}
    
    \item  Cryptographic administration for secure group messaging, Balb\'as et al.~\cite{USENIX:BalColVau23}
    
    \item Continuous Group Key Agreement with Flexible Authorization and Its Applications, Kajita et al.~\cite{IWSPA:KEONO23}
    
    \item Fork-Resilient Continuous Group Key Agreement, Alwen et al.~\cite{C:AlwMulTse23}
    
    \item Server-Aided Continuous Group Key Agreement, Alwen et al.~\cite{CCS:AHKM22}
    
    \item WhatsUpp with Sender Keys? Analysis, Improvements and Security Proofs, Balb\'as et al.~\cite{AC:BalColGaj23}
    
    \item Keep the Dirt: Tainted TreeKEM, Adaptively and Actively Secure Continuous Group Key Agreement, Alwen et al.~\cite{SP:KPWKCCMYAP21}
    
    \item TreeKEM: Asynchronous Decentralized Key Management for Large Dynamic Groups A protocol proposal for Messaging Layer Security (MLS), Bharghavan et al.~\cite{TreeKEM}
\end{itemize} 

Explore available libraries implementing the above cryptographic primitives, such as:
\begin{itemize}
    \item The list of MLS implementations provided by the MLS Working Group~\cite{MLSWGimpl}
    \item AWS mls-rs~\cite{AWSMLSrs}
    \item TreeKEM~\cite{TreeKEMimpl}
\end{itemize}
Based on the above, choose the programming ecosystem, 
which language and which runtime environment to use for the implementation. 
Availability of required cryptographic primitives, 
the security deriving from such a programming language or use of libraries,
and the possibility to easily evolve the implementation into an accessible and usable product
will be crucial in making this decision.

In addition to the literature review and initial setup of the implementation, participate in the meetings and discussions on the theoretical model and contribute ideas to the SSF scheme.

\end{workpackage}

\begin{workpackage}{System requirements: functionality and security goals}\label{wp:system-reqs}
    Due to the flexibility of CGKA in combination with key regression, 
    the shared folder scheme can cater to a variety of different requirements and threat models.
    This is part of the goal of the theoretical side of this project: 
    ideally, we wish do develop a set of systems which, depending on the strength of the adversary, provides different levels of security.
    While all the systems in the set should share the same basic building blocks, they might behave rather differently on an implementation level. 
    Hence, the implementation needs to restrict the setting such that the system requirements are clear.
    
    The focus of this work package is to pick a specific threat model for the SSF scheme, which will become the (initial) focus of the implemented system.
    Additionally, the expected output is a specification of the functionality which the system should and can provide in the chosen threat model and an (informal) description of the corresponding security goals.
    In particular, specify:
    \begin{itemize}
        \item Correctness: How does a correct system works, and which functionality does it provide to users in a setting where all parties are honest?
        \item Threat model: Who are the adversaries, and what are their capabilities?
        \item Security: Which kind of attacks does the system protect against? What security guarantees does it provide to its honest users?
    \end{itemize}
    As part of this work package, survey the proposed threat models for the SSF scheme, and compare to those of other, existing file sharing systems.
    If needed, adjust the proposed threat models to make them suitable for a ``real-world'' system.
    Choose the most appropriate threat model for the implementation, and motivate the choice.
    
    A formal proof of security of the system in the chosen threat model is not expected as part of this project.
\end{workpackage}

\begin{workpackage}{Reference implementation}\label{wp:benchmark-baseline}
	Implement a suitable baseline to benchmark the secure shared folder protocol against, 
    using the programming ecosystem chosen in WP\ref{wp:background}.
    The implementation is concerned with the client library, while the server side will use a 
    mainstream commercial cloud storage.
    The baseline will include:
    \begin{itemize}
		\item setting up a folder
		\item adding new files to the folder
		\item sharing the folder with other users
	\end{itemize}
    Notice that with this set of operations we can already support, although nai\"vely, removal of 
    a member from a group. All is needed is to set up a new folder with all the users in the old group
    except the one to be removed.

    More specifically, the baseline will be implemented in the following way:
    \begin{itemize}
        \item assume the existence of a public key infrastructure (PKI)
        \item let every shared folder have a symmetric key, with which all the files in the folder are encrypted
        \item sharing is implemented by sharing the folder key over the PKI
        \item there are no advanced security guarantees
        \item in particular, there is no group management beyond adding new users, and no forward security or post-compromise security
    \end{itemize}

    Addition of functionalities or security guarantees to the baseline will be conducted when relevant to benchmark the SSF. 
    After a first baseline implementation is completed,
    particular focus will be given on exploring which security guarantees defined in WP\ref{wp:system-reqs} are achievable in the na\"ive system above and at what cost.

\end{workpackage}

\begin{workpackage}{Implementing CGKA}\label{wp:impl-cgka}
	The deliverable of this work package is an implementation of (administrated) CGKA to be used in the SSF scheme implementation.
	
    Before implementing CGKA ourselves, we will explore the possibility of using an existing library.
    CGKA is part of the MLS protocol, and the protocol itself defines an \texttt{exporter interface} which seems suitable for our purposes. 
    The interface exposes an export secret, which is essentially a PRF evaluation of the epoch secret, and also ensures proper domain separation with other secrets (init, membership etc).

    The AWS rust library~\cite{AWSMLSrs} offers a \texttt{export\_secret} method in module 
    \texttt{mls\_rs::group::Group} to produce the current exporter secret. In addition, disabling the \texttt{private\_message} feature 
    will exclude all non-CGKA related features from compilation.
The BouncyCastle~\cite{BouncyCastle} Java implementation, contained in the mls package, also seems to expose the
    low-level API for group management.
    Both libraries above are still under active development. Other solutions might be considering
    implementation of the TreeKEM protocol~\cite{TreeKEMimpl}.

    If none of the above libraries (and others that might be found) are suitable, the task is to implement an admin-CGKA scheme to be used for the SSF MVP implementation.
\end{workpackage}

\begin{workpackage}{Implementing the secure shared folder system}\label{wp:impl-ssf}
    Using the implementation of CGKA from WP\ref{wp:impl-cgka}, develop a CLI which provides all the functionalities in scope for the MVP of the SSF.
    The CLI will initially use an existing cloud storage provider for the server-side storage, but should provide flexibility for future integration with other providers. 
    Whenever possible, add resiliency and scalability to the solution without
    compromising the security of the overall system.
\end{workpackage}

\begin{workpackage}{Benchmarking}\label{wp:benchmark}
	Benchmark the SSF implementation against the baseline and analyze its efficiency and security in practice. 
    The results should make clear what part of the resources spent (e.g. time, memory, storage) 
    are consumed by the cryptographic constructions. Further, the analysis should also include
    resiliency and availability considerations, both for the baseline and the proposed system.  
    If the need arises during benchmarking, expand the baseline to be able to perform a more
    extensive comparison with the SSF system.
\end{workpackage}

\subsection{Extensions}
The extensions are not part of the main project and only complementary to the main goals.
They are not to be taken into account for the grading of the project if not delivered. 

\begin{workpackage}{Case-studies}\label{wp:extension}
	Specify the real-world functionality and security requirements of users.
	Compare to the features of existing systems, such as for example\\
	Commercial systems:
    \begin{itemize}
        \item Dropbox
        \item Google Drive
        \item Microsoft OneDrive
    \end{itemize}
    Systems in the academic literature:
    \begin{itemize}
    	\item Burnbox~\cite{USENIX:TMRM18}
    	\item SiRiUs~\cite{NDSS:GSMB03}
    	\item Plutus~\cite{USENIXSAGE:KaRiSwWaFu03}
    \end{itemize}
    Interesting questions to explore are:
    \begin{itemize}
        \item What security guarantees do these systems provide?
        Are they met by the SSF scheme?
        
        \item What scalability requirements do the above systems have? Are these requirements met by SSF?
        
        \item Which security guarantees are provided by SSF that are not provided by the above systems?
        At what cost?
        Is the usability of the system disadvantaged by the stronger security?
        
        \item What features are present in existing systems, but are missing from the SSF implementation?
        Could they be added to SSF without compromising the security?
        At what cost?
    \end{itemize}
    
    The exploration of extensions will contribute to understanding the viability of SSF to serve as a base for further constructions, and its limitations.
    The extensions could also lead to a re-design of the SSF system, adjusting it to serve the usability purposes and the various target audiences discovered during the survey.
\end{workpackage}

\begin{workpackage}{Adjusted threat model}\label{wp:adjusted-thread-model}
    Adjust the threat model and if needed also the design of the protocol to suit a selected set of requirements discovered in WP\ref{wp:extension} which could lead to real-world adoption of the system.
    Together with the new threat-model, also specify the target audience that could be interested in using the system.  
\end{workpackage}

\begin{workpackage}{Implement the new design}\label{wp:new-design}
    Implement the new system, either by extending the existing SSF if possible, or, if the protocol needs to be re-designed, by implementing the new protocol.
\end{workpackage}

\begin{workpackage}{Benchmark and compare the three implementations}\label{wp:new-implementation}
    As a final step, conduct a benchmark of the new SSF scheme, original SSF scheme and the baseline implementation.
    The goal of the benchmark is to understand how the different threat models and usability requirements (and hence system designs) influence the overall performance of the system and what the costs incurred by the different security guarantees are.
\end{workpackage}


%In the line below, use "\gradingMA" for a Master's thesis or "\gradingBA" for a Bachelor's thesis
\gradingMA




\bibliography{../cryptobib/abbrev3,../cryptobib/crypto,references}

\end{document}
